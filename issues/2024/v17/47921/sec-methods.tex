\section{Methods}\label{sec-methods}
\subsection{Overview and Demographics}\label{sub-sec-overviewanddemographics}

This exploratory Scholarship of Teaching and Learning (SoTL) joint
research project studied the cross-cultural \enquote{internationalization at
home} through a eight-week (March-May, 2022) Virtual Exchange (VE) of
students preparing to be teachers, pre-service teachers (PST) enrolled in
Second Language Acquisition (SLA) related courses in the state of
Pennsylvania in the US (18 PST) and in the Paraná state, Brazil (16
PST). The VE initiative was carried out with another group of PST
(March-April, 2023; US= 21 PST; Brazil =22 PST) with
some adjustments based on findings in 2002.

The US course is focused on English as a Second Language (ESOL). The
state certification in Pennsylvania requires that PSTs in all teaching
disciplines take courses (at least 90 hours of study) to prepare for
teaching children who immigrate to the US with minimal English skills
since these children spend 90\% of their day in non-bilingual
classrooms. The course is taken in their second or third year of their
studies. Most US PSTs were not bilingual (N=35\% bilingual) and for
those who were bilingual, predominantly the first language was Spanish
and none spoke Portuguese. The course format was asynchronous web-based
enrolling PSTs from several regions of the state.

In Brazil, the group of students were enrolled in the course \enquote{English
Teacher Education Practice} in the third year of their undergraduate
curriculum. They are from the Language Arts undergraduate Program
(English major) and they were studying to be English teachers (See \Cref{tab-01}).
		
		

\begin{table}[htpb]
\centering
\begin{threeparttable}
\caption{Courses in US and Brazil and their respective enrollments.}
\label{tab-01}
\small
\setlength{\tabcolsep}{3pt}
\begin{tabular}{*{4}{p{0.23\textwidth}}}
\toprule
\multicolumn{2}{p{0.46\textwidth}}{US Preservice Teacher (PST) Course} & \multicolumn{2}{p{0.46\textwidth}}{Brazil Preservice Teacher (PST) Course} \\
\midrule    
\multicolumn{2}{p{0.5\textwidth}}{\textbf{Course}: Intro. teaching English Language Learners (ELL)
\newline *Required course to teach all subjects/grade levels in the State in the US because most schools "Submerge" Emergent Bilingual students in General Ed classrooms.
} & 
\multicolumn{2}{p{0.5\textwidth}}{\textbf{Course}: English Teacher Education
\newline *Required course in the third year of the Language Arts Undergraduate Program.}
\\

\multicolumn{2}{p{0.5\textwidth}}{\textbf{Format}: Asynchronous Web course, (Asych. format to accommodate student sched-ules).} & \multicolumn{2}{p{0.5\textwidth}}{\textbf{Format}: In-person.}
\\
				
$\textmd{PSTs 2022} = 18$ & $\textmd{PSTs  2023} = 21$	& $\textmd{PSTs  2022} = 16$ & $\textmd{PSTs  2023} = 22$ \\
\bottomrule
\end{tabular}
\source{Own elaboration.}
\end{threeparttable}
\end{table}
		
\subsection{VE tasks implementation schedules 2022 \& 2023}\label{sub-sec-vetasksimplementation}
	
US and Brazil instructors planned the VE Project for these SLA-related
courses, such as the timeline for the exchanges, the platforms to be
used, the tasks to be developed, the topics for the students'
cross-cultural projects, the forms of assessment, the rubric, and the
selection of ICT. The interactions were developed using both synchronous
and asynchronous platforms: Google Classroom, a platform for both
classes to get updates, directions, and submit assignments for this VE
project; VoiceThread \& Padlet for the autobiographies for them to get
to know each other and Google Meet for synchronous interactions when the
students presented the results of their projects. The students were also
free to choose ICT tools that team members could use to divide up the
responsibilities for their research, collate their information, develop
their project, decide on formats for presentation, and complete the
projects (\emph{i.e.}, WhatsApp, Zoom, Google docs, Google Meet, Google
slides; Google chat, Canvas, etc.).
	
In both years, the VE projects were set up with phases \enquote{Getting to Know
Each Other} and working on collaborative projects as well as two other
phases: presentation and reflection. Teams of students (each team)
engaged in semi-structured tasks, such as Phase 1 - autobiographical
introductions to get to know each other, Phase 2 - cross-cultural
student inquiry projects with culminating presentations focused on
topics related to comparisons of teacher development and language and
culture development in each respective country, and Phase 3 - peer
evaluations of other teams' cross-cultural projects and reflections on
process and projects of their own team's cross-cultural project (see \Cref{tab-02}).
		
Professors then guided the VE Project in phases (getting-to-know,
projects, reflection/evaluation) for these courses in 2022 and then
again with the project design revised in 2023 with two other classes in
respective courses. Based on information gathered in 2022, the timeline
and tasks were adjusted for 2023 to increase the opportunities for
student interactions and for faculty mentoring through periodic
formative activities and virtual meetings (see \Cref{tab-02}).

\begin{table}[htpb]
\centering 
\begin{threeparttable}
\caption{VE tasks implementation schedules 2022 \& 2023.}
\label{tab-02}
\small
\setlength{\tabcolsep}{3pt}
\begin{tabular}{*{2}{p{0.48\textwidth}}}
\toprule
\multicolumn{1}{c}{2022} & \multicolumn{1}{c}{2023}\\
\midrule
\textbf{Week 1}: PSTs were assigned to one of four teams for Cross-Cultural Student Inquiry projects and the professors explained the project.
The topics of their projects were: 1) K-12th Grade language teaching \& learning in Brazil \& USA; 2) Bilingual K-12th Gr. education in Brazil \& US; 3) The status of foreign languages in Brazil \& US; 4) How teacher education programs in Brazil region \& US state prepare to teach/ support English/other languages of K-12th Gr.
During this first week, the US professor participated in one of the Brazil classroom meetings and the professors explained the project in a synchronous class for Brazil students. This was recorded and the video was uploaded to Google classroom for the US students to view since their class was asynchronous. &
									
\textbf{Week 1}: PST self-selected which team to join (to promote ownership of the task) to one of five teams.
Added a fifth topic to accommodate larger classes in 2023:  Contextual information about Brazil and State in US \& their Education System.
Added Watch Video general information about what is COIL/ Virtual Exchange \& our Cross-Cultural Collaborations (Video explanation); Read short article; then Post reflection.
Added a Padlet Intro. activity - post a picture of yourself and about 50 words to describe yourself.
Repeated the virtual meeting with professors and Brazil students; four US students attended \& others watched the recording. \\
					
\textbf{Week 2}: Students asynchronously exchanged and commented on each others’ VoiceThread autobiographies to get to know each other (Per faculty directions: on how to use VoiceThread, required content, and grading rubrics). &
			
\textbf{Week 2}: Autobiography as 2022. \\
				
\textbf{Week 3}: Student-lead \& faculty-guided discussions to organize their plans for carrying out their cross-cultural inquiry projects (Per faculty directions: on required content and grading rubrics). &
					
\textbf{Week 3}: – Repeated the discussions.
Added a Periodic Progress 1: Semi-structured Narrative post: Students shared their perceptions of the Introduc-tions activities of Week 1 \& 2). \\
										
\textbf{Weeks 4 and 5}: Students worked on their Cross-Cultural Collaboration pro-jects; Each team conducted library and internet research on the selected topic (Students chose ICT for meetings, planning, and collaboration on the project content). & 
					
\textbf{Weeks 4 \& 5}: Students worked on their Cross-Cultural Collaboration projects in similar ways to 2022.  
Added a Periodic Progress 2: Semi-structured Narrative post: Students shared their perceptions of their collaborative process, working in teams on the project. \\
					
\textbf{Week 6}: Synchronous Presentations (all chose to use PowerPoint). &
									
\textbf{Week 6}: Synchronous Presentations (PowerPoint or Canvas). \\
					
\textbf{Week 7}: Semi-structured Peer evaluations of other teams' cross-cultural projects, and reflections through a survey on the process and projects of their own team's cross-cultural project. &
					
\textbf{Week 7}: Semi-structured survey on the process and projects of their own team's cross-cultural project.
Added a Final Narrative of an aspect of the project and included a picture symbolic of the experience. \\
					
\textbf{Week 8}: Faculty met to collaboratively grade, using the grading rubrics and analyzed the projects regarding related ICT used for content, pictures, Live Google meets presentations, and organization. &
					
\textbf{Week 8}: Faculty met to collaboratively grade, using the grading rubrics and analyze the projects regarding related ICT used for content, pictures, Live Google meets presentations, and organization.\\
\bottomrule
\end{tabular}
\source{Own elaboration.}
\end{threeparttable}
\end{table}

		
\subsection{Data \& Analysis}\label{sub-sec-dataandanalysis}
		
The general aim of our research project is to investigate an experience
of virtual exchange (VE) with PSTs from the state of Pennsylvania, US
and the state of Paraná, Brazil in terms of teacher development,
intercultural communicative competence (ICC), language practices, local
and global perspectives, and collaborative practices through information
and communication technologies (ICT).

In this text, we focus on the outcomes and challenges of the VE,
analyzing data from periodic and end-of-semester surveys and narratives
as well as from the PST interactions and tasks in different digital
platforms.

For this qualitative case study research method to investigate this
cross-cultural collaboration experience between the two SLA classes
\cite{stake2005qualitative}, to explore the research aims of this paper, researchers
independently searched for patterns and themes in the
participants’ interactions in the surveys \& narratives
of experiences, autobiographies, cross-cultural inquiry projects,
experiences with ICT, faculty notes during the presentations, grading
rubrics, and faculty reflections on the projects.

Each PST is designated by country (US or BR) and year participated. Each
was also assigned by team and an individual number, i.e. US 2022
ST1(team)-6(individual). Coded data were analyzed using a Constant
Comparative Analysis Method \cite{glaser1967discovery} with triangulation
\cite{denzin2000discipline} of independently coded data, in-depth analysis
of the existing research, and narratives to explain the findings.
		
		
		
		
