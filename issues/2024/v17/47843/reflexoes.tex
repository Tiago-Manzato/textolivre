\section{Algumas reflexões sobre leitura e a sala de aula}\label{sec-algumasreflexoessobreleituraeasaladeaula}


\textcite[p. 123]{carr2010} nos traz ainda dados sobre o funcionamento do cérebro em relação à atividade de leitura: a leitura de um livro causa menor estímulo aos sentidos do que o uso de computadores e, por esse motivo, essa atividade é mais produtiva intelectualmente. Por nos permitir filtrar e deixar de fora distrações e aquietar as funções do lobo frontal de solução de problemas, a leitura profunda se torna um tipo de raciocínio profundo. A mente de um leitor de livros experiente é uma mente calma, não barulhenta. 

\textcite{wolf2019} também endereça essa questão, nos informando que a habilidade de ler cria novos circuitos em nosso cérebro e que quando uma criança adquire a leitura fluente, o caminho de sinais no cérebro muda \cite{bbc2021}, passando a fazer um trajeto mais rápido e eficiente e, por isso, permitindo ao leitor maior integração entre seus sentimentos e pensamentos à própria experiência. Entra-se aí num ponto muito importante, que vai além. Não é apenas o raciocínio profundo que requer uma mente calma e atenta. Psicólogos têm estudado as fontes de instintos mais nobres no ser humano, como a compaixão e a empatia, para descobrir que essas emoções mais nobres emergem de processos neurais que são inerentemente lentos \cite[p. 220]{carr2010}. Experimentos mostram que o cérebro reage rapidamente a estímulos de dor física, mas, por outro lado, os processos mentais necessários para gerarem empatia em relação ao sofrimento mental ou psíquico se desenvolvem com muito mais lentidão. Ou seja, quanto mais distraídos nos tornamos, menos capazes ficamos de experimentar emoções mais sutis. Como não relacionar a esse dado algumas das mais relevantes questões referentes ao aumento da violência que se tem visto?

Assim, tem-se, de um lado, o estímulo lento da leitura sem distrações que livros proporcionam e, do outro, o tipo de estímulo mental mais intenso proporcionado pela leitura online, com \textit{hiperlinks} e propagandas \textit{pop up}, pelas redes sociais e jogos aos quais os estudantes (e os sujeitos contemporâneos em geral) estão muito acostumados. Estes certamente são e serão muito úteis para o desempenho de funções profissionais e pessoais nesse novo mundo em que já nos encontramos, mas parece se tratar de um tipo de habilidade que os jovens são capazes de desenvolver por si mesmos, como se pode observar empiricamente. Por outro lado, o tipo de habilidade e de benefícios decorrentes dela que a leitura de livros, a leitura quieta pode trazer aos jovens não é de aprendizado tão intuitivo (ou tão prazeroso?) como o uso dos games e o consumo de informações online. A construção e o incentivo da continuidade dessa aquisição é tarefa da escola e das gerações, digamos, “mais analógicas”. Para \textcite[p. 176–177]{wolf2019}, a ênfase na leitura deve ser sobre a importância do significado, em vez de velocidade, e também sobre como fazer boas decisões a respeito do conteúdo e da autoregulação da própria atenção.

Vale lembrar aqui que há diversos tipos de leitura, e que o uso intenso das redes sociais e de aplicativos de trocas de mensagens instantâneas tem direcionado as pessoas para as leituras mais superficiais e rápidas, do tipo escaneamento do texto, à busca de uma ou outra informação. Entretanto, para se compreender conceitos, relacionados ou não a determinada disciplina, ou mesmo uma compreensão melhor do que se lê, essa leitura não serve. \textcite[p. 177]{wolf2019} coloca como objetivo que se ajude os jovens leitores a terem o que ela chama de “um cérebro bilíngue”, capaz de empenhar tempo e atenção à leitura profunda em qualquer que seja o suporte utilizado, digital ou analógico. É bom também lembrar que a escola vem de uma tradição iluminista, da palavra, da reflexão, do tempo \cite{sibilia2012}, e os alunos estão na época das imagens, da velocidade, das opiniões que tomam o lugar do conhecimento. Essa diferença precisa ser acolhida e considerada não como um impedimento, mas como algo a ser trabalhado, integrado, buscando-se esse equilíbrio “bilíngue”.

Pensamos que, no processo de busca de informações e de conhecimento, a profusão de fontes de conhecimento e o grande número delas que estão disponíveis para consulta geram confusão e levam à impossibilidade de construção de significado para a informação – pensando em construção de significado como derivada de pensamento reflexivo e profundidade. Assim, à escola caberia guiar mais de perto seus alunos nessa aquisição. Sem supervisionamento próximo, os estudantes muito provavelmente vão buscar informações sobre algum assunto no Google ou em ferramenta semelhante e encontrarão uma infinidade de fontes e de diferentes versões para um mesmo conteúdo, o que dificulta a identificação do que pode ser mais bem aproveitado e leva-os a se sentirem perdidos e desestimulados. O excesso de opções frustra e pode levar à desistência. Como processo de aprendizado, poderia ser mais proveitoso indicar duas ou três fontes para que, a partir delas, eles elejam mais duas ou três para pesquisar (o que já é um número grande, em se tratando de jovens leitores pesquisadores). Desse modo, haverá mais segurança no início da caminhada e redução da oportunidade de aquisição de pseudo conhecimento de fontes que não têm credenciais para tanto\footnote{Desenvolvemos essa ideia em \textcite{noronha2019}.}. 

Voltando às características apontadas por \textcite{lash2012} no início deste texto, o desaparecimento da noção de profundidade modifica a relação entre escola e alunos, entre escola e sociedade, entre o saber e o fazer. Tem-se a percepção de que as coisas se colocam em um mesmo patamar, dificultando o estabelecimento de hierarquias de importância em relação ao conhecimento. Trata-se de um desafio para toda a comunidade acadêmica, na medida em que os alunos passam a enxergar o conhecimento acadêmico como igual ou mesmo inferior, porque menos extenso, do que o de máquinas de busca como o Google. Enxerga-se a extensão do conhecimento de um buscador / indexador de informações como o Google, mas não se enxerga sua falta de profundidade, o que, eventualmente, poderia mostrar o quanto desse conhecimento é ou não acertado. Ao mostrar as diferenças entre o mundo físico e o digital, entre o tipo de leitura que se faz em um e em outro, com finalidades diferentes, entre fontes de conhecimento disponíveis, justificando-as, se está apresentando aos jovens o mundo que existe de um modo que eles conhecem menos ou mesmo desconhecem e que, por desconhecerem, se equivocam achando que podem prescindir dele. 