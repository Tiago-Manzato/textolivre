\section{Caracterizando as formas tecnológicas de vida}\label{sec-caracterizandoasformastecnologicasdevida}

Assim, coloca-se em questão uma sociedade que está conectada o tempo todo, na qual a leitura de um livro, em momento de desconexão, é considerada \enquote{excêntrica}. Para refletir sobre ela, recorremos primeiramente a um clássico nos estudos de sociologia sobre cultura e sociedade tecnológicas, \enquote{Formas Tecnológicas de Vida}, de \textcite{lash2012}, que nos permite pensar as características de uma sociedade cujas formas de vida se tornaram tecnológicas, ou seja, cujo sentido que se atribui às coisas se dá por meio de sistemas tecnológicos. 

\textcite{lash2012} explica que a ideia de forma de vida, provinda de Wittgenstein, é fenomenológica, donde os sentidos advêm do corpo, da percepção, da experiência. O sujeito cartesiano da razão dá lugar ao sujeito husserliano que percebe, que é mais intuitivo e que apreende o sentido na imediato da experiência. Nas formas de vida, o conhecimento se situa no mundo, em estruturas ontológicas profundas.

Daí que para as formas tecnológicas de vida atuamos como interfaces homem-máquina. Essas são formas que operam à distância. Os dados que antes eram armazenados internamente, na memória humana, passam a existir externamente, à distância, em bancos de dados digitais. O corpo passa a operar como um sistema aberto, porque em contato permanente com ferramentas, em vez de fechado. Acoplamos ferramentas ao funcionamento do corpo, como uma vez foram o machado, o martelo, passaram ao relógio de pulso, ao telefone e hoje são o \textit{smartphone}, o teclado do computador e a tela que se torna uma extensão dos olhos \cite{pang2013}. Quando esse sistema (humano) se abre, seus órgãos ficam expostos aos fluxos de informação e comunicação. 

Segundo \textcite{lash2012}, com a tecnologia, as formas de vida apresentam três características: se achatam (ou aplanam); se tornam não-lineares; e se tornam suspensas.

Sobre o achatamento, \textcite{lash2012} traz que deixa de existir a verticalidade de sujeito e objeto, classificador e classificado, particular e universal. Desaparece o sentido profundo, a dimensão transcendental que antes era encontrada na religião e na psicanálise, por exemplo, para darem lugar ao sentido empírico, cotidiano, contingente. Desaparece a distância entre o conhecimento e a prática e o fazer passa a ser sinônimo de saber. A atribuição de sentido deixa de ser um atributo de interioridade, fruto da reflexão e passa a ser comunicação: pensar é comunicar-se.

A respeito da não-linearidade, ela envolve compressão, aceleração e descontinuidade.  Na compressão, as unidades lineares de sentido, como a narrativa e o discurso, são comprimidas e convertidas em unidades de informação e comunicação. Os textos se tornam cada vez mais breves e as imagens tornam-se onipresentes, muitas vezes substituindo-os. Eles também se tornam cada vez mais breves, e o sincretismo, junção de texto imagético e verbal que vemos, por exemplo, nos memes, predominam em lugares onde antes havia o que hoje é, pejorativamente, chamado de “textão”. Nas empresas, relatórios dão lugar a apresentações de slides. Tem-se assim a aceleração: as formas tecnológicas de vida são rápidas demais para a reflexão e aceleradas demais para linearidade. Tudo se torna efêmero. A invenção é tão veloz que ultrapassa a lógica da causa-e-efeito. O colapso do tempo linear nos traz a sociedade do risco, cujo olhar é sempre voltado para o futuro. No raciocínio causal, olha-se para o passado para se buscar explicar o presente. No raciocínio de consequências, o presente é olhado como aquilo que causa riscos no futuro \cite{lash2012}. E, por fim, sobre a descontinuidade, deixa-se de ter linhas que conectam (como nos cabos telefônicos de outrora), e passa-se a ter redes de relações, tênues, que são percorridas de modo descontínuo, saltitando-se em várias direções ao mesmo tempo. Não há mais um direcionamento, mas múltiplos.

A terceira característica que as formas de vida tecnológica trazem, segundo \textcite{lash2012} é a suspensão ou o desalojamento. Enquanto suspensas, as formas de vida tecnológicas absorvem menos a característica de um lugar particular, podendo ser de todos os lugares ou de lugar nenhum. Caracteriza-se, assim, não tanto por uma multiplicidade de identidades, mas por uma ausência delas: não há identidade. A internet é um espaço genérico, que não se localiza em lugar nenhum. Quanto ao mundo físico, como se ele passasse a espelhar algumas características desse mundo online, as diferenças também vão se atenuando. Uma loja da Ikea (loja de móveis finlandesa) ou um restaurante do McDonald’s (rede norteamericana de \textit{fast food}) são (praticamente) os mesmos em Londres, São Paulo ou Tóquio.

A partir dessas três grandes características apontadas por \textcite{lash2012}, pode-se refletir sobre a “revolução digital” que vivemos e perguntar-nos como fazer sentido na compreensão do mundo que se dá, hoje, primordialmente, por meio da tecnologia digital. Um mundo que se apresenta para nós em funcionamento “vinte e quatro por sete”, ou seja, 24 horas por dia, sete dias por semana \cite{crary}. Segundo \textcite{sibilia2020} vivemos um tempo de hiperconectividade e de abundância de atividades e de consumo. Vivemos sob incessante suspeita de que há, em algum outro lugar, algo mais interessante, mais divertido, mais útil ou mais imprescindível para fazer, ler, comentar, compartilhar... Isso leva à exaustão, por buscarmos sempre mais, e à frustração, por nunca conseguirmos, pois se trata de algo impossível. A velocidade acelerada garante que o contato permaneça superficial. E a falta de limites avança até mesmo sobre nosso sono: dormir é visto como “perda de tempo” (e veremos logo mais como isso se liga com outras questões também pertinentes ao modo de vida contemporâneo).
