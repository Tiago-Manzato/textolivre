\section{Introdução}\label{sec-introdução}
O canal Porta dos Fundos tem um vídeo chamado Excêntrico, de 2015. Ele se inicia com um homem sentado em seu sofá quando seus amigos entram assustados em seu apartamento, trazendo bombeiros e outros homens, em sobressalto fazendo parecer que precisaram arrombar a porta. Eles estavam preocupados porque esse homem não havia respondido a um \textit{e-mail} que havia sido enviado há mais de duas horas, nem respondido a mensagens em diversas redes sociais nesse meio tempo. Os amigos dizem que achavam que ele poderia ter sido sequestrado porque, afinal, quem é que fica \enquote{tanto tempo} offline atualmente? O homem estava em casa, lendo um livro, motivo pelo qual é chamado de \textit{vintage} por um dos colegas. 

\enquote{Excêntrico}, o título do vídeo, refere-se, segundo o Dicionário Aurélio da Língua Portuguesa, ao indivíduo original, extravagante ou esquisito. Depreende-se daí que, na sociedade em que vivemos, desconectar-se é ser excêntrico e, por oposição, que o padrão estabelecido seja o da conexão ininterrupta. Este é o cenário que nos cerca hoje, no século XXI. Cabe ainda observar que o vídeo foi feito em 2015, quase uma década atrás, e desde então essa conexão tem se intensificado e se estendido para mais âmbitos da vida, em especial depois de 2020, como efeito da pandemia de Covid-19.
