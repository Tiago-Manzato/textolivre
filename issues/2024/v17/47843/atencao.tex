\section{A economia da atenção}\label{sec-aeconomiadaatenção}

No seio da esfera cultural reconfigurada pela internet, a atenção se tornou o bem mais raro \cite{citton2013}. Antes percebida como algo garantido, ou mesmo podemos nos arriscar a dizer que ela não era muito levada em consideração, a atenção agora entra no lugar dos bens e dos serviços como sendo aquilo que tem o maior valor. Tomando-se economia como o tratamento da gestão e da produção dos bens que são escassos, tem-se que, hoje, o que é escasso não é a informação, mas a atenção que ela demanda. Tem-se, assim, que na chamada sociedade da informação a economia seja a da atenção, pois esta é o bem a ser gerido, uma vez que a informação é não apenas abundante, mas parece mesmo tender ao infinito. 

Os grandes conglomerados tecnológicos sabem disso há algum tempo:

\begin{quote}
O Dr. Eric Schmidt é o CEO e presidente da Google, [\ldots]. Dirigindo-se a outros executivos de alta tecnologia há alguns anos, ele declarou que o século 21 será sinônimo de \enquote{\textbf{economia da atenção}} e que os vencedores serão aqueles que conseguirem maximizar o número de \enquote{globos oculares} que puderem controlar consistentemente. O objetivo aqui é a interface contínua, não literalmente contínua, mas um envolvimento relativamente ininterrupto com telas iluminadas de diversos tipos que incessantemente solicitam interesse ou resposta. \cite[s/p tradução e negrito nossos]{crary}.\footnote{Texto original: Dr Eric Schmidt is the CEO and chairman of Google [\ldots]. Addressing other high-tech executives a few years ago, he declared that the 21st-century will be synonymous with the ‘attention economy’ and that the winners will be those who succeed in maximizing the number of ‘eyeballs’ they can consistently control. The goal here is the continuous interface, not literally seamless, but a relatively unbroken engagement with illuminated screens of diverse kinds that ceaselessly solicit interest or response.}

\end{quote}

Os bens culturais e de entretenimento são hoje oferecidos abundantemente, muitas vezes também de forma aparentemente gratuita para o consumidor. Na sociedade em que vivemos, no entanto, um sujeito precisa aprender a gerir sua atenção, que é insuficiente para todas as demandas que chegam até si e que é demandada incessante e insistentemente dos mais diversos modos. Vive-se sob um verdadeiro bombardeio de estímulos e apelos. Os consumidores, detentores desse precioso ativo, não se dão conta dessa situação, distraídos que estão e acabam sendo, perdão pelo lugar comum, consumidos. A disputa é pelo tempo e pela atenção.

Sobre a interface contínua dos serviços denunciada por Crary no excerto acima, é fácil encontrar exemplos ao nosso redor. Os serviços de TV por assinatura deram lugar a serviços de internet, chamados \textit{streaming}\footnote{A palavra \textit{streaming}, em inglês, vem de \enquote{stream}, traduzida como riacho ou córrego, ou seja, um fluxo de água que não para. Um ótimo exemplo de termo que figurativiza em si a ideia que designa.}, disponíveis o tempo todo. Suas ações visam à não-interrupção, por exemplo, a Netflix (e outros provedores de entretenimento desse tipo) colocam automaticamente o próximo episódio de uma série para rodar, ou a rede social Instagram que criou o \textit{stories}, que mostra o conteúdo continuamente, sem a necessidade de o usuário tocar a tela. O sujeito, desse modo, fica enredado nessas demandas por atenção contínua, passivamente. Para pensarmos na diferença na gestão do tempo, de outrora e de agora: as crianças não têm mais horário para assistir desenho, pois há canais que os exigem o tempo todo ou eles estão disponíveis nos serviços de \textit{streaming}. Há duas décadas, era necessário esperar pelo horário certo. O tempo da espera foi, assim, suprimido, do mesmo modo que o “início, meio e fim” de uma narrativa ou de uma música, antes bem-marcados, que hoje já não são mais tão facilmente perceptíveis e, em alguns casos, são mesmo inexistentes. Quais as consequências dessa ausência de marcas na temporalidade da narrativa para as gerações que se criam dentro desses modos de vida? Há também uma questão decorrente desse uso da tecnologia que diz respeito a repertório compartilhado, como outras gerações tiveram cantores, apresentadores de TV e programas televisivos que eram em muito menor número e, portanto, comuns a um número maior de pessoas. Qual é o universo comum que se forma entre os jovens da contemporaneidade e quais as consequências de compartilhar (ou não) um repertório no aprendizado e na identidade de uma geração? São perguntas que ainda não podem ser respondidas de modo mais completo e que colocam-se no horizonte de investigação.

A propósito do modo de se perceber o mundo, o que está diretamente ligado à atenção, pode-se buscar pensá-la a partir do seu contrário, a dispersão. Em seu livro \citetitle{carr2010}, de \textcite{carr2010}\footnote{Traduzido no Brasil como "A geração superficial" e lançado em 2011 pela editora Agir.}, aponta que dentre os paradoxos mais importantes que a internet envolve estava o fato de que ela captura nossa atenção apenas para dispersá-la. Nosso foco fica no meio de comunicação, na tela brilhante e atrativa, mas o tiroteio de mensagens e de estímulos nos deixa dispersos. Não se trata do modo de dispersão temporária que tem o propósito de revigorar a mente quando estamos tentando tomar uma decisão, mas uma dispersão vazia ou mesmo negativa. Para \textcite[p. 119, tradução nossa]{carr2010}:

\begin{quote}
    A cacofonia da internet de estímulos de curto-circuito de pensamentos conscientes e inconscientes impedem nossas mentes de pensar profunda ou criativamente. Nossos cérebros se tornam simples unidades de processamento de sinais, rapidamente encaminhando informação para a consciência e depois dispersando-a\footnote{The Net’s cacophony of stimuli short-circuits both conscious and unconscious thought, preventing ourminds from thinking either deeply or creatively. Or brains turn into simple signal-processing units, quickly shepherding information into consciousness and then back out again.}.
\end{quote}
 

Essa primeira percepção de Carr sobre a importância de se olhar para o paradoxo atenção e dispersão tem sido, mais recentemente, corroborada por neurocientistas, como se vê a seguir.