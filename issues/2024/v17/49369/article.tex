\documentclass[english]{textolivre}

% metadata
\journalname{Texto Livre}
\thevolume{17}
%\thenumber{1} % old template
\theyear{2024}
\receiveddate{\DTMdisplaydate{2024}{5}{20}{-1}}
\accepteddate{\DTMdisplaydate{2024}{5}{20}{-1}}
\publisheddate{\DTMdisplaydate{2024}{5}{20}{-1}}
\corrauthor{Francisco Javier Olivar de Julián}
\articledoi{10.1590/1983-3652.2024.49369}
%\articleid{NNNN} % if the article ID is not the last 5 numbers of its DOI, provide it using \articleid commmand 
% list of available sesscions in the journal: articles, dossier, reports, essays, reviews, interviews, editorial
\articlesessionname{articles}
\runningauthor{Olivar de Julián}
%\editorname{Leonardo Araújo} % old template
\sectioneditorname{Daniervelin Pereira}
\layouteditorname{João Mesquita}

\title{Automated journalism in news about traffic accidents: emergency services information dump}
\othertitle{Jornalismo automatizado em notícias sobre acidentes de trânsito: despejo de informações de serviços de emergência}

\author[1]{Francisco Javier Olivar de Julián~\orcid{0000-0002-2030-2458}\thanks{Email: \href{mailto:franciscojavier.olivar@unir.net}{franciscojavier.olivar@unir.net}}}

\affil[1]{Universidad Internacional de La Rioja, Facultad de Empresa y Comunicación / ESIT, Logroño, La Rioja, España.}

\addbibresource{article.bib}

\begin{document}
\maketitle
\begin{polyabstract}
\begin{abstract}
In this work, the process of producing news about
events in the Spanish digital press has been studied. A content analysis
was carried out on news related to the main external causes of mortality
(traffic accidents, accidental falls, drowning and suicides) published
in the six main Spanish digital media (\url{elpais.com}, \url{ elmundo.es}, \url{abc.es}, \url{lavanguardia.com}, \url{elconfidencial.com} and \url{20minutos.es})
during the period 2010-2020. The selected news items ($n=5,727$) were
obtained through the digital newspaper library \emph{Mynewsonline}. A
qualitative study has also been carried out (in-depth interviews with
professionals and a focus group). The results show a high number of news
items published on traffic compared to other external causes of death
with higher mortality. It is also appreciated that the media
 automatically transfer the information they receive from the emergency
services. It has even been detected that most of the news events coded
in this study are exact copies that are published on the same day in
different media (headlines and the body of the news).

\keywords{Automated journalism \sep News \sep Traffic accidents \sep Digital press \sep Mynewsonline}
\end{abstract}

\begin{portuguese}
\begin{abstract}
Neste trabalho foi estudado o processo de produção de
notícias sobre acontecimentos na imprensa digital espanhola. Foi
realizada uma análise de conteúdo de notícias relacionadas às principais
causas externas de mortalidade (acidentes de trânsito, quedas
acidentais, afogamentos e suicídios) publicadas nos seis principais
meios de comunicação digitais espanhóis (\url{elpais.com}, \url{ elmundo.es}, \url{abc.es}, \url{lavanguardia.com}, \url{elconfidencial.com} e \url{20minutos.es})
durante o período 2010-2020. As notícias selecionadas ($n=5.727$) foram
obtidas por meio da joalheria digital \emph{Mynewsonline}. Também foi
realizado um estudo qualitativo (entrevistas em profundidade com
profissionais e grupo focal). Os resultados mostram elevado número de
notícias publicadas sobre trânsito em comparação com outras causas
externas de morte com maior mortalidade. Também é apreciado que os meios
de comunicação transfiram automaticamente as informações que recebem dos
serviços de emergência. Foi ainda detectado que a maior parte das
notícias codificadas neste estudo são cópias exatas que são publicadas
no mesmo dia em diferentes meios de comunicação (manchetes e corpo da
notícia).

\keywords{Jornalismo automatizado \sep Notícias \sep Acidentes de trânsito \sep Imprensa digital \sep Mynewsonline}
\end{abstract}
\end{portuguese}
\end{polyabstract}

\section{Introduction}\label{sec-intro}

Previous studies by Institutions of Higher Education (IHE) found ongoing
participation in online academic events, webinars, and Virtual Exchange
(VE) continued beyond the COVID-19 pandemic \cite{bowen2021virtual,garces2020upscaling,woicolesco2022internationalization}. Virtual
Exchange (VE) includes the engagement of groups of learners in extended
periods of online intercultural interaction and collaboration with
international peers as an integrated part of their educational programs
and under the guidance of educators and/or facilitators \cite[p.~1]{garces2020upscaling}. For students unable to physically travel for financial or
personal reasons, participation in virtual exchanges serves to
complement physical travel through cross-cultural \enquote{internationalization
at home} (IaH) and provide opportunities to develop intercultural
communicative competence and skills \cite{beelen2015redefining}.

One of these VE approaches is the Collaborative Online International
Learning (COIL) which brings together classes with a shared syllabus and
projects in which participants from universities in different parts of
the world come to collaborate, usually for four to eight weeks and
embedded as part of the classwork. Participants have opportunities to
discuss course materials, solve a problem, compare cultural norms, and
often create a gradable project. Participants can interact synchronously
(in real time) or asynchronous (not in real time) and can engage with
email, in a learning management platform (i.e., Google Classroom),
voice, social media, Padlet, or any other form of Information and
Communications Technology abbreviated as ICT \cite{odowd2018telecollaboration}.

This paper focuses on a VE/COIL project developed by the researchers in
2022 and 2023. The Brazilian professor, when looking for others
interested in projects based on the COIL model, found the corresponding
at a university in the US. The professors saw each
other’s interests and the VE director at the US
university arranged an E-meeting to discuss potential cross-cultural
collaborations. After a series of exchanges and realized common goals,
professors utilized their knowledge of VE task design \cite{kurek2017task} and this VE project was developed and implemented.

This exploratory Scholarship of Teaching and Learning (SoTL) qualitative
research project \cite{stake2005qualitative} studied the VE of students preparing to
be teachers (called Preservice Teachers, abbreviated at PST) who were
enrolled in Second Language Acquisition (SLA) related courses, one in
the US and another in Brazil. In the two years of the VE Project, there
were 77 students participating in the initiative. PST in the US were
studying how to teach children in their classes who had immigrated to
the US and whose first language was not English. PST in Brazil were
studying to be English teachers.

In both years (2022 and 2023), professors planned an eight-week VE
Project embedded into our existing courses considering: the timeline for
the exchanges, the platforms to be used, the tasks to be developed,
introductory activities, the topics for the students'
cross-cultural projects, the forms of assessment, and the rubric. In
2022 with a class in the US and a class in Brazil and then again with
the research design revised in 2023 with two other classes of students
in respective courses, professors organized collaborative teams of PST
(in each team, PSTs from Brazil and US working together).

The aims of the joint research project\footnote{The research project was
	approved by the Ethics Committee of both universities.} carried out by
the authors of this paper were a) to develop and study a collaborative
inquiry project between students and professors from US and Brazil; b)
to study participants' virtual exchange experience and knowledge on
teacher education; language teaching and learning in both countries; c)
to analyze the development of the participants' intercultural
communicative competence (ICC) and learning and d) the investigation of
the different resources they used when communicating with each other and
the benefits and challenges of information and communication
technologies (ICT) for the VE experience.

In this text, we focus on the outcomes and challenges of the VE,
analyzing data from periodic and end-of-semester surveys and narratives
as well as from the PST interactions and tasks in different digital
platforms.
\section{Theoretical background}\label{sec-theoreticalbakcground}

The use of automation in the production of journalistic articles
confronts journalism with threats, opportunities, and ambiguities.
Research has been found indicating that news articles written by humans
differ from news articles written through automation processes. News
written using algorithms shares some similarities with news written by
humans, such as the focus on current events or the use of the inverted
pyramid. But there are also differences in terms of news value; Articles
written by humans tend to exhibit more negativity and impact than
articles written by algorithms. News articles written by humans are more
likely to include interpretation, while articles written by algorithms
tend to be shorter and do not use human sources \cite{tandoc2022noticias}.

Studies have also been found that review the practice of automated
journalism and identify an important limitation on the potential of
automating journalistic writing, such as the absence of sufficient data
models to encode the journalistic knowledge necessary to automatically
write narratives based on events \cite{caswell2018automated}.

On the other hand, in the interview with Eric Aislan Antonelo \cite{rocha2019inteligencia} he warns of the existence of AI, as an emulation of the
intelligence of human beings with applications such as automatic
translation, voice recognition and question and answer systems.

Likewise, articles have also been found based on the discourse of the
information values of the elites and the personalization in the source
of information \cite{manoso2020news} and journalism studies inspired by
Bourdieu that try to explain how journalists make sense of the change
related to technology in the journalistic field \cite{lindblom2022digitalizing}. This seems to link with the idea of the precariousness
to which journalists have been subjected, the main consequence of which
has been to turn their profession into a mere transmitter of messages
instead of being a builder of news with the proper framing work.

\section{Methodology}\label{sec-methodology}

This work follows the recommendations that recommend the combined use of
quantitative and qualitative methodologies in studies of a social
nature. This mixed methodology provides a broader vision for the study
on the media treatment of an issue or event \cite{jensen2015comunicacion}, since it
reduces the limitations that would entail exclusively analyzing
subjectivity with the qualitative methodology and objectivity with the
quantitative one \cite{sanchezgomez2015}.

Therefore, this study combines a quantitative content analysis of the
informative treatment offered by the Spanish digital media on traffic
accidents and other main events with a qualitative analysis that
assesses, through in-depth interviews and a discussion group, the
perception on the publication of news about events according to the
different causes that motivate them. In this way, it is intended to
obtain final information that offers a more complete vision of the
object of study.

\subsection{Sample selection}\label{sub-sec-sampleselection}

The chosen unit of analysis is the journalistic piece published by the
selected digital newspapers, related to the causes of accidents
indicated in the study interval (2010-2020).

The choice of digital media to carry out this research is not by chance,
since more and more press is being consumed on the internet and less on
paper \cite{boasberg2019marco}.

To determine the sample, a mixture of immigrant and digital native
newspapers has been selected, considering that the combination of both
will collect the sample of all online newspapers with greater fidelity.
Digital immigrant newspapers are those that have made an \enquote{adaptation of
traditional newspapers to new digital media and their interface}
\cite[p. 27]{peña2016european}. On the other hand, in this work digital native newspapers are understood to be those that were directly born in the digital sphere, also considering
those that have become digital in a period not exceeding five years from
their birth. An example of this is \emph{20minutos.es}, which was born
on February 3, 2000, as a traditional newspaper and became digital in
2005, the year in which it became the first Spanish newspaper to have a
Creative Commons license (which allows copying literally their news
citing the source). It was also the first online newspaper to open all
its contents to the comments of its readers \cite{lopez2012tratamiento}.

Following criteria of relevance and popularity for this type of media
(number of visits/month and number of unique users), the digital
immigrant newspapers, \emph{elpais.com, elmundo.es, abc.es} and
\emph{lavanguardia.com} have been selected. and the digital native
newspapers \emph{elconfidencial.com} and \emph{20minutos.es}.

The audience data has been obtained from the General Media Study
\cite{boasberg2019marco} and from the internet
marketing research company \textcite{comscore2017rating}. These are entities that
offer data from offline and online media and that are the usual pattern
as data sources for both national \cite{galletero2018estudio} and international studies \cite{potvin2018frequency}.

The exclusion/inclusion criteria have been defined according to the
recommendations followed for other types of research, such as that of
\textcite{zimmermann2019content}.

Only news related to traffic accidents, drowning, accidental falls, and
suicides in the 2010-2020 study period were coded. News about accidents
with or without victims published by the media selected for the sample
and related to events that occurred anywhere in the world, were
included. Those journalistic pieces that had a character of current
event were considered valid. This current circumstance refers to the
present time but also to the immediate past, where the news relates an
event that is happening or that has just happened. For this reason, news
of an event that occurred within the same day of publication or the day
immediately before were included.

All those pieces not related to one (or several) specific aspects have
been left out of the study due to the causes analyzed.

\subsection{Extraction method}\label{sub-sec-extractionmethod}

The sample was retrieved from the digital newspaper library
\emph{Mynewsonline}, which includes material published since 2010. This
tool has been used in another research works that required a
chronological sample \cite{repiso2018universidades,garcia2018quality}. The keywords used in the search were the causes
of mortality (suicide, falls, drownings and traffic accidents).
According to this search procedure, a total of 46,987 news items were
collected for the period 2010-2015 alone. This volume of data, due to
its large size, was considered unfeasible to handle in an individual
study, so it was decided to opt for a constructed week sampling \cite{hester2007efficiency}, a scientific methodology commonly used in the field
of Communication used, for example, in the work of Valenzuela, \cite{valenzuela2017behavioral}.

For the construction of these weeks, the random number generator Random
Integer Set Generator has been used in this study:
(\url{https://www.random.org/integer-sets/}). For this, 7 sets per year were
requested, of 10 unique random integers in each one, taken from the
range [1, 52]. The integers in each set were sorted ascending in the
following order: Monday (Set 1), Tuesday (Set 2), Wednesday (Set 3),
Thursday (Set 4), Friday (Set 5), Saturday (Set 6) and Sunday (Set 7).
In each set the number of the week of the year was indicated, from 1 to
52.

In short, 70 days per year (770 days in total) were chosen, with the aim
of achieving a logical and adequate sample size for this study. In this
way, it was possible to reduce the volume of news to be analyzed while
maintaining the representativeness of the sample.

Once the sample was obtained, a content analysis was carried out using
different variables according to some blocks of categories, considering
aspects of location, sources, content, and others. In this way, it has
also been possible to select the exact news headlines for each of the
causes of events studied.

For the in-depth interviews, four people were selected according to
criteria of relevance in the exercise of the profession and
institutional authority within the specific field of communication in
Navarra.

In the case of professional media, both directors of traditional media
and digital natives have been combined, also considering the point of
view of a journalist from the events section. The association body for
journalists in the \emph{Comunidad Foral de Navarra} is integrated into
the Association of Journalists of Navarre (AJN), so it is considered
appropriate to obtain information from the president of the AJN.
According to these criteria, the people selected have been the
following:


\begin{itemize}
	\item D. Patxi Pérez Fernández (president of the Association of Journalists
of Navarre).
\item Mr. Jesús Morales (responsible for the Events Section of \emph{Diario
	de Noticias}).
\item Mr. Gabriel González Ortiz (journalist in the Events and Courts
Section of \emph{Diario de Navarra}).
\item D. Ignacio Murillo (director of \url{Navarra.com}, \emph{Glocal Influence,
	S.L}. He has also been editor of the Events, Society and Courts section
in \emph{Diario de Navarra}).
\end{itemize}

For the qualitative analysis, a discussion group (focus group) has been
created, made up of a heterogeneous group of eight people, with the aim
of collecting existing visions in society in general. It is considered
that a lower number would make it difficult to contrast opinions and a
higher it could suppose a fragmentation of the topics to be dealt with
\cite{krueger2002designing}. A double criterion has been followed that
contemplates people involved with the reality represented and people who
represent a public, without involvement in the subject of study. Efforts
have also been made that the discussion group is made up of people of
different socioeconomic levels, that there is gender parity, different
levels of training and a variety of ages. Individuals selected based on
these criteria have given their consent to include their names. However,
to protect them as much as possible, it has been decided to introduce a
code that indicates their sex and age:

\begin{itemize}
	\item F58 (female sex, married, 58 years old, president of the Besarkada --
Abrazo Association).
\item M68 (male sex, separated, 68 years old, poet, former vice president of
the Ateneo Navarro, currently retired).
\item F18 (female sex, single, 18 years old, high school student).
\item M18 (male sex, single, 18 years old, Engineering Degree student).
\item F57 (female sex, single, 57 years old, anesthesia nurse at Clínica
Universidad de Navarra).
\item M58 (male sex, married, 58 years old, forensic doctor and director of
the Forensic Clinic Service of the Navarre Institute of Legal Medicine).
\item F44 (female sex, married, 44 years old, assistant at the emergency
call center and housewife).
\item M70 (male sex, married, 70 years old, PhD in Information Sciences,
currently retired).
\end{itemize}

The structured interview is made up of three blocks: Practices and
routines of professionals, Audience, and Guidelines and ethical
criteria.

The discussion group script is structured in four parts: social
perception of accidents and suicides, social reaction to this type of
news, role of the media in disseminating news about events, and future
challenges.

\section{Results and Discussion}\label{sec-results}

\Cref{tab-03} presents the general results of determining the level of
development of POECC of pre-service mathematics teachers in CG and EG
based on the described criteria, including motivational, operational,
cognitive, and creative aspects.

\begin{table}[htpb]
\centering
\begin{threeparttable}
\caption{Levels of development of POECC of pre-service teachers of mathematics at the pre-and post-stages of experimental training.}
\label{tab-03}
\begin{tabular}{ l l l l l l l l l l}
\toprule		
 &  & \multicolumn{8}{c}{Criteria}\\
\multicolumn{2}{c}{Levels of development} & \multicolumn{2}{c}{Motivational} & \multicolumn{2}{c}{Operational} & \multicolumn{2}{c}{Cognitive} & \multicolumn{2}{c}{Creative} \\
 & & \emph{CG} & \emph{EG} & \emph{CG} & \emph{ЕG} & \emph{CG} & \emph{ЕG} & \emph{CG} & \emph{ЕG} \\
\midrule
\multirow{2}{*}{Low level} & N & 26 & 27 & 21 & 20 & 22 & 20 & 29 & 30 \\
 & \% & 52 & 54 & 42 & 40 & 44 & 42 & 58 & 60 \\
\cmidrule{2-10}
\multirow{2}{*}{Average level} & N & 19 & 19 & 23 & 25 & 21 & 24 & 17 & 16 \\
 & \% & 38 & 38 & 46 & 50 & 42 & 48 & 34 & 32 \\
\cmidrule{2-10}
\multirow{2}{*}{High level} & N & 5 & 4 & 6 & 5 & 7 & 6 & 4 & 4 \\
 & \% & 10 & 8 & 12 & 10 & 14 & 12 & 8 & 8 \\
\cmidrule{2-10}
\multirow{2}{*}{Total} & N & 50 & 50 & 50 & 50 & 50 & 50 & 50 & 50 \\
 & \% & 100 & 100 & 100 & 100 & 100 & 100 & 100 & 100  \\
\bottomrule
\end{tabular}
\source{Own elaboration.}
\end{threeparttable}
\end{table}

The analysis of the results obtained from both the EG and CG showed a
generally low level of students' knowledge of professional English
terminology, as well as their ability to translate professional English
texts and comprehend oral professional speech. The students lacked both
professional foreign language skills and basic listening skills.
However, the results also indicated an average level of the students'
ability to analyze professional English sources. The students were able
to understand the English content, but they struggled to comprehend it
as a whole.

The analysis of the results of establishing the level of development of
POECC of pre-service mathematics teachers in CG and EG showed no
significant statistical difference. 11\% of CG prospective teachers and
9\% of EG future teachers demonstrated a high level of POECC
development, while 40\% (CG) and 42\% (EG) demonstrated an average
level, and 49\% (CG) and 49\% (EG) demonstrated a low level.

The $\chi^2$ criterion was used to determine statistical significance (with 2
degrees of freedom and a significance level of $p < 0.05$). 
The critical value,$\chi^2_{cr}$, was found to be 5.991, while the empirical
value, $\chi^2_{emp}$ , was 0.468. As $\chi^2_{emp} < \chi^2_{cr}$, the null hypothesis
$H_0$ was accepted, indicating that there was
no statistically significant difference in the level of development of
POECC between pre-service teachers of mathematics in the CG and EG and
that any differences observed were likely due to chance. It has been
concluded that the difference in results between the EG and CG at the
beginning of experimental training was statistically insignificant ($\chi^2_{emp} < \chi^2_{cr}$). 
The results obtained indicate that there were no
significant differences in the level of development of POECC between the
pre-service teachers of mathematics in the CG and EG.

Therefore, the assessment of pre-service mathematics teachers' level of
development in terms of their knowledge of basic English professional
terms, translation of professional texts, analysis of the content of
English professional sources, and ability to listen to oral professional
speech indicated a low or average level of proficiency in English for
Specific Purposes (ESP). The study confirmed the necessity of adjusting
the training program by incorporating the use of ICT tools in the
development of POECC for pre-service mathematics teachers. The
insufficiently formed level of POECC among pre-service mathematics
teachers may be related to their low awareness of the potential of using
English information resources in ESP online classes.

During the experimental training, CG students followed the traditional
ESP training program and curriculum, while EG students were taught using
virtual resources and digital tools.

Following the experimental training, a final test of ESP proficiency was
conducted for both groups according to four criteria. The experts
conducted an assessment using a 100-point scale for four components,
graded similarly to the beginning of experimental training. The level of
development of POECC among pre-service mathematics teachers after
experimental training (post-stage) was evaluated based on motivational,
cognitive, operational, and creative criteria. \Cref{tab-04} presents the
results of the dynamics of the development of POECC levels of
pre-service mathematics teachers according to motivational, cognitive,
operational, and creative criteria.

\begin{table}[!htpb]
\centering
\begin{threeparttable}
\caption{Generalized comparative analysis of the levels of POECC development of pre-service teachers of mathematics.}
\label{tab-04}
\begin{tabular}{*{9}{l}}
\toprule
\multicolumn{1}{p{3cm}}{\multirow{3}{=}{Levels of development of POECC}} & \multicolumn{4}{p{4cm}}{Test of ESP proficiency before experimental training} & \multicolumn{4}{p{4cm}}{Test of ESP proficiency after experimental training} \\
 &	\multicolumn{2}{c}{CG} & \multicolumn{2}{c}{ЕG} & \multicolumn{2}{c}{CG} & \multicolumn{2}{c}{ЕG}\\
 & N & \% & N & \% & N & \% & N & \% \\
\midrule
\multicolumn{9}{c}{Motivational criterion} \\
Low & 26 & 52 & 27 & 54 & 24 & 48 & 21 & 42 \\
Average & 19 & 38 & 19 & 38 & 21 & 42 & 22 & 44 \\
High & 5 & 10 & 4 & 8 & 5 & 10 & 7 & 14 \\			
\midrule
\multicolumn{9}{c}{Operational criterion}\\
Low & 21 & 42 & 20 & 40 & 19 & 38 & 16 & 32 \\
Average & 23 & 46 & 25 & 50 & 26 & 52 & 26 & 52 \\
High & 6 & 12 & 5 & 10 & 5 & 10 & 8 & 16 \\
\midrule
\multicolumn{9}{c}{Cognitive criterion}\\
Low & 22 & 44 & 20 & 40 & 23 & 46 & 17 & 34 \\
Average & 21 & 42 & 24 & 48 & 21 & 42 & 25 & 50 \\
High & 7 & 14 & 6 & 12 & 6 & 12 & 8 & 16 \\			
\midrule
\multicolumn{9}{c}{Creative criterion}\\
Low & 29 & 58 & 30 & 60 & 28 & 56 & 22 & 44 \\
Average & 17 & 34 & 16 & 32 & 18 & 36 & 20 & 40 \\
High & 4 & 8 & 4 & 8 & 4 & 8 & 8 & 16 \\
\bottomrule
\end{tabular}
\source{Own elaboration.}
\end{threeparttable}
\end{table}

The results indicate a positive change in the level of POECC development
of pre-service mathematics teachers in the experimental group (EG)
compared to the control group (CG). Specifically, the percentage of
pre-service mathematics teachers with a low level of POECC decreased to
11\% in the EG, while the percentage of those with an average level
increased to 4.5\% and those with a high level increased to 6.5\%.
Although the positive changes were not significant, they were
statistically large enough to be noticed. The results demonstrate that
the use of ICT in ESP learning significantly contributed to the
development of pre-service mathematics teachers' POECC. It is evident
that POECC development is a long process, and only minor positive
changes were observed during the one-semester study.

To ensure the reliability of the results, we formulated the null
hypothesis ($H_0$) that the difference in the
levels of POECC development of pre-service mathematics teachers in the
control group (CG) and experimental group (EG) was statistically
insignificant. The alternative hypothesis
($H_1$) stated that the difference in the
levels of POECC development of prospective mathematics teachers in the
CG and EG was statistically significant and reliable. The statistical
criterion $\chi^2$, as per \cref{eq-01}, was used to evaluate the homogeneity of
groups.



The critical value of the $\chi^{2}$ criterion was 5.991 with 2 degrees of freedom
and a significance level of $p < 0.05$. The empirical
value was calculated to be $\chi^2_{emp} = 9.156$. The criterion $\chi^{2}$ was used to determine
the statistical significance of the difference in the levels of POECC
development of pre-service teachers of mathematics in the CG and EG. As $\chi^2_{emp}$
was greater than $\chi^2_{cr}$, the null hypothesis $H_0$ was
rejected. This indicates that the difference in the levels of POECC
development between the two groups was statistically significant and had
a regular character.



The statistical analysis of the data from EG and CG revealed that the
difference between the values before and after the experimental training
in EG was significant, while the difference between the pre- and
post-training data in CG was insignificant (refer to Table 6).
	
The comparative result analysis of the averaged indicators of the levels of POECC development is shown in \Cref{tab-05}.
	
\begin{table}[!htpb]
\centering
\begin{threeparttable}
\caption{Comparative analysis of the average indicators of the levels of POECC.}
\label{tab-05}
\begin{tabular}{l l l l l}
\toprule
\multicolumn{1}{p{3cm}}{\multirow{3}{=}{Levels of development of POECC}} & \multicolumn{2}{p{3cm}}{before experimental training} & \multicolumn{2}{p{3cm}}{after experimental training} \\
& \emph{CG} & \emph{ЕG} & \emph{CG} & \emph{ЕG}  \\
& \emph{\%} & \emph{\%} & \emph{\%} & \emph{\%}  \\
\midrule
Low & 49 & 49 & 47 & 38 \\
Average & 40 & 42 & 43 & 46.5 \\
High & 11 & 9 & 10 & 15.5 \\
Total & 100 & 100 & 100 & 100 \\
\bottomrule
\end{tabular}
\source{Own elaboration.}
\end{threeparttable}
\end{table}
	
The results indicate a positive change in the level of POECC development
of pre-service mathematics teachers in the experimental group (EG)
compared to the control group (CG). Specifically, the percentage of
pre-service mathematics teachers with a low level of POECC decreased to
11\% in the EG, while the percentage of those with an average level
increased to 4.5\% and those with a high level increased to 6.5\%.
Although the positive changes were not significant, they were
statistically large enough to be noticed. The results demonstrate that
the use of ICT in ESP learning significantly contributed to the
development of pre-service mathematics teachers' POECC. It is evident
that POECC development is a long process, and only minor positive
changes were observed during the one-semester study.
	
To ensure the reliability of the results, we formulated the null
hypothesis ($H_0$) that the difference in the
levels of POECC development of pre-service mathematics teachers in the
control group (CG) and experimental group (EG) was statistically
insignificant. The alternative hypothesis
($H_1$) stated that the difference in the
levels of POECC development of prospective mathematics teachers in the
CG and EG was statistically significant and reliable. The statistical
criterion $x^{2}$, as per \Cref{formula-01}, was used to evaluate the homogeneity of groups.
	
The critical value of the $x^{2}$ criterion was 5.991 with 2 degrees of freedom
and a significance level of $p < 0.05$. The empirical
value was calculated to be $x_{emp}^{2}$ = 9.156. The $x^{2}$ criterion was used to determine
the statistical significance of the difference in the levels of POECC
development of pre-service teachers of mathematics in the CG and EG. As
$x_{emp}^{2}$ was greater than $x_{cr}^{2}$ , the null hypothesis $H_0$ was
rejected. This indicates that the difference in the levels of POECC
development between the two groups was statistically significant and had
a regular character.

The statistical analysis of the data from EG and CG revealed that the
difference between the values before and after the experimental training
in EG was significant, while the difference between the pre- and
post-training data in CG was insignificant (refer to \Cref{tab-06}).
	
\begin{table}[!htpb]
\centering
\begin{threeparttable}
\caption{$x^{2}$ Pearson test values.}
\label{tab-06}
\begin{tabular}{lll}
\toprule
\multirow{2}{*}{Groups} & \multicolumn{1}{p{3cm}}{\multirow{2}{=}{The calculated value of $x_{emp}^{2} $}} & $x_{cr}^{2} $ \\ 
 & & 0,05 \\
\midrule
\multicolumn{3}{c}{Before the experimental training} \\ 
ЕG and CG & 0,468 & 5,991 \\[0.3cm]
\multicolumn{3}{c}{After the experimental training} \\ 
ЕG and CG & 9,156 & 5,991 \\
\bottomrule
\end{tabular}
\source{Own elaboration.}
\end{threeparttable}
\end{table}
	
The data revealed that, following experimental training, students in the
experimental group (EG) demonstrated a higher level of development in
the area of POECC than students in the control group (CG). This was not
due to chance, but rather due to the advantages of implementing ICT in
ESP learning by pre-service mathematics teachers. The study confirmed
the hypothesis that the use of ICT in ESP learning is effective. The
technology was developed and implemented in experimental training.
Following the integration of ICT technology into education, there has
been a noticeable improvement in the development of pre-service
mathematics teachers' POECC. This is due to an
increased focus on the ability to understand and translate professional
English texts, analyze their content, and comprehend professional
speech. It is important to note that these skills require a solid
foundation in basic English professional terminology.

The obtained study results are consistent with previous research
conducted by Sutherland \emph{et al.} (2004), Awada \emph{et al.}
(2020), Røkenes and Krumsvik (2016) that proved the effectiveness of ICT
use in ESP learning and its positive impact on English communicative
competency. Rosa (2016), Gałan and Półtorak (2019) analyzed the digital
English courses on educational platforms, which included all necessary
learning materials and tools for checking completed tasks, and as well
argued that the level of English proficiency increased in the case of
ICT use in the ESP learning.


Due to the small sample size (N=100), the survey results cannot be
generalized to the entire population. Therefore, this study should be
considered as an exploratory investigation aimed at identifying possible
issues and trends for further research.

	

\section{Conclusions}\label{sec-conclusions}

One of the main types of evidence of this study has been the
confirmation of the great difference that exists between the number of
news published in all media about traffic accidents (89.14\%) compared
to the other three causes studied.

In addition to this important data, the great finding of this study has
been the confirmation of the high number of accurate headlines that also
fully coincide with the development of the news (27.15\%).

This result suggests that the media are using automated journalism,
directly uploading the information they receive from emergency services
or information agencies, without modifying, working on or adapting the
news at all.

This news production mechanism is used by all the media, to a greater or
lesser extent, with \url{lavanguardia.com} being the media that uses it
the most (66.23\% of all its news on traffic accidents are identical to
those of other media). In total, 663 duplicated, 17 tripled and 6
quadrupled news items have been detected, which means that the same
headline and body of the news item is repeated on up to six occasions in
exactly four media outlets at the same time.

In any case, an analysis has also been carried out on the exact news
published by all these media about other types of events, obtaining
similar results for drowning and accidental falls but observing a
different informative treatment in the case of suicides, where only
11.00\% of coincidences with other journalistic pieces (12 news).

Regarding the results of the qualitative analysis, a dependency of the
media on the notes issued by the emergency services is observed in the
in-depth interviews, although it is stated that a contrast of this
information is carried out before incorporating it as a publication.

It is indicated that the news of events are published to a greater
extent because they arouse interest in the audience and that the
sensitive data of the victims is taken into account in their production.

On the other hand, the discussion group believes that journalistic work
is carried out superficially and, in the case of suicides, without
social commitment. Regarding this first statement, the fact that more
and more traffic news is being published automatically seems to prove
them right.

Therefore, taking the agenda-setting theory as a basis for reflection,
where the media influence the issues that concern and speak of citizens
-in this case, events-, it seems that they have finally managed to shape
the public agenda so that the citizens talk preferentially about traffic
accidents over other types of events \cite{scheufele2007framing}.

In view of all these special characteristics described, it can also be
seen that there is automated journalism in the writing of traffic
accident news, which in most cases are automatically reported from the
emergency services to the media themselves without adapt the news
neither to its editorial line nor to its audience.

In short, attention must be drawn to the vicious circle that can be
generated if the media continue to offer exhaustive information on
traffic accidents, while ignoring the reality of some events that also
represent a social problem.


\printbibliography\label{sec-bib}

\end{document}
