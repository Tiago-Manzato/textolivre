\section{Considerações finais}\label{sec-consideraçõesfinais}

Tivemos como objetivo neste estudo adaptar e validar um instrumento de avaliação de \textit{podcast} educacional cujo conteúdo vise ao desenvolvimento do letramento em saúde. Uma limitação deste estudo é relativa à escolha do instrumento a ser adaptado e avaliado. Como há poucos instrumentos psicométricos disponíveis para a avaliação de \textit{podcasts} educativos, não foi possível escolher em um universo maior um instrumento que seria mais apropriado para tal. Destaque-se que embora tenha sido pensado no contexto do letramento em saúde, o IAPE pode ser aplicado em diferentes contextos e públicos-alvo. Estudos futuros permitirão identificar a necessidade ou não de modificação do instrumento.  

Enfocamos neste escrito principalmente a adaptação e a validação do instrumento. As etapas da adaptação e validação contaram com a tradução por tradutor profissional e retrotradução do instrumento por falantes nativos da língua-cultura de partida; adaptação do instrumento, após a tradução, por um comitê de especialistas e a validação do instrumento em português brasileiro por juízas e juízes especialistas. Na adaptação e validação do instrumento foram considerados os aspectos intersubjetivos da tradução, destacando-se especialmente um olhar para o caráter interacional da linguagem. No processo de construção da arquitetura textual-discursiva do IAPE várias vozes se entrecruzam: a dos autores do QEAP, a dos tradutores, a das especialistas e a das juízas e juízes. Essas vozes e as dos novos interlocutores que usarão o protocolo continuarão o tecido da cadeia ininterrupta de enunciados que compõe a interação verbal.