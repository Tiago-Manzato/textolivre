\section{Resultados e discussões }\label{sec-resultados}

Antes de apresentar o resultado da avaliação, caracterizemos os juízes
escolhidos. A maior parte das juízas e juízes participantes foi do sexo
feminino (71,42\%), com idade variando de 33 a 64 anos, 24 anos de formação, em
média; 57,1\% delas com doutorado, havendo quatro professoras universitárias e
três \textit{podcasters}/\textit{youtubers}, um destes também exercendo
atividade na docência de ensino superior. Passemos aos resultados.

Em relação à pertinência, todos os itens de cada fator atingiram o valor médio
mínimo para aprovação (acima de 3 pontos), como exposto na \cref{tab-01}, a
seguir.

\begin{table}[htpb]
\small
\centering
\begin{threeparttable}
\caption{Pontuação média obtida nos itens relevância e clareza dos tópicos de cada fator do Instrumento de Avaliação de Podcast Educativo (IAPE), segundo avaliação das juízas e juízes especialistas}
\label{tab-01}
\begin{tabular}{p{10cm} l l}
\toprule
Itens do Instrumento & Relevância & Clareza \\
\midrule

Fator 1. Acesso e uso &  &  \\
1. Foi fácil acessar o podcast & 4,14 & 4,29 \\
2. Consegui visualizar o podcast em diversos dispositivos (smartphone, PC etc.) & 4,29 & 3,86 \\
3. Consegui visualizar o podcast em vários lugares (casa, rua, emprego, ônibus, shopping etc.) & 4,43 & 4,00 \\
4. Foi fácil encontrar o podcast online & 4,14 & 4,14 \\
		
Fator 2. Design e estrutura &  &  \\	
5. A duração do podcast é apropriada para a compreensão de seu conteúdo & 4,71 & 4,71 \\
6. O design do podcast (capa) é atraente & 4,29 & 4,00 \\
7. O formato de apresentação do podcast é bom & 4,29 & 3,43 \\
8. O áudio do podcast é claro & 4,29 & 4,29 \\
9. O áudio e o vídeo estão devidamente sincronizados & 4,14 & 4,43 \\
		
Fator 3. Adequação de conteúdo &  &  \\
10. O podcast oferece um bom resumo do assunto & 4,14 & 3,86 \\
11. A terminologia usada no podcast é apropriada & 4,14 & 3,86 \\
12. Os exemplos usados no podcast são adequados & 4,14 & 4,00 \\
13. O conteúdo do podcast é relevante para o assunto & 4,29 & 4,00 \\
		
Fator 4. Importância como ferramenta de aprendizagem &  &  \\	
14. O podcast ofereceu uma boa ajuda para aprender sobre o assunto & 4,43 & 4,00 \\
15. O podcast reforçou minha compreensão do assunto & 4,29 & 4,00 \\
16. O podcast tornou o assunto mais agradável & 4,29 & 4,00 \\
17. O podcast foi útil para aprender sobre o assunto & 4,29 & 4,14 \\
18. Estou satisfeito com o podcast como um recurso para aprender este assunto & 4,43 & 4,00 \\
19. O podcast incentiva os alunos a aprenderem sozinhos & 3,57 & 4,14 \\
20. O podcast melhorou minha compreensão do conteúdo de cada assunto & 4,29 & 4,00 \\
\bottomrule
\end{tabular}
\source{Realização própria.}
\end{threeparttable}
\end{table}

O índice de concordância médio obtido na avaliação quanto à relevância, tanto
geral, como por fator, foi 0,86.

Já quanto à concordância das juízas e juízes em relação ao tópico
\enquote{relevância da presença de cada item no questionário}, a concordância
foi acima de 0,70 em todos os itens. Quanto à clareza, por sua vez, para dois
itens o índice de concordância foi 0,57, a saber, no fator 1, o item 2:
\enquote{Consegui visualizar o \textit{podcast} em diversos dispositivos
(\textit{smartphone}, PC, etc.)} e, no fator 2, o item 7: \enquote{O formato de
apresentação do \textit{podcast} é bom}, como pode ser visto na \cref{tab-02}.



\begin{table}[htpb]
\small
\centering
\begin{threeparttable}
\caption{Índice de concordância das juízas e juízes especialistas quanto à avaliação dos itens relevância e clareza dos tópicos de cada fator do Instrumento de Avaliação de \textit{Podcast} Educativo (IAPE)}
\label{tab-02}
\begin{tabular}{p{10cm} l l}
\toprule
Itens do Instrumento & Relevância & Clareza \\
\midrule
Fator 1. Acesso e uso & & \\
1. Foi fácil acessar o podcast & 4,14 & 4,29 \\
2. Consegui visualizar o podcast em diversos dispositivos (smartphone, PC etc.) & 4,29 & 3,86 \\
3. Consegui visualizar o podcast em vários lugares (casa, rua, emprego, ônibus, shopping etc.) & 4,43 & 4,00 \\
4. Foi fácil encontrar o podcast online & 4,14 & 4,14 \\
		
Fator 2. Design e estrutura & & \\
5. A duração do podcast é apropriada para a compreensão de seu conteúdo & 4,71 & 4,71 \\
6. O design do podcast (capa) é atraente & 4,29 & 4,00 \\
7. O formato de apresentação do podcast é bom & 4,29 & 3,43 \\
8. O áudio do podcast é claro & 4,29 & 4,29 \\
9. O áudio e o vídeo estão devidamente sincronizados & 4,14 & 4,43 \\
		
Fator 3. Adequação de conteúdo & & \\
10. O podcast oferece um bom resumo do assunto & 4,14 & 3,86 \\
11. A terminologia usada no podcast é apropriada & 4,14 & 3,86 \\
12. Os exemplos usados no podcast são adequados & 4,14 & 4,00 \\
13. O conteúdo do podcast é relevante para o assunto & 4,29 & 4,00 \\
		
Fator 4. Importância como ferramenta de aprendizagem & & \\
14. O podcast ofereceu uma boa ajuda para aprender sobre o assunto & 4,43 & 4,00 \\
15. O podcast reforçou minha compreensão do assunto & 4,29 & 4,00 \\
16. O podcast tornou o assunto mais agradável & 4,29 & 4,00 \\
17. O podcast foi útil para aprender sobre o assunto & 4,29 & 4,14 \\
18. Estou satisfeito com o podcast como um recurso para aprender este assunto & 4,43 & 4,00 \\
19. O podcast incentiva os alunos a aprenderem sozinhos & 3,57 & 4,14 \\
20. O podcast melhorou minha compreensão do conteúdo de cada assunto & 4,29 & 4,00 \\
\bottomrule	
\end{tabular}
\source{Realização própria.}
\end{threeparttable}
\end{table}

O índice de concordância médio obtido na avaliação quanto à relevância, tanto
geral, como por fator, foi 0,86. Já em relação à clareza, os índices dos
fatores 1 a 4 foram, respectivamente, 0,71; 0,80; 0,71; e 0,76 e o global foi
0,75.

Alguns, dentre as juízas e juízes, fizeram algumas sugestões para melhoria do
instrumento, especificando substituições de palavras que tornariam o
instrumento mais claro. Houve, ainda, alguns comentários de que alguns
enunciados ou algumas palavras poderiam ser modificados para que se tornassem
mais compreensíveis, mas não foram fornecidas sugestões específicas de
modificação.

Um dos juízes, entretanto, pontuou que o instrumento poderia ser aprimorado
explicitando-se o que se pretendia obter com afirmativas em que houvesse
palavras, cujos significados atribuíssem juízos de valor que poderiam ser
relativos, considerando-se o conhecimento prévio de cada leitor. Assim, foi
sugerido que usar termos como \enquote{agradável}, \enquote{adequado},
\enquote{melhorar a compreensão}, \enquote{aprender sozinho}, por exemplo,
poderia demandar explicações adicionais ao leitor do instrumento para que ele
pudesse melhor interpretar as respostas quando o instrumento for aplicado.

Continuando o processo de adaptação do instrumento, a partir de uma visão
dialógica da interação verbal, o comitê de especialistas responsivamente
\cite{volochinov2013} realizou modificações no instrumento de acordo com as
sugestões recebidas dos interlocutores no processo de validação, considerando a
plateia social à qual o instrumento se dirige. As modificações podem ser vistas
na \cref{tab-03}, a seguir.

\begin{table}[htbp]
\small
\centering
\caption{Alterações no Instrumento de Avaliação de \textit{Podcast} Educativo (IAPE) avaliado, considerando sugestões das juízas e juízes especialistas participantes.}
\label{tab-03}
\begin{threeparttable}
\begin{tabular}{p{0.475\textwidth} p{0.475\textwidth}}
\toprule
Versão preliminar & Versão definitiva\\ 
\midrule
Fator 1. Acesso e uso & \\
2. Consegui visualizar o \textit{podcast} em diversos dispositivos (\textit{smartphone}, PC etc.) & 2. Consegui visualizar o \textit{podcast} em diversos dispositivos (celular, computador, \textit{tablet}, \textit{notebook} etc.)\\

Fator 2. Design e estrutura & \\
6. O design do \textit{podcast} (capa) é atraente & 6. A capa (design) do \textit{podcast} é atraente\\ 
7. O formato de apresentação do \textit{podcast} é bom & 7. Gostei do formato de apresentação do \textit{podcast}\\
9. O áudio e o vídeo estão devidamente sincronizados & 9. O áudio e o vídeo estão bem sincronizados\\

Fator 3. Adequação de conteúdo &\\
11. A terminologia usada no \textit{podcast} é apropriada & 11. As palavras usadas no \textit{podcast} são apropriadas\\ 

Fator 4. Importância como ferramenta de aprendizagem & \\
15. O \textit{podcast} reforçou minha compreensão do assunto & 15. O \textit{podcast} melhorou minha compreensão do assunto\\
19. O \textit{podcast} incentiva os alunos a aprenderem sozinhos  & 19. O \textit{podcast} incentiva os ouvintes a aprenderem sozinhos\\
\bottomrule
\end{tabular}
\end{threeparttable}
\source{Realização própria.}
\end{table}

Com as devidas modificações, a \cref{tab-04} exibe o IAPE definitivo. Como já
afirmamos anteriormente, de acordo com os autores do instrumento original
\cite{alarcon2017,alarcon2020}, o \textit{podcast} deve ser avaliado pelo
público-alvo segundo uma escala tipo Likert, com as opções: não, baixa, média,
alta ou muito alta, sendo a elas atribuídas pontos, respectivamente, de 1 a 5.
Para ser considerado aprovado, a cada tópico deve ser atribuída a opção alta ou
muito alta, com pontuação mínima acima de 3 pontos. A avaliação global dos
fatores e global do instrumento deve refletir essa pontuação. Na \cref{tab-04},
a seguir, apresentamos a versão definitiva do IAPE.

\begin{table}[htb]
\small
\centering
\caption{Instrumento de Avaliação de \textit{Podcast} Educativo (IAPE), versão definitiva em português, proposto para avaliação de \textit{podcast} educativo por diferentes públicos.}
\label{tab-04}
\begin{threeparttable}
\begin{tabular}{p{0.975\textwidth}}
\toprule
Fator 1. Acesso e uso \\
1. Foi fácil acessar o \textit{podcast}.\\
2. Consegui visualizar o \textit{podcast} em diversos dispositivos (celular, computador, \textit{tablet}, \textit{notebook} etc.). \\
3. Consegui visualizar o \textit{podcast} em vários lugares (casa, rua, emprego, ônibus, shopping etc.).\\
4. Foi fácil encontrar o \textit{podcast} \textit{online}. \\

Fator 2. Design e estrutura\\
5. A duração do \textit{podcast} é apropriada para a compreensão de seu conteúdo. \\
6. A capa (design) do \textit{podcast} é atraente. \\
7. Gostei do formato de apresentação do \textit{podcast}.\\
8. O áudio do \textit{podcast} é claro. \\
9. O áudio e o vídeo estão bem sincronizados. \\

Fator 3. Adequação de conteúdo\\
10. O \textit{podcast} oferece um bom resumo do assunto.\\
11. As palavras usadas no \textit{podcast} são apropriadas.\\
12. Os exemplos usados no \textit{podcast} são adequados. \\
13. O conteúdo do \textit{podcast} é relevante para o assunto. \\

Fator 4. Importância como ferramenta de aprendizagem\\
14. O \textit{podcast} ofereceu uma boa ajuda para aprender sobre o assunto. \\
15. O \textit{podcast} melhorou minha compreensão do assunto. \\
16. O \textit{podcast} tornou o assunto mais agradável. \\
17. O \textit{podcast} foi útil para aprender sobre o assunto. \\
18. Estou satisfeito com o \textit{podcast} como um recurso para aprender este assunto. \\
19. O \textit{podcast} incentiva os ouvintes a aprenderem sozinhos. \\
20. O \textit{podcast} melhorou minha compreensão do conteúdo de cada assunto.\\
\bottomrule
\end{tabular}
\end{threeparttable}
\source{Realização própria.}
\end{table}

Neste escrito, adotamos a perspectiva indisciplinar da linguística aplicada
\cite[p. 14]{moita2006}, em que se tem na LA \enquote{um modo de criar
inteligibilidade sobre problemas sociais em que a linguagem tem um papel
central}. No caso desse estudo, tratamos da necessidade de adaptação de
protocolo que auxilie profissionais de saúde e o público em geral na avaliação
de \textit{podcasts} educacionais, uma vez que, na construção do letramento em
saúde, recursos educacionais digitais podem ser uma das formas de promover a
prevenção e o bem-estar em saúde. Apresentamos assim a adaptação e a validação
de um instrumento de avaliação de \textit{podcast} educacional cujo conteúdo
vise ao desenvolvimento do letramento em saúde. Dessa forma, damos a conhecer à
comunidade científica, o Instrumento de Avaliação de \textit{Podcast} Educativo
(IAPE). O instrumento foi avaliado e aprovado por juízes especialistas e, como
já afirmamos, cobre uma lacuna quanto à avaliação de \textit{podcasts} pela
população leiga ou profissional.

Este tipo de instrumento colabora para assegurar que esta estratégia
metodológica de educação encontra respaldo na comunidade, no que tange a ser
apreciada, compreendida e passível de ser colocada em prática. Devido à forma
como o IAPE se apresenta, pode ser possível ser utilizado em outros contextos
com proposta educativa, seja na área da saúde ou em outras áreas. Fica aberto
então para que outros pesquisadores possam testar sua validade. 

Na proposta de \textcite{alarcon2020}, os pesquisadores já sugerem avaliar a
aplicabilidade do QEAP a diferentes públicos, como, por exemplo, na atenção
básica em saúde, em escolas de ensino fundamental e médio, entre outros. Como o
instrumento se destina a avaliar \textit{podcasts} educativos de forma geral e
como RED em forma de \textit{podcast} são recursos que estão se popularizando,
ainda segundo os autores, a exposição do instrumento a diversos cenários pode
favorecer o aprimoramento do protocolo em função de possíveis dificuldades ou
inadequações que porventura sejam encontradas nesses novos contextos.  

Como refletiu um dos juízes participantes da avaliação do IAPE, algumas
orientações sobre o protocolo precisam ser explicitadas. É importante, então,
que ao respondente seja explicado que o conteúdo de suas respostas deve
refletir sua opinião e percepção individuais [e não o que ele acredita que já
deveria saber sobre \textit{podcast}, por exemplo], antes mesmo que ele  ouça e
avalie cada episódio de \textit{podcast}. 

A produção de instrumentos de avaliação de \textit{podcasts} educativos ainda é escassa, mesmo em âmbito internacional. Os mesmos autores do QEAP \cite{alarcon2020} tinham, previamente, desenvol vido um instrumento mais resumido para avaliar \textit{podcasts}, o \textit{Student Satisfaction with Educational Podcasts Questionnaire} (SSEPQ), composto por 10 itens, mas que não estratificava as questões nos fatores presentes no QAEP \cite{alarcon2017}. Os 4 fatores presentes no QAEP, e agora no IAPE, permitem identificar mais detalhadamente a qualidade de um \textit{podcast} quanto à acessibilidade, utilização, estrutura e conteúdo educativo.

O IAPE permite, portanto, fugir da avaliação focada no acesso, presente nas
plataformas de distribuição. Em plataformas, como Spotify, Google Podcasts,
Apple Podcasts, entre outras, é possível verificar quantos acessos cada
episódio de um \textit{podcast} teve e o perfil de quem o acessou, mas não se
consegue uma análise de conteúdo e respectiva compreensão sob a ótica do
público-alvo, tampouco há um \textit{feedback} direto sobre os aspectos
negativos. A indicação desses últimos, por exemplo, permitiria que os autores
realizassem modificações, quando necessárias.

No campo da saúde, segundo revisão de \textcite{paterson2015}, destaca-se que a
falta de métricas para avaliar qualidade de \textit{blogs} e \textit{podcasts}
tem consequências negativas, como a falta de uma diretriz que ajude o
profissional de saúde a identificar a qualidade daquilo que está recebendo; a
dificuldade de educadores em ter segurança para indicar uma fonte para ser
vista ou ouvida; dificuldades de avaliação de qualidade de produção acadêmica;
dificuldades para blogueiros e \textit{podcasters} que não têm um guia para
desenvolver um bom material educativo. Em sua revisão, os autores identificaram
três blocos de indicadores: credibilidade, conteúdo e \textit{design}. Por
outro lado, destacaram a ausência de estrutura e estratificação destes
indicadores para aplicação pelos interessados \cite{paterson2015}.

Considerando a revisão de \textcite{paterson2015}, \textcite{chan2016}
desenvolveram indicadores para avaliar qualidade e confiabilidade de
\textit{podcasts}. Esses autores chegaram a 13 indicadores, propondo uma
avaliação através de uma escala tipo Likert, com sete graduações, desde
atributo não exibido até atributo bem exibido. No entanto, tais indicadores são
apropriados apenas para avaliação profissional.

\textcite{biabdillah2021} associaram \textit{podcast} com elementos de
gamificação no campo da história, no ensino médio. Os autores utilizaram teste
de usabilidade (\textit{ThePost-Study System Usability Questionnaire – }PSSUQ)
como instrumento de avaliação, integrado por 16 itens, mas tal avaliação apenas
se aplica quando o conteúdo educativo demanda interatividade.

Este estudo traz a limitação relativa à escolha do instrumento a ser adaptado e
avaliado, pois a escassez de instrumentos disponíveis não permitiu uma análise
aprofundada sobre qual o instrumento seria mais apropriado para tal. Por outro
lado, o instrumento aqui escolhido é de fácil aplicação devido à clareza e
objetividade dos itens, considerando o público-alvo, e foi avaliado como
relevante pelos especialistas convidados. Mesmo assim, sua aplicação junto a
diferentes públicos-alvo poderá apontar melhor sua adequação e, ao mesmo tempo,
permitirá ajustes, caso sejam necessários.

Um ponto positivo do presente estudo foi a presença de uma linguista aplicada
na equipe de pesquisadores, o que garante a melhor adaptação do conteúdo aos
aspectos interacionais da linguagem, assim como reafirma o olhar da LA sobre a
relação entre linguagem e práticas sociais.

O estudo indica, ainda, a necessidade de desenvolvimento de mais instrumentos
de avaliação de \textit{podcasts} educativos, principalmente devido ao alto
número de \textit{podcasts} existentes e ao contingente crescente de ouvintes
dessa mídia.
