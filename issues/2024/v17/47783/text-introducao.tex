\section{Introdução}\label{sec-introdução}
A produção e o consumo de \textit{podcasts}, tal como o \textit{podcast} educativo, vêm aumentando nos últimos anos, tanto no mundo, como no Brasil. Esse tipo de recurso educacional digital é de fácil utilização devido a não pressupor tempo e locais determinados para a visualização, à facilidade de armazenamento e de acesso em plataformas gratuitas e à possibilidade de papel ativo do aluno no processo de aprendizagem [quando os \textit{podcast} são produzidos pelos próprios alunos, por exemplo]. Embora seja um tipo de arquivo conhecido por usuários do ambiente digital, ainda não há um consenso para sua definição \cite{viana2020}. Há mais de uma década, \textcite{bottentuit2007} afirmaram que \textit{podcasts} seriam arquivos digitais de áudio, facilmente baixados e ouvidos em diferentes dispositivos. Segundo \textcite{ballsberry2018}, seu conteúdo pode englobar várias temáticas e seu objetivo é a transmissão da informação.

Considerando a noção de gêneros do discurso como tipos relativamente estáveis de enunciados que funcionam nas diferentes esferas da atividade humana \cite{bakhtin2016}, admite-se aqui que o \textit{podcast} é enunciado relativamente estável, construído social e culturalmente, e que como tal procede de alguém e se dirige a alguém \cite{volochinov2013}. Em outras palavras, entendemos o \textit{podcast} como um gênero discursivo oral que se caracteriza por sua dinamicidade multissemiótica, \enquote{uma vez que essas produções podem explorar diferentes recursos/mecanismos que são indiciadores de sentido, promovendo possibilidades de análise dos usos da modalidade oral} \cite[p. 53]{villarta2022}.

Sobre o surgimento e a circulação do gênero, a primeira produção de \textit{podcast} no Brasil aconteceu no ano de 2004. No ano seguinte, 2005, ocorreu a Conferência Brasileira de \textit{Podcasters}, como são chamados aqueles que produzem [do ponto de vista da linguagem] esse tipo de enunciado. Atualmente, podem ser encontradas produções bem diversificadas, com os mais diferentes objetivos (entreter, informar, ensinar, resenhar, entre outros) e sobre os mais variados temas em forma de \textit{podcast} \cite{abpod2021}. Entretanto, apesar da popularização desse gênero nas últimas décadas, parece limitado, ainda, o número de \textit{podcasts} educativos direcionados à população em geral.

O \textit{podcast} educativo é um tipo de recurso educacional digital (RED). Os REDs, segundo \textcite{araujo2019}, são entidades digitais que têm como objetivo o ensino. Os formatos em que se apresentam os REDs são bastante variados: \textit{softwares}, videoaulas, vídeos, áudios, \textit{podcast}, textos, infográficos, jogos, por exemplo. As principais características técnicas dos REDs são a granularidade (apresentação de pequenos recortes de conteúdo) e a reusabilidade (capacidade de ser usado diversas vezes e em diferentes situações e contextos de aprendizagem). 

Existem entidades digitais que já nascem como REDs, isto é, já são concebidas com o objetivo de ensinar, tal como um jogo educacional digital, por exemplo, ou como um \textit{podcast} educativo. Existem, por outro lado, recursos digitais que são, por assim dizer, convertidos em REDs, ou seja, que são transformados em algo para ensinar. Seria, por exemplo, o caso de um documentário do tipo jornalístico, que pode se converter em uma entidade digital para ensinar a respeito de determinado tema.

Assim como todo recurso pedagógico, os REDs, como recursos complementares de ensino, devem ser avaliados. Especificamente sobre instrumentos de avaliação de REDs em forma de \textit{podcast}, estudos como os de \textcite{semakula2017,semakula2020} e os de \textcite{sulistiawati2022} avaliam \textit{podcasts} educativos por meio de instrumentos criados para verificar qual foi a aprendizagem do conteúdo exposto no(s) episódio(s). É pertinente que se faça isso e que sejam avaliados conteúdos específicos, por exemplo.  No entanto, é necessário que haja um instrumento geral que permita a percepção do usuário sobre sua própria aprendizagem. Com a avaliação facilitada, tanto criadores de RED em forma de \textit{podcast}, como professores ou, no caso da área de saúde, profissionais da rede de atenção básica poderiam entender melhor como, para quem e por que utilizarem \textit{podcast} para promover a aprendizagem.

Por outro lado, mesmo considerando a avaliação específica importante, reconhecemos que, operacionalmente, é difícil desenvolver e validar diferentes instrumentos cujos conteúdos sejam específicos, mesmo considerando-se um recorte de área, como a da saúde, a cada \textit{podcast} construído. Por isso mesmo, propomos aqui a  adaptação e validação de um instrumento de avaliação geral. 

Destaque-se que os REDs em forma de \textit{podcast} educativo são recursos complementares a outras atividades educativas, o que permite que a avaliação específica de conteúdo possa ser realizada em outros momentos do contato educador-educando.

Especificamente quando se trata de letramento em saúde, os REDs normalmente são produzidos por equipes multidisciplinares, cujos responsáveis são profissionais de saúde. Esses recursos então já são produzidos com o objetivo de ensinar um conteúdo sobre saúde a leigos ou a profissionais.

Admitindo-se, numa perspectiva bakhtiniana, que a compreensão dialógica ativa exige a inserção do objeto a ser depreendido em um contexto dialógico \cite{bakhtin2017}, explicamos brevemente o contexto a partir do qual se concebe o letramento em saúde, exemplificando-o no cenário de funcionamento da Atenção Primária em Saúde (APS), que é a porta de entrada de usuários do sistema público e universal de saúde brasileiro, o Sistema Único de Saúde (SUS).

No âmbito do SUS, de acordo com {\textcite{marques2022}}, a APS é elemento fundamental para a qualidade do sistema, especialmente quando estruturada a partir da Estratégia de Saúde da Família. Este nível de atenção, por sua vez, apresenta como um dos atributos essenciais a longitudinalidade ou vínculo longitudinal da assistência.

 Na construção do vínculo ao longo do tempo entre usuários do SUS e profissionais de equipes multidisciplinares de APS, a comunicação é essencial para que se estabeleça relação de confiança e responsabilidade mútua. Sendo assim, como afirmam \textcite{sampaio2015}, é fundamental refletir sobre como as pessoas compreendem e utilizam as orientações da equipe profissional [de APS] para tomar decisões e agir no cuidado consigo mesmas. É este então o contexto das discussões que perpassam a noção de letramento em saúde.

Esse tipo específico de letramento diz respeito ao conjunto de conhecimentos e habilidades desenvolvidos pelas pessoas em interações sociais mediadas, segundo a \textcite{who2021}, por estruturas organizacionais [de saúde] e pela disponibilidade de recursos que permitem às pessoas acessar, entender, avaliar e usar informações e serviços de maneira a promover e manter boa saúde e bem-estar para si e para os que estão ao seu redor. Considerando-se a definição de letramento em saúde que acabamos de mencionar, os \textit{podcasts}, como REDs, podem atuar como recursos em prol do letramento em saúde de quem o escuta.

Elencando características de \textit{podcast} educativo na área de letramento em saúde, \textcite{lopes2015} e \textcite{sampaio2021} afirmam que esse tipo de RED deve apresentar alguns atributos previamente definidos, antes de sua oferta ao público-alvo. Dentre elas, as principais seriam: 
\begin{enumerate}%[label=(\alph*)]
	\item tipo (orientação/instrução; expositivo/informativo; \textit{feedback}/comentário);
	\item formato (áudio, vídeo com locução ou apenas vídeo, \textit{videocast} e \textit{screencast});
	\item duração (curto – entre 1 a 5 minutos; moderado – 6 a 15 minutos; longo – mais de 15 minutos);
	\item estilo (linguagem utilizada: formal ou informal);
	\item funcionalidade (informar, divulgar, motivar, orientar);
	\item integralidade (envolvimento do autor (\textit{podcaster}) e ouvinte);
	\item definição do conteúdo (livre escolha do \textit{podcaster}).
\end{enumerate}

O \textit{podcast} educativo, como um RED, se diferencia dos demais porque sua finalidade é o ensino de temas/conteúdos neles contidos e não apenas a divulgação de informações. Nesta perspectiva, esses tipos de \textit{podcasts} necessitam ser avaliados e, para tanto, há a necessidade de instrumentos delineados para essa avaliação.

Recentemente, \textcite{muniz2021} desenvolveram um instrumento para validação de \textit{podcast} por profissionais da saúde, que nesse caso eram enfermeiros. O instrumento foi desenvolvido para avaliação de \textit{podcast} cujo conteúdo era referente à hanseníase, mas pode ser utilizado para validação de \textit{podcasts} com diferentes conteúdos educativos em saúde, substituindo-se o tema hanseníase pelo tema que estiver sendo avaliado. O instrumento desenvolvido por \textcite{muniz2021} já foi aplicado por \textcite{mota2021} e por \textcite{leite2022}.

Por outro lado, há poucos instrumentos desenvolvidos para avaliação de \textit{podcast} pelo público-alvo. Um desses instrumentos, que foi desenvolvido por \textcite{alarcon2020}, traz essa proposta. Trata-se do \textit{Questionnaire for Assessing Educational Podcasts} (QAEP). Os autores realizaram avaliação de episódios de \textit{podcast} sobre métodos e estatísticas de pesquisa, disciplina ministrada em um curso de Psicologia, da Universidade de Málaga, Espanha. O público que avaliou os episódios foi representado pelos alunos deste curso. O QAEP mostrou boas propriedades psicométricas e os autores sugeriram sua aplicação em outros contextos, envolvendo pessoas de diferentes graus de escolaridade. No Brasil, porém, não há ainda um instrumento validado para avaliação de \textit{podcasts} educativos por público-alvo.

Assim, o objetivo deste estudo foi traduzir, adaptar e validar o QAEP para o português brasileiro. Dividimos assim este escrito em quatro partes: esta introdução, na  qual apresentamos o \textit{podcast} como um recurso educacional digital utilizado para o letramento em saúde e objetivo do estudo aqui apresentado; a metodologia, em que descrevemos o percurso metodológico, bem como os aspectos éticos, empreendidos na  adaptação e validação do instrumento; os resultados e discussão, em que apresentamos as avaliações e sugestões feitas pelos juízes especialistas a respeito da tradução, assim como a discussão a respeito do Instrumento de Avaliação de \textit{Podcast} Educativo (IAPE), denominação dada ao instrumento em português a partir da adaptação e validação descritas neste artigo e, finalmente, trazemos a conclusão do estudo.
