\section{Metodología}\label{sec-metodología}

La presente investigación se abordó principalmente desde una perspectiva metodológica cuantitativa, apoyada en el uso de técnicas bibliométricas que fueron respaldadas y completadas mediante un análisis cuantitativo que favoreció la interpretación de los resultados.

El tratamiento bibliométrico se enfocó en la producción científica entorno al uso didáctico de historias, y consistió en el estudio de las pautas de publicación a través de análisis cuantitativos y estadísticos. Como señala \cite{mcburney2002what}, la bibliometría nos indica cuántos artículos se han publicado en referencia a una línea de investigación concreta y, mediante el análisis de citas, podemos evaluar el impacto que algunos de ellos han tenido en la comunidad científica.

De esta manera, el análisis bibliométrico se considera un método riguroso para explorar y analizar grandes volúmenes de datos científicos, que permite desentrañar los matices evolutivos de un campo específico a la vez que nos muestra nuevas vías de estudio \cite{donthu2021how}. Las principales ventajas de este tipo de trabajos se centran en su capacidad de cuantificar y organizar la producción bibliográfica segmentando por valores como la relevancia de los autores e instituciones que trabajan en una línea de investigación. Para el desarrollo de este trabajo se empleó un método de revisión de la bibliografía específica que ha sido utilizado y contrastado por diversos autores en estudios similares, llevados a cabo en diferentes campos de estudio \cite{herrera-franco2020research,md2018bibliometric,pham2021bibliometric}.

De acuerdo con \textcite{andalai2010scopus}, consideramos la base de datos SCOPUS como la más apropiada para realizar búsquedas relativas a un campo específico de estudio, debido a la cantidad y calidad de sus archivos científicos. 

De esta manera, se realizó una búsqueda de información bibliográfica aplicando los filtros de búsqueda en SCOPUS para obtener una síntesis cuantitativa acerca de la producción científica que podría fundamentar futuros trabajos sobre Aprendizaje Basado en Historias y otras estrategias didácticas, que también tomen a la narración como base de su propuesta de intervención. Para concretar nuestro ámbito de estudio, se establecieron las palabras clave apropiadas que condujeran al correcto desarrollo de nuestra revisión documental. Aunque se seleccionaron todos los idiomas, el idioma elegido para la definición de estas palabras fue el inglés ya que, como varios autores señalan, esta lengua cubre el 90 \% de las publicaciones científicas en Scopus y otras bases de datos, con lo que nos aseguramos cubrir el ámbito internacional \cite{albarillo2014language,mongeon2016journal}.

Los criterios establecidos para la filtración de la información objeto de estudio fueron los siguientes:

\begin{itemize}
	\item
	Palabras clave en el título: (\enquote{Storytelling}, \enquote{Learning} y
	\enquote{Digital})
	\item
	Intervalo de publicación: Desde el uno de enero de 2019 hasta el 31 de
	diciembre de 2022.
	\item
	Tipo de documentos: Artículos.
	\item
	Ámbito de publicación: Todos.
	\item
	Idiomas: Todos.
	\item
	Tipo de publicación: Acceso abierto.
\end{itemize}

Antes de aplicar estos filtros, pudimos localizar un total de 1.572 artículos relacionados con nuestras palabras clave, pero al aplicar el intervalo de publicación indicado obtuvimos un total de 145 artículos disponibles, lo que representa un 9,5 \% del total de publicaciones en SCOPUS. Sobre esta muestra se realizó un análisis aplicando los siguientes criterios:

\begin{itemize}
	\item
	Número de artículos anuales.
	\item
	Artículos más citados.
	\item
	Autores destacados por su frecuencia de producción.
	\item
	Instituciones filiadas.
	\item
	Producción por países.
	\item
	Disciplina científica.
	\item
	Línea de investigación.
\end{itemize}

Para el correcto desarrollo de este análisis cualitativo se realizó una lectura comprensiva de cada una de las publicaciones seleccionadas, con el objeto de establecer las principales líneas temáticas que ocupaban el desempeño de los autores. Para definirlas se plantearon tres preguntas iniciales de investigación y de esta manera, se establecieron diferentes categorías, atendiendo a los conceptos que resultaban más interesantes para la comunidad científica en relación con el Aprendizaje Basado en Historias.

Por lo tanto, en este trabajo se combinan técnicas cuantitativas y cualitativas para ofrecer una revisión bibliográfica que pueda mostrar una visión holística sobre el estado de la cuestión objeto de estudio.

Las preguntas de investigación fueron las siguientes:

\begin{enumerate}
	\def\labelenumi{\arabic{enumi}.}
	\item
	¿Cuáles son las disciplinas más interesadas en el Aprendizaje Basado
	en Historias?
	\item
	¿Cuáles son las finalidades pedagógicas de cada una de ellas?
	\item
	¿Cuáles son los términos o conceptos más investigados en cada una de
	ellas?
\end{enumerate}