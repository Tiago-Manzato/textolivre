\begin{polyabstract}
\begin{abstract}
    O objetivo deste artigo é refletir sobre os discursos que circulam acerca do livro e da leitura e, especificamente, analisar, sob a ótica da Análise do Discurso, a partir das contribuições de \textcite{foucault2006escrita,foucault2009arqueologia,foucault2010ordem,foucault2014historia} a este campo, vídeos de \textit{booktubers} com apresentações de estantes dos comentadores em uma prática chamada por eles de \textit{bookshelf tour}. Destacam-se os sentidos de: mostrar que se é um leitor; de capital social para aquele que lê; e de socialização das práticas leitoras, por meio da fluidez e da dinamicidade da web, em suas possíveis correlações com o ensino de leitura e escrita no Brasil. Questiona-se, com a regularidade de tais práticas, novas pluralidades para os significados da leitura, abertas a formatos audiovisuais e a formas variadas de falar sobre livros e leituras. O \textit{corpus} é composto por três vídeos do YouTube nos quais acontece a montagem de estantes e organização desses espaços. O conceito fundamental da Análise do Discurso para este estudo é o conceito de sujeito, na perspectiva foucaultiana, além dos conceitos de poder e saber também mobilizados por este autor. A concepção de leitura adotada advém dos estudos de \textcite{chartier1998aventura,chartier1999ordem,chartier2019lersem} e da história cultural da leitura.

\keywords{Discurso \sep Leitura \sep Sujeito \sep \textit{Booktubers}}
\end{abstract}

\begin{abstract}
The objective of this article is to reflect on the discourses that circulate about books and reading and, specifically, to analyze, from the perspective of Discourse Analysis, based on the contributions of \textcite{foucault2006escrita,foucault2009arqueologia,foucault2010ordem,foucault2014historia} to this field , videos of booktubers with presentations of commentators' shelves in a practice they call bookshelf tour. The meanings of: showing that you are a reader; of social capital for those who read; and socialization of reading practices, through the fluidity and dynamism of the web, in its possible correlations with the teaching of reading and writing in Brazil. With the regularity of such practices, new pluralities for the meanings of reading are questioned, open to audiovisual formats and varied ways of talking about books and reading. The corpus consists of three YouTube videos in which the assembly of shelves and organization of these spaces takes place. The fundamental concept of Discourse Analysis for this study is the concept of subject, from a Foucauldian perspective, in addition to the concepts of power and knowledge also mobilized by this author. The concept of reading adopted comes from the studies of \textcite{chartier1998aventura,chartier1999ordem,chartier2019lersem} and the cultural history of reading.

\keywords{Discourse \sep Reading \sep Subject \sep Booktubers}
\end{abstract}
\end{polyabstract}
