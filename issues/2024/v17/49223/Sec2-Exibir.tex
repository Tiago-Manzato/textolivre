\section{Exibir o livro, ler o quê? Práticas do sujeito-leitor moderno entre a leitura e a exibição}\label{sec-exibirolivro}

Escrever: gesto e prática que mudou a história, que constitui sociedades
grafocêntricas e que colocou à margem tantas outras sociedades que
insistiram em suas práticas orais. O homem da escrita, o homem da letra,
é cultuado na sociedade contemporânea como o homem que sabe, fazendo jus
à nomenclatura latina \textit{homo sapiens}. Escrita que endossa o poder e
permite a vigilância da sociedade, uma vez que o que é escrito,
permanece e marca atos, gestos, pensamentos e idiossincrasias do sujeito
no movimento da vida. Dos sinais gráficos nas cavernas, das plaquetas de
argila na Mesopotâmia ao códex e, atualmente, o livro eletrônico. Não
bastasse termos um jeito novo de ler nos suportes de telas, luzes, cores
e sons, temos também um jeito novo, não tão novo assim, de se falar do
que lemos e de como consumimos o que lemos. Perscrutamos brevemente com
Chartier e outros um pouco da historicidade do livro, da leitura e do
leitor, concentrados aqui nesta historicidade de se mostrar que se é um
leitor e fundamentamos com Foucault esse sujeito moderno, autocentrado,
politizado nas tramas dos poderes e saberes e que escreve (e lê) a si em
práticas das \enquote{artes da existência} para problematizar, complexificar e
marcar sua posição no mundo.

Na história cultural da leitura e do livro, \textcite{chartier1999ordem} pontua que \enquote{a leitura é sempre uma prática encarnada em gestos, em espaços, em
hábitos} \cite[p. 13]{chartier1999ordem}. Temos no ato de mostrar a estante gestos singulares e representativos, em espaços diversos como a própria casa
daquele que grava e ao mesmo tempo apresenta o vídeo e em hábitos
multifacetados de leitura dos \textit{booktubers}. Essas práticas
hodiernas corroboram o pensamento de Chartier uma vez que a leitura
desses \textit{booktubers}, revelam suas práticas constituindo-se numa
prática de leitura dos jovens leitores da internet. Falar dos livros e
mostrar-se como um leitor é uma prática há muito tempo cotejada e
praticada tanto que \enquote{escrever sobre suas próprias leituras tornou-se um
verdadeiro gênero \enquote*{literário}, praticado com prazer por intelectuais e
escritores} \citeyear[p. 7]{chartier2019lersem}. Nos textos que são publicados
acerca das práticas de memórias das leituras, Chartier concebe dois
tipos de leitor: o herdeiro e o leitor que nasceu num mundo sem livros.
No primeiro caso:

\begin{quote}
[\ldots] os livros estão desde sempre presentes. A história dos
leitores nascidos em um mundo pleno de livros é como uma viagem iniciada
bem cedo entre títulos, autores e gêneros. A seleção da memória e a
maneira de se apresentar nessas narrativas enfatizam a precocidade da
capacidade de ler, as descobertas furtivas, as leituras transgressivas e
sempre opostas às leituras escolares, pesadas e chatas. Tal como se
tivesse nascido em uma biblioteca, o leitor \enquote{herdeiro} constrói suas
leituras de infância à distância do modelo e do repertório escolar
\cite[p. 7–8]{chartier2019lersem}.
\end{quote}

No segundo caso:

\begin{quote}
Os leitores que nasceram em um mundo sem livros, ou quase sem nenhum,
escolhem outro padrão narrativo: aquele segundo o qual a leitura é uma
conquista e não uma herança. Nas suas memórias a escola desempenha um
papel fundamental. Suas leituras mais pessoais são, de fato, as leituras
requeridas ou recomendadas pelos professores. Seus livros e autores
preferidos, seus gostos mais íntimos, se conformam aos repertórios mais
canônicos. Esse leitor não entra no universo da leitura graças a uma
biblioteca familiar. Ele se torna leitor na sala de aula de uma escola
\cite[p. 8]{chartier2019lersem}.
\end{quote}

No artigo \enquote{Ler sem livros}, Chartier nos dá pistas de uma prática que
pode estar sendo realizada com os \textit{booktubers}, deslocando o nosso
olhar dos produtores dos vídeos para o público espectador. Será que os
vídeos sobre livros e leituras, fazem com que adolescentes e jovens
(público que parecer ser a maioria deste tipo de canal no YouTube) e até
mesmo adultos leiam mais (ou melhor) a partir destas ferramentas
audiovisuais nesta famosa plataforma da internet? Mostrar a montagem de
uma estante de livros revela uma intimidade com a leitura e aproxima o
público leitor e fomenta as práticas leitoras? E de qual leitura estamos
falando? Ao tratarmos da questão destes resenhistas ou blogueiros
literários no YouTube, precisamos também pensar nas formas de
comunicação e linguagem que se transformam com as plataformas digitais.
Para \textcite{chartier2019lersem},

\begin{quote}
[\ldots] o mundo digital produz sobretudo a transformação das
categorias mais fundamentais da experiência humana, por exemplo, as
noções de amizade multiplicada até o infinito, de identidade fictícia ou
pluralizada, de privacidade ocultada ou exibida, como a invenção de
novas formas de cidadania -- ou de controle e de censura \cite[p. 15]{chartier2019lersem}.
\end{quote}

O interesse do historiador da leitura está também no suporte material do
livro. Uma vez que há uma comunidade virtual do livro e da leitura e com
o advento do livro eletrônico, fica, para o autor, mais evidente a cada
dia a possibilidade de ler sem livro, tomado como o objeto material da
forma que o conhecemos, o códex.

\begin{quote}
A digitalização de todas essas práticas e relações sociais (entre os
indivíduos, com o mercado, ou com as instituições) impõe uma ubiquidade
da escrita e da leitura sobre as mesmas telas (do computador, do \textit{tablet}, do \textit{smartphone}) e sob as mesmas formas breves, segmentadas, maleáveis. Se até agora o livro ainda manteve sua presença como objeto no mercado editorial e como tipo de discurso na edição digital, devemos considerar que as práticas cotidianas, multiplicadas, incessantes, de escrita e de leitura se afastam e nos afastam radicalmente do livro em sua dupla
natureza, material e textual. Cada dia se lê mais sem livro \cite[p. 15–16]{chartier2019lersem}.
\end{quote}

O objeto livro físico, impresso, ainda é muito comum nos canais dos
\textit{booktubers} tanto que há a montagem das estantes para se guardar
os livros. Essa prática pode ter uma influência muito forte do mercado
livreiro e, como outros estudos já mostraram \cite{costa2018booktubers,aguiar2017critica}, este mercado utiliza-se dos
\textit{booktubers} como iscas, ou possibilidades de aquecer o mercado e
faz um marketing indireto para que por meio de suas resenhas e de suas
práticas consigam captar e formar novos leitores que comprarão livros
nas editoras que patrocinam, enviam mimos, livros e outros para os
comentadores do YouTube. \textcite{fialho2023booktubers}, em se tratando do
fenômeno \textit{booktuber} no geral e, mais especificamente, dos
\textit{booktubers} brasileiros, \enquote{as narrativas, na perspectiva
multimodal, perpassam por outros aspectos vinculados à leitura, a
exemplo da indicação de marcas de leitores digitais e das recomendações
de compras das obras apresentadas} \cite[p. 12]{fialho2023booktubers}. Neste
sentido, permite-nos inferir que não se trata mais apenas dos livros e
dos seus conteúdos, mas de todo um modo próprio de dizer sobre a prática
da leitura e dos leitores que marcam a si. Dessa forma, esse contexto
permite-nos refletirmos também sobre a ordem dos livros na sociedade
atual. Quem tem a possibilidade de comprar livros e ter uma estante
requintada no quarto ou na sala de estar? Outro ponto que é preciso
olhar com cautela é que os vídeos dos jovens resenhistas são vídeos bem
editados, com cores, linguagem atrativa e jovial, marcando, nas práticas
de linguagem, o sujeito que se constitui e faz emergir os jogos de
poderes e saberes na tessitura da rede que pluraliza os sentidos,
mascarando uma inclusão, que muitas vezes exclui e coloca à margem
leitores outros a quem são interditados esses espaços da internet bem
como o acesso aos livros e à leitura.

Para \textcite{chartier1998aventura}, \enquote{o texto eletrônico torna possível uma relação muito mais distanciada, não corporal} \cite[p. 16]{chartier1998aventura}. É por isso, talvez, que esse contato mais íntimo com os livros e com os aparatos ou objetos do universo da leitura ganham cada vez mais a cena e precisam até mesmo ser divulgados em vídeos no YouTube. Uma vez que a leitura nos formatos eletrônicos e digitais é fria, para aqueles que são amantes da cultura impressa é preciso se aproximar ainda mais para não deixar que o impresso se perca. É por isso que \textcite{chartier1998aventura} pontua que não há uma ruptura muito brusca entre o livro impresso e o livro eletrônico, assim como não houve entre o manuscrito e o impresso. Dessa forma a estante aparece como um subterfúgio para aproximar leitores e ao mesmo tempo
ostentar a sua aquisição, o seu espaço de leitura que supostamente torna
o sujeito exibidor um leitor. O sujeito não mais fala do livro em si,
mas dos objetos ligados a ele, pois há um envolvimento afetivo. E é
nisso que podemos pensar numa \enquote{leitura de si}, pensamento este advindo
da \textit{escrita de si} em \textcite{foucault2006escrita} que nos faz apontar para o leitor mostrando a si como leitor, num transbordamento daquilo que na
\textit{História da Sexualidade} é chamado cultura de si, como mostraremos
a partir de agora.
