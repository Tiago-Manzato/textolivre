\section{Da intimidade à ostentação: a leitura de si como
exibição, o livro e a leitura como mercado e autopromoção nas teias
capitalistas do YouTube}\label{sec-daintimidadeaostentação}

Para efetivar as análises que aqui apresentamos e constituir o
\textit{corpus} e os recortes que submetermos à análise, fizemos uma busca
na plataforma YouTube por meio dos termos \enquote{estantes de livros},
\enquote{booktubers e estantes} e \enquote{\textit{bookshelf tour}}. Encontramos uma enormidade de vídeos e optamos por aqueles que tinham mais
visualizações. Depois disso, selecionamos os perfis e os vídeos que se
aproximam da hipótese que aventamos, ou seja, que mostrar a estante é um
movimento de intimidade e de ostentação e que repercute em sentidos
outros para a constituição de si como sujeitos leitores. Feita essa
seleção adotamos quatro recortes para analisarmos neste texto devido a
sua brevidade.

O primeiro vídeo analisado intitula-se \enquote{Minha estante nova: montagem,
projeto, como ficou (antes da organização)}, produzido pela
\textit{booktuber} Mell Ferraz cujo canal intitula-se
\enquote{Literature-se}\footnote{Disponível em:
  \url{https://www.youtube.com/watch?v=NdVi4FZ05sc}}. Nele, Mell Ferraz
apresenta sua nova estante de madeira, ampla, bonita e imponente, sem os
livros. Na apresentação do vídeo a \textit{booktuber} diz:

\begin{quote}
\textbf{Recorte 1:}
Olá pessoal! Eu sou a Mell Ferraz e, sim, você está no canal
Literature-se, apesar deste cenário \textbf{super diferente}. Eu
estou na minha casa e esta é a \textbf{minha estante}, que estou
há muito tempo falando para vocês que um dia ela chegaria, apesar de
todos os \textbf{percalços}, que eu vou contar neste vídeo para
vocês e explicar melhor de onde surgiu \textbf{a ideia}, como
foi \textbf{o desenvolvimento} dessa ideia e também como colocar
ela (sic) em prática e chegarmos até hoje que é a finalização da
montagem desta \textbf{minha estante nova} aqui em casa.
\end{quote}

%De acordo com o "Butterick’s Practical Typography", manual de tipografia, jamais se deve utilizar duas maneiras diferentes de destacamento em um mesmo texto, quiçá ao mesmo tempo. Além disso, segundo o mesmo, as linhas inferiores, underlines, sublinhamentos, são uma maneira de destacar um texto durante o uso de máquinas tipográficas, ou seja, em uso digital é antiquado e pouco-recomendado, em face do itálico e do negrito.

Logo após esta apresentação é veiculada a vinheta de abertura do canal e
em seguida um clipe de 4 minutos mostrando a montagem da estante pelos
marceneiros, com furadeiras e todo material de marcenaria. Nomear a
\textbf{minha estante} é algo comum nos diversos vídeos
encontrados, não só dos recortes aqui apresentados, mas de todo o
\textit{corpus}. O lugar da estante parece simbolizar um sentido não para
todo e qualquer espectador do vídeo, mas para aqueles que também são
adeptos desta prática, outros \textit{booktubers}. As regularidades
enunciativas mostram que os dizeres são direcionados a um público
específico formado pelo que Foucault denominou em \textit{A ordem do
discurso} de sociedades do discurso \enquote{cuja função é conservar ou
produzir discursos, mas para fazê-los circular em um espaço fechado,
distribuí-los somente segundo regras estritas, sem que seus detentores
sejam despossuídos por essa distribuição} \cite[p]{foucault2010ordem}.
Apesar de os vídeos serem abertos para o grande público, percebe-se a
formação de grupos ou nichos que se organizam a partir de estruturas
prévias, visto o número enorme de vídeos de estantes que mostram uma
forma de ostentar para o seu semelhante e compartilharem entre si suas
\enquote{leituras de si}.

O grau da novidade - \textbf{super diferente} - é um modo
próprio de dizer, pois, grosso modo, não há nada de diferente numa
estante, ainda mais vazia, mas o enunciado faz o seu percurso de
sentido, uma vez que não cumpre uma comunicação para qualquer um, mas
para aqueles que esperam e aguardam ansiosamente por vídeos como este.

Ainda nesse recorte, podemos verificar que a sequência -
\textbf{percalços, a ideia, o desenvolvimento} - remetem a um
sentido de projeto. A estante de Mell Ferraz foi projetada e ao marcar a
sua leitura de si, a alma que entrega a si por meios dos livros precisa
definir que passou por todos os passos de um projeto e que a leitura que
desenvolve em seu canal é uma leitura projetada, não é qualquer leitura.
Assim, planejar a estante e mostrar o resultado final deste planejamento
gera uma ligação fortuita com os espectadores do vídeo que acompanham
não só os vídeos mas todo o desenrolar, como se fosse uma websérie, dos
projetos da \textit{booktuber}.

No segundo recorte, assim como a nomeação \enquote{minha estante} do vídeo
anterior, a apresentação do \textit{booktuber} Christian Assunção no vídeo
\enquote{Nova estante: como fazer}\footnote{Disponível em:
  \url{https://www.youtube.com/watch?v=0O1tq\_CCr1g\&list=WL\&index=177}} traz
uma nomeação interessante como a \enquote{A saga da estante}. Vejamos

\begin{quote}
\textbf{Recorte 2:}
Oi gente! A saga da estante tá acabando, né? Primeiro, eu vou contar o
que aconteceu, de ontem pra hoje... Quem me segue lá no Instagram
já sabe o que tá acontecendo, viu? Eu tava procurando uma estante e aí
as estantes tudo cara, muito cara e as que eu tava encontrando eram
frágeis pra livros...
\end{quote}

Em todos os vídeos sobre estantes, o que mais importa não é falar de
leituras, mas falar do material, da compra, da forma espacial como se
vai organizar os livros na estante fazendo uma junção de discursos que
vêm de outros campos como a economia, o comércio, a arquitetura. Neste
recorte, ao intitular como \enquote{a saga da estante} também vemos a mesma
regularidade do sentido de projeto do recorte anterior. Há aqui ainda o
trocadilho com o mundo dos livros, uma vez que sagas são narrativas e
desloca-se o sentido da narrativa dos livros com trama e enredo, para
uma narrativa própria que são os projetos de estante. O autor se marca e
se subjetiva, singularizando dizeres que estão nas regularidades dos
discurso que todos os vídeos de \textit{bookshelf tour}, mais precisamente
de montagem de estantes guardam.

No decorrer do vídeo, Christian Assunção detalha como foi produzindo sua
estante de forma artesanal, como foi à loja da Leroy Merlin e comprou
tábuas avulsas, depois envernizou, afixou as mãos francesas na parede,
todo um percurso que não condiz com um percurso de leitura, mas que gera
interesse, uma vez que o vídeo tem quase 12 mil visualizações. Depois de
apresentar a montagem da estante, o material feito, as medidas de cada
repartição, Christian Assunção enuncia que vai organizar os livros.

\begin{quote}
\textbf{Recorte 3:}
Vou organizar depois \textbf{mostro} pra vocês, aqui mesmo nesse
Vlog...  
\end{quote}


É preciso destacar que a nova estante de Christian é colocada ao lado de
uma outra estante abarrotada de livros e que não é qualquer estante
(assim como todas que são mostradas nos vídeos deste tipo). É uma
estante bonita, na cor branca, de material de boa qualidade. Ninguém
mostra uma estante de ferro, mal organizada, cheia de poeira ou estantes
simples. O verbo mostrar foi destacado na enunciação da organização da
estante com os livros para inferenciar que os sentidos caminham para a
exibição, a ostentação, a necessidade da estante estar organizada para
que possa ser mostrada. O sujeito mostra-se a si, mas não se mostra de
qualquer jeito. Há toda uma encenação, uma cenografia, que precisa ser
performizada para alcançar a circulação do discurso que se quer fazer
veicular e dos sentidos produzidos na referida ação. Logo após, aparecer
com a estante organizada, já com alguns livros, Christian faz outra
enunciação.

\begin{quote}
\textbf{Recorte 4:}
Eu acho que é só isso que eu tenho de \textbf{Cosac...}
Enfim, esses são os Cosac, eu deixei todos aqui nesse lugar.
\textbf{Que lindo, gente! Era o meu desejo} deixá-los sempre
assim \textbf{expostos}.
\end{quote}

Temos neste enunciado uma construção metonímica, ao se tomar a editora
como um todo no lugar dos seus livros publicados. Destacar a marca que
faz referência à editora Cosac Naify de um modo íntimo chamando apenas
pela primeira parte do nome, demonstra intimidade e ao mesmo tempo uma
ordem consumista que coloca o capital sobreposto ao interesse leitor. O
deslumbramento do sujeito - \textbf{\enquote{Que lindo, gente!}} - que revela
os seus desejos, as suas aspirações - \textbf{\enquote{Era meu desejo\ldots}}
coloca um sentido de sacralização da estante, tomada como algo com um
sentimento superior a apenas um suporte de livro. A estante é algo para
ser cultuado, admirado, reverenciado, desde que tudo seja \enquote{exposto}.

Interessante pensarmos nessas novas, não tão novas assim, formas de se
mostrar que se é um leitor entrecruzadas com aquilo que \textcite{galinari2005autorialidade}
chama de \enquote{filosofia espetacular da autorialidade}. O autor parte das
concepções de \textcite{debord_sociedade_1997,subirats_cultura_1989} acerca da \enquote{sociedade do
espetáculo}. Segundo esta filosofia, proposta por \textcite{galinari2005autorialidade}:

\begin{quote}
[\ldots] o livro tende a ser apenas o suporte legitimante de um
autor que se quer arte, que deseja trocar de lugar com a obra para,
finalmente, ser lido e assimilado. Talvez seja justamente aí que se
encontra pressuposto um novo arquétipo de leitor-modelo -- o
leitor-espectador -- , o qual acaba fazendo do livro um duplo fetiche:
(i) de decoração, ou seja, como um enfeite doméstico da sala de estar e
(ii) de autopromoção, isto é, como uma forma cômoda de comunicar
\textit{inteligência}, mediante a exibição de uma \textit{vitrine}
\cite[p. 52–53]{galinari2005autorialidade}.
\end{quote}

O autor está tratando da questão da autoria, mas aqui pode ser
transposta para a questão da leitura e da formação do leitor na
contemporaneidade, uma vez que os destaques em itálico na citação
mostram essa prática que aqui apresentamos nos dizeres/enunciados
materializados linguisticamente em vídeos que se discursivizam. \textcite{galinari2005autorialidade} continua:

\begin{quote}
A produção de toda essa \enquote{cultura}, provavelmente, efetivaria o autor
como uma boa estratégia de \textit{marketing}. Seria, por exemplo, o caso
de autores-celebridades como Jô Soares ou Chico Buarque: o \textit{Xangô
de Baker Street} e o \textit{Budapeste} não venderiam antes pela
paternidade do que por qualquer outra coisa? A autoria não seria o
principal fator responsável pelo sucesso das vendas, ao invés dos
sentidos imanentes à obra? \cite[p. 53]{galinari2005autorialidade}
\end{quote}

Mais uma vez o excerto de \textcite{galinari2005autorialidade} contempla aqui a estratégia de  marketing da qual lançam mão os \textit{booktubers}, pois muitas vezes, ao citar Cosac Naify, por exemplo, o \textit{booktuber} está ganhando um
patrocínio da editora, nem que seja um livro novo na sua caixa de
correio. Em um outro vídeo que compõe o \textit{corpus} da pesquisa, não
aqui analisado em detalhes, a \textit{bookotuber} Isabella Lubrano inicia
seu \textit{bookshelf tour}\footnote{Disponível em:
  \url{https://www.youtube.com/watch?v=abKuqpybE\_0\&t=517s}} com o pedido
para as pessoas curtirem determinada página na internet ou rede social
para que ela possa ser contemplada com uma viagem para a Alemanha,
promoção de um instituto citado pela \textit{influencer} literária.

Neste sentido, respaldados pelas análises já feitas por outros
estudiosos, nossas análises discursivas permitem-nos chegar ao movimento
da \enquote{intimidade} para a \enquote{ostentação}. Uma prática simples e que se marca com um gesto de intimidade, de proximidade com o público leitor,
se desmascara em uma prática de ostentação, de mostrar quem pode mais,
quem tem mais livros, qual estante é mais bonita e mais equipada, numa
competição indireta entre os \textit{booktubers}. Esses apontamentos
conclusivos são possíveis de serem elaborados, uma vez que a gigantesca
quantidade de vídeos desse formato - mostrando a montagem de uma
estante, compra de materiais, desenvolvimento, organização - é muito
frequente no YouTube. Esses vídeos não estão ali por um simples acaso,
ou por uma prática despretensiosa por parte dos seus executores, mas
representam uma prática mercadológica que gera não só capital social
para os seus autores, mas capital material mesmo, movimentam e fazem
aquecer o mercado literário das editoras e autores.

