\documentclass[portuguese]{textolivre}

% metadata
\journalname{Texto Livre}
\thevolume{17}
%\thenumber{1} % old template
\theyear{2024}
\receiveddate{\DTMdisplaydate{2023}{12}{18}{-1}}
\accepteddate{\DTMdisplaydate{2024}{2}{23}{-1}}
\publisheddate{\today}
\corrauthor{Danilo Vizibeli}
\articledoi{10.1590/1983-3652.2024.49223}
%\articleid{NNNN} % if the article ID is not the last 5 numbers of its DOI, provide it using \articleid{} commmand 
% list of available sesscions in the journal: articles, dossier, reports, essays, reviews, interviews, editorial
\articlesessionname{articles}
\runningauthor{Vizibeli}
%\editorname{Leonardo Araújo} % old template
\sectioneditorname{Daniervelin Pereira}
\layouteditorname{João Mesquita}

\title{Intimidade ou ostentação? As estantes dos \textit{booktubers} e os sentidos sobre a leitura e o livro nas redes sociais na contemporaneidade}
\othertitle{Intimacy or ostentation? The bookshelves of booktubers and the meanings of reading and books on social media in contemporary times}

\author[1]{Danilo Vizibeli~\orcid{0000-0002-4456-0216}\thanks{Email: \href{mailto:danilovizibeli@gmail.com}{danilovizibeli@gmail.com}}}
\affil[1]{Instituto Federal de Educação Ciência e Tecnologia do Sul de Minas Gerais, Campus Passos, Centro de Educação a Distância, Passos, MG, Brasil.}

\addbibresource{article.bib}

\begin{document}
\maketitle
\begin{polyabstract}
\begin{abstract}
Este ensaio tem por objetivo trazer algumas reflexões sobre os efeitos do uso exacerbado da tecnologia nos dias de hoje e sobre o modo como isso afeta os sujeitos contemporâneos e sua capacidade de raciocínio complexo e de aprendizagem dentro e fora da escola. Essas reflexões se assentam sobre questões situadas na intersecção entre tecnologia e ciências humanas, como a superestimulação e a continuidade trazidas pelo uso das tecnologias digitais que se chocam com a necessidade de quietude e desaceleração para a aprendizagem e o raciocínio complexo. A intensificação do uso das tecnologias digitais que vivemos afeta o funcionamento do cérebro e dispersa a atenção, apontada como o principal ativo contemporâneo – donde o termo “economia da atenção” surge para substituir o de \enquote{economia da informação}. Reflete-se também sobre qual o papel da escola no direcionamento de seus alunos quanto a essa mudança de paradigma.
	
\keywords{Tecnologia e Educação \sep Atenção \sep Economia da Informação \sep Hiperconectividade}
\end{abstract}

\begin{abstract}
The aim of this essay is to provide some reflections on the effects of the excessive use of technology today and how this affects contemporary subjects and their capacity for complex thinking and learning in and out of school. These reflections are based on issues situated at the intersection of technology and the human sciences, such as the over-stimulation and continuity brought about by the use of digital technologies, which clash with the need for stillness and deceleration for learning and complex reasoning. The intensified use of digital technologies that we are experiencing affects the functioning of the brain and disperses attention, which has been identified as the main contemporary asset - hence the term \enquote{attention economy} has emerged to replace \enquote{information economy}. It also reflects on the role of schools in guiding their students through this paradigm shift.
	
\keywords{Technology and Education \sep Attention \sep Information economy \sep Hyperconnectivity}
\end{abstract}
\end{polyabstract}

\section{\enquote{Passeando pela nova estante}: novas configurações da leitura e representações do leitor contemporâneo na cultura digital}\label{Sec-passeando}

 Falar de livros, falar de filmes, falar dos mais variados objetos
culturais não é prática nova. Contudo, com o advento das redes sociais
e, mais precisamente, da plataforma audiovisual YouTube, comentar um
livro tem se tornado prática rotineira e realizada, ao contrário da
crítica literária profissional, por pessoas amadoras, ou estudantes do
universo das Letras, ganhando o nome de \textit{booktubers}. Os sujeitos
adeptos dessa prática, geralmente, são pessoas jovens que comentam
livros em diversas ações que vêm marcando uma certa regularidade em não
só resenhar os livros, mas apresentar estantes, leituras coletivas,
compras de livros, inventários entre tantas outras.

Observando diversos canais que têm por objetivo comentar livros por meio
de resenhas, percebeu-se entre eles uma prática comum que é chamada de
\textit{bookshelf tour}, ou seja, um passeio pela estante onde o
apresentador guarda seus livros. Além desta, outras práticas surgem
nesse universo dos \textit{booktubers} como, por exemplo, a prática do
\textit{bookhaul}, que é comentar os livros que chegaram através de
editoras que enviam para terem suas obras comentadas pelos resenhistas
amadores.

Neste artigo, pretende-se mostrar em análise de discurso de matriz
francesa, pensando com Michel Foucault, algumas problematizações acerca
dessa prática em dois vídeos que, antes de serem um \textit{bookshelf
tour,} mostram a montagem da estante como um suporte, algo que
questionamos se é da ordem da intimidade ou da ostentação, ou de ambas,
e que sentidos fazem circular sobre a leitura e os livros na
contemporaneidade.

Para a fundamentação teórica da análise, adotamos o conceito de sujeito
em Michel Foucault perpassado pelos mecanismos de saber e poder.
Trazemos também a reflexão foucaultina de \textit{escrita de si}, para
pensarmos na demonstração das estantes como uma \enquote{leitura de si}.
Também é preciso considerar a história do livro e da leitura em Roger
Chartier e outros estudiosos da história cultural.

O \textit{corpus} geral do estudo é composto por sete canais literários no
YouTube e, para este artigo, foram recortados dois canais, sem deixarmos
de fazermos uma apresentação geral das práticas que se repetem em
regularidades enunciativas \cite{foucault2009arqueologia}.

Em outro estudo, \textcite{vizibeli_contrastes_2016}, partimos do pressuposto de que a crítica promovida pelos \textit{booktubers} guardam características próprias que se distinguem da crítica literária especializada. Avançando neste e em outros estudos referentes à temática, percebemos que os leitores que se
apresentam como \textit{booktubers} são colocados à margem e que
apresentar a estante é também uma forma de gerar intimidade e uma forma
de demonstração de quem são os leitores hoje, parafraseando a indagação
foucaultiana de \enquote{Quem somos nós hoje?}.
\section{Exibir o livro, ler o quê? Práticas do sujeito-leitor moderno entre a leitura e a exibição}\label{sec-exibirolivro}

Escrever: gesto e prática que mudou a história, que constitui sociedades
grafocêntricas e que colocou à margem tantas outras sociedades que
insistiram em suas práticas orais. O homem da escrita, o homem da letra,
é cultuado na sociedade contemporânea como o homem que sabe, fazendo jus
à nomenclatura latina \textit{homo sapiens}. Escrita que endossa o poder e
permite a vigilância da sociedade, uma vez que o que é escrito,
permanece e marca atos, gestos, pensamentos e idiossincrasias do sujeito
no movimento da vida. Dos sinais gráficos nas cavernas, das plaquetas de
argila na Mesopotâmia ao códex e, atualmente, o livro eletrônico. Não
bastasse termos um jeito novo de ler nos suportes de telas, luzes, cores
e sons, temos também um jeito novo, não tão novo assim, de se falar do
que lemos e de como consumimos o que lemos. Perscrutamos brevemente com
Chartier e outros um pouco da historicidade do livro, da leitura e do
leitor, concentrados aqui nesta historicidade de se mostrar que se é um
leitor e fundamentamos com Foucault esse sujeito moderno, autocentrado,
politizado nas tramas dos poderes e saberes e que escreve (e lê) a si em
práticas das \enquote{artes da existência} para problematizar, complexificar e
marcar sua posição no mundo.

Na história cultural da leitura e do livro, \textcite{chartier1999ordem} pontua que \enquote{a leitura é sempre uma prática encarnada em gestos, em espaços, em
hábitos} \cite[p. 13]{chartier1999ordem}. Temos no ato de mostrar a estante gestos singulares e representativos, em espaços diversos como a própria casa
daquele que grava e ao mesmo tempo apresenta o vídeo e em hábitos
multifacetados de leitura dos \textit{booktubers}. Essas práticas
hodiernas corroboram o pensamento de Chartier uma vez que a leitura
desses \textit{booktubers}, revelam suas práticas constituindo-se numa
prática de leitura dos jovens leitores da internet. Falar dos livros e
mostrar-se como um leitor é uma prática há muito tempo cotejada e
praticada tanto que \enquote{escrever sobre suas próprias leituras tornou-se um
verdadeiro gênero \enquote*{literário}, praticado com prazer por intelectuais e
escritores} \citeyear[p. 7]{chartier2019lersem}. Nos textos que são publicados
acerca das práticas de memórias das leituras, Chartier concebe dois
tipos de leitor: o herdeiro e o leitor que nasceu num mundo sem livros.
No primeiro caso:

\begin{quote}
[\ldots] os livros estão desde sempre presentes. A história dos
leitores nascidos em um mundo pleno de livros é como uma viagem iniciada
bem cedo entre títulos, autores e gêneros. A seleção da memória e a
maneira de se apresentar nessas narrativas enfatizam a precocidade da
capacidade de ler, as descobertas furtivas, as leituras transgressivas e
sempre opostas às leituras escolares, pesadas e chatas. Tal como se
tivesse nascido em uma biblioteca, o leitor \enquote{herdeiro} constrói suas
leituras de infância à distância do modelo e do repertório escolar
\cite[p. 7–8]{chartier2019lersem}.
\end{quote}

No segundo caso:

\begin{quote}
Os leitores que nasceram em um mundo sem livros, ou quase sem nenhum,
escolhem outro padrão narrativo: aquele segundo o qual a leitura é uma
conquista e não uma herança. Nas suas memórias a escola desempenha um
papel fundamental. Suas leituras mais pessoais são, de fato, as leituras
requeridas ou recomendadas pelos professores. Seus livros e autores
preferidos, seus gostos mais íntimos, se conformam aos repertórios mais
canônicos. Esse leitor não entra no universo da leitura graças a uma
biblioteca familiar. Ele se torna leitor na sala de aula de uma escola
\cite[p. 8]{chartier2019lersem}.
\end{quote}

No artigo \enquote{Ler sem livros}, Chartier nos dá pistas de uma prática que
pode estar sendo realizada com os \textit{booktubers}, deslocando o nosso
olhar dos produtores dos vídeos para o público espectador. Será que os
vídeos sobre livros e leituras, fazem com que adolescentes e jovens
(público que parecer ser a maioria deste tipo de canal no YouTube) e até
mesmo adultos leiam mais (ou melhor) a partir destas ferramentas
audiovisuais nesta famosa plataforma da internet? Mostrar a montagem de
uma estante de livros revela uma intimidade com a leitura e aproxima o
público leitor e fomenta as práticas leitoras? E de qual leitura estamos
falando? Ao tratarmos da questão destes resenhistas ou blogueiros
literários no YouTube, precisamos também pensar nas formas de
comunicação e linguagem que se transformam com as plataformas digitais.
Para \textcite{chartier2019lersem},

\begin{quote}
[\ldots] o mundo digital produz sobretudo a transformação das
categorias mais fundamentais da experiência humana, por exemplo, as
noções de amizade multiplicada até o infinito, de identidade fictícia ou
pluralizada, de privacidade ocultada ou exibida, como a invenção de
novas formas de cidadania -- ou de controle e de censura \cite[p. 15]{chartier2019lersem}.
\end{quote}

O interesse do historiador da leitura está também no suporte material do
livro. Uma vez que há uma comunidade virtual do livro e da leitura e com
o advento do livro eletrônico, fica, para o autor, mais evidente a cada
dia a possibilidade de ler sem livro, tomado como o objeto material da
forma que o conhecemos, o códex.

\begin{quote}
A digitalização de todas essas práticas e relações sociais (entre os
indivíduos, com o mercado, ou com as instituições) impõe uma ubiquidade
da escrita e da leitura sobre as mesmas telas (do computador, do \textit{tablet}, do \textit{smartphone}) e sob as mesmas formas breves, segmentadas, maleáveis. Se até agora o livro ainda manteve sua presença como objeto no mercado editorial e como tipo de discurso na edição digital, devemos considerar que as práticas cotidianas, multiplicadas, incessantes, de escrita e de leitura se afastam e nos afastam radicalmente do livro em sua dupla
natureza, material e textual. Cada dia se lê mais sem livro \cite[p. 15–16]{chartier2019lersem}.
\end{quote}

O objeto livro físico, impresso, ainda é muito comum nos canais dos
\textit{booktubers} tanto que há a montagem das estantes para se guardar
os livros. Essa prática pode ter uma influência muito forte do mercado
livreiro e, como outros estudos já mostraram \cite{costa2018booktubers,aguiar2017critica}, este mercado utiliza-se dos
\textit{booktubers} como iscas, ou possibilidades de aquecer o mercado e
faz um marketing indireto para que por meio de suas resenhas e de suas
práticas consigam captar e formar novos leitores que comprarão livros
nas editoras que patrocinam, enviam mimos, livros e outros para os
comentadores do YouTube. \textcite{fialho2023booktubers}, em se tratando do
fenômeno \textit{booktuber} no geral e, mais especificamente, dos
\textit{booktubers} brasileiros, \enquote{as narrativas, na perspectiva
multimodal, perpassam por outros aspectos vinculados à leitura, a
exemplo da indicação de marcas de leitores digitais e das recomendações
de compras das obras apresentadas} \cite[p. 12]{fialho2023booktubers}. Neste
sentido, permite-nos inferir que não se trata mais apenas dos livros e
dos seus conteúdos, mas de todo um modo próprio de dizer sobre a prática
da leitura e dos leitores que marcam a si. Dessa forma, esse contexto
permite-nos refletirmos também sobre a ordem dos livros na sociedade
atual. Quem tem a possibilidade de comprar livros e ter uma estante
requintada no quarto ou na sala de estar? Outro ponto que é preciso
olhar com cautela é que os vídeos dos jovens resenhistas são vídeos bem
editados, com cores, linguagem atrativa e jovial, marcando, nas práticas
de linguagem, o sujeito que se constitui e faz emergir os jogos de
poderes e saberes na tessitura da rede que pluraliza os sentidos,
mascarando uma inclusão, que muitas vezes exclui e coloca à margem
leitores outros a quem são interditados esses espaços da internet bem
como o acesso aos livros e à leitura.

Para \textcite{chartier1998aventura}, \enquote{o texto eletrônico torna possível uma relação muito mais distanciada, não corporal} \cite[p. 16]{chartier1998aventura}. É por isso, talvez, que esse contato mais íntimo com os livros e com os aparatos ou objetos do universo da leitura ganham cada vez mais a cena e precisam até mesmo ser divulgados em vídeos no YouTube. Uma vez que a leitura nos formatos eletrônicos e digitais é fria, para aqueles que são amantes da cultura impressa é preciso se aproximar ainda mais para não deixar que o impresso se perca. É por isso que \textcite{chartier1998aventura} pontua que não há uma ruptura muito brusca entre o livro impresso e o livro eletrônico, assim como não houve entre o manuscrito e o impresso. Dessa forma a estante aparece como um subterfúgio para aproximar leitores e ao mesmo tempo
ostentar a sua aquisição, o seu espaço de leitura que supostamente torna
o sujeito exibidor um leitor. O sujeito não mais fala do livro em si,
mas dos objetos ligados a ele, pois há um envolvimento afetivo. E é
nisso que podemos pensar numa \enquote{leitura de si}, pensamento este advindo
da \textit{escrita de si} em \textcite{foucault2006escrita} que nos faz apontar para o leitor mostrando a si como leitor, num transbordamento daquilo que na
\textit{História da Sexualidade} é chamado cultura de si, como mostraremos
a partir de agora.

\section{Leitura como prática de si: apontamentos foucaultianos}\label{sec-leituracomopraticadesiapontamentosfoucaultianos}

O interesse pelas práticas de si surgiu em \textcite{foucault2014historia} no terceiro volume
da \textit{História da Sexualidade 3: o cuidado de si.} Todo o volume é
dedicado a sondar as artes de si mesmo. \textcite{miranda2006licao}, no
prefácio ou introdução intitulado \enquote{A lição de Foucault} do livro
\textit{O que é um autor?}, coletânea de textos na qual está inserido o
texto foucaultiano \enquote{A escrita de si}, publicado em 1983, pontuam que:
\enquote{As teses de Foucault sobre a `estética da existência' entroncarão
neste novo espaço, que é um espaço onde a própria crítica do que somos e
do que fizeram de nós emerge como um problema histórico} \cite[p. 11]{miranda2006licao}. No capítulo segundo do terceiro volume da
\textit{História da Sexualidade,} intitulado \enquote{A cultura de si}, \textcite{foucault2014historia}
descreve detalhadamente do que se trata a \enquote{arte da existência} e o que
seria esta \enquote{cultura de si}.

\begin{quote}
Ora, é esse tema do cuidado de si, consagrado por Sócrates, que a
filosofia ulterior retomou, e que ela acabou situando no cerne dessa
\enquote{arte da existência} que ela pretende ser. É esse tema que,
extravasando de seu quadro de origem e se desligando de suas
significações filosóficas primeiras, adquiriu progressivamente as
dimensões e as formas de uma verdadeira \enquote{cultura de si}. Por essa
expressão é preciso entender que o princípio do cuidado de si adquiriu
um alcance bastante geral: o preceito segundo o qual convém ocupar-se
consigo mesmo é, em todo caso, um imperativo que circula entre numerosas
doutrinas diferentes; ele também tomou a forma de uma atitude, de uma
maneira de se comportar, impregnou formas de viver; desenvolveu-se em
procedimentos, em práticas e em receitas que eram refletidas,
desenvolvidas, aperfeiçoadas e ensinadas; ele constituiu, assim, uma
prática social, dando lugar a relações interindividuais, a trocas e
comunicações e até mesmo a instituições; ele proporcionou, enfim, um
certo modo de conhecimento e a elaboração de um saber \cite[p. 58]{foucault2014historia}.
\end{quote}

A citação de Foucault corrobora a percepção de \textcite{miranda2006licao} uma vez que a cultura de si se dá no desenvolvimento da história e por meio de
práticas diversas. Como o cuidado de si tem um alcance geral, podemos
perceber nas práticas de mostrar as estantes todo um modo próprio de
mostrar a si que arrola um processo de subjetivação, ou seja, de
constituição dos sujeitos imbricados em diferentes práticas discursivas.

Em toda sua obra, Foucault procura debater sobre os diferentes modos de
constituição do sujeito (seja quanto às formas de sujeição, seja quanto
às aberturas e às possibilidades de recusa e de resistência, seja ainda
quanto à constituição ética de si). É esta última - a constituição ética
de si - que o filósofo escreve no capítulo \enquote{A cultura de si}, desde os
gregos para depois chegar aos primeiros séculos da era cristã no
restante do livro. Também não há como isolar, na sua concepção de
discurso, aquilo que ele pensa sobre sujeito e o que afirma sobre
relações de poder.

O sujeito na visão foucaultiana é tomado como posição. As posições
ocupadas pelo sujeito advém da inscrição em determinadas formações
discursivas. Portanto, a título de compreendermos o sujeito leitor na
contemporaneidade e entender em quais formações discursivas seus
discursos, ou seus dizeres, se inscrevem atentamos para a conceituação
que \textcite{foucault2009arqueologia} já traz desde \textit{A arqueologia do saber}. Este conceito está ligado com a função enunciativa, e para Foucault o enunciado não é a frase ou a proposição. Ele assim o define: \enquote{Descrever uma formulação enquanto enunciado não consiste em analisar as relações entre o autor e o que ele disse (ou quis dizer, ou disse sem querer), mas em determinar
qual é a posição que pode e deve ocupar todo indivíduo para ser seu
sujeito} \cite[p. 108]{foucault2009arqueologia}.

Qual posição o(s) sujeito(s) ocupa(m) diante das práticas de leitura?
De qual sujeito leitor moderno podemos falar nos entremeios destes
canais de \textit{booktubers} no YouTube que analisamos? Neste texto, a
maioria dos \textit{booktubers} analisados são famosos e com grande
visualização dentro desta famosa rede social audiovisual. Há também
produtores de vídeos deste tipo do universo literário que não são
famosos, mas também marcam as suas leituras de si na grande rede e com
isso pode-se ter um movimento mais democrático da crítica literária e do
incentivo à leitura. Ao mesmo tempo em que há esse movimento
democrático, podemos perceber uma ordem da repetição e se há tantos
canais no YouTube deste mesmo nicho e com as mesmas práticas que são as
\textit{bookshelf tour} tal acontecimento vão ao encontro daquilo que
\textcite{chartier2019lersem} pontua como \enquote{A ilusão biográfica}, que \enquote{conduz o indivíduo a pensar-se como irredutivelmente único e singular quando na verdade seu discurso ou sua memória se pautam em modelos amplamente compartilhados} \cite[p. 8]{chartier2019lersem}. Essa ilusão é muito característica nas práticas que vamos analisar logo mais.

Enfatizando a revolução do texto eletrônico, \citeauthor{chartier2019lersem} no mesmo estudo chama a atenção para aspectos que vão estar presentes nesta construção de si que traz elementos não do universo dos livros, mas da afetividade que como estamos designando carrega sentidos de intimidade. O
historiador destaca:

\begin{quote}
[...] ela [a revolução do texto eletrônico] redefine a
materialidade das obras porque desata o laço visível que une o texto e o
objeto que o materializa e a partir do qual chega ao leitor, não se
limitando mais ao autor ou ao editor o domínio sobre a forma e o formato
das unidades textuais que se quer ler. Assim, é todo o sistema de
percepção e de uso dos textos que se encontra transformado \cite[p. 13–14]{chartier2019lersem}.
\end{quote}

É nesta transformação do sistema de percepção dos livros e da leitura
que os \textit{booktubers} se inscrevem. Não se pode pensar a percepção
destes espaços apenas com um critério avaliativo se isso é bom ou ruim,
mas é preciso observar a sua existência, as condições de possibilidade
(cf. \textcite{foucault2009arqueologia}) que pactuam para que a cultura de si aí se exerça como posicionamento de um sujeito que lê a si e ao mundo. Chartier
resgata ainda \textcite{lajolo_literatura_2017} oferecendo-nos suporte de
análise desta transformação que são os \textit{booktubers} que não mais
mostram apenas livros ou resenhas clássicas, mas até a montagem de suas
estantes ou de sua biblioteca. Vejamos:

\begin{quote}
Por outro lado, a inventividade dos criadores (particularmente no campo
da literatura infantil e juvenil) aproveita as possibilidades digitais
para propor gêneros, objetos, criações irredutíveis à forma impressa. As
inovações não se limitam a introduzir no \enquote{livro} os gêneros da rede
(\textit{e-mails}, \textit{blogs}, \textit{links}). Elas também produzem criações, segundo as
expressões de \textcite{lajolo_literatura_2017}, explorando o
\enquote{hibridismo de linguagens} ou \enquote{amálgamas de linguagens}. O \enquote{site} substitui o livro, a liberdade do leitor, que pode escolher entre opções narrativas, substitui o absolutismo do texto, e, muitas vezes, a
gratuidade do acesso substitui o comércio editorial. Essa aposta não é
sem importância, pois pode nos levar tanto à introdução, na textualidade
eletrônica, de alguns dispositivos capazes de perpetuar os critérios
clássicos de identificação de obras tal como é no impresso, relativos a
sua identidade e propriedade, quanto também ao abandono dessas
categorias para inventar uma nova maneira de compor novas produções
estéticas que explorem uma \enquote{plurimidialidade}, mais rica que a simples
relação entre texto e imagens e que localizem o leitor numa posição que
lhe permita fazer escolhas ou participar do processo criativo. Na
primeira hipótese, o desaparecimento do livro como objeto material não
significaria seu desaparecimento como modalidade de discurso que supõe a
percepção e compreensão da coerência e totalidade da obra. Na segunda
hipótese, ao contrário, se desenha uma nova ordem do discurso na qual o
leitor produz, corta, desloca, associa ou reconstrói fragmentos móveis e
maleáveis \cite[p. 14–15]{chartier2019lersem}.
\end{quote}

Para finalizar essa conceituação da leitura como prática de si numa nova
espacialidade que é o YouTube encarnada em gestos e práticas que são
talvez os mesmos de outros tempos, mas deslocados e ressignificados
pelos mecanismos digitais, como Chartier ilustrou e demonstrou com
\textcite{lajolo_literatura_2017}, os \textit{booktubers} também fazem parte dessa perpetuação do discurso dos livros e da leitura e ajudam a reconstruir
significados da leitura enquanto se marcam como sujeitos leitores e
muitos deles já como sujeitos autores, escrevendo e publicando suas
próprias narrativas, ou seja, livros. Em seu clássico \textit{A ordem dos
livros} \textcite{chartier1999ordem} escreve que: \enquote{A mais interessante pergunta formulada pela história da leitura hoje é, sem dúvida, aquela que diz respeito às relações entre esses três conjuntos de mutações: as tecnológicas, as
formais e as culturais}. \cite[p. 24]{chartier1999ordem}. Percebemos que todos
esses conjuntos das mutações estão também na perspectiva da leitura como
prática desenvolvida pelos comentadores e resenhadores de obras
literárias no YouTube e que num movimento ora intimista e ora de
ostentação mostram suas estantes desde a sua montagem e que, de agora em
diante, passaremos a analisar nas práticas discursivas destes sujeitos
como se dá as emergências destes acontecimentos.


\section{Da intimidade à ostentação: a leitura de si como
exibição, o livro e a leitura como mercado e autopromoção nas teias
capitalistas do YouTube}\label{sec-daintimidadeaostentação}

Para efetivar as análises que aqui apresentamos e constituir o
\textit{corpus} e os recortes que submeteremos à análise, fizemos uma busca
na plataforma YouTube por meio dos termos \enquote{estantes de livros},
\enquote{\textit{booktubers} e estantes} e \enquote{\textit{bookshelf tour}}. Encontramos 
um grande número de vídeos, cerca de 15 mil indicações que aparecem nas buscas,
e optamos por aqueles que tinham mais
visualizações. Depois disso, selecionamos os perfis e os vídeos que se
aproximam da hipótese que aventamos, ou seja, que mostrar a estante é um
movimento de intimidade e de ostentação e que repercute em sentidos
outros para a constituição de si como sujeitos leitores. Feita essa
seleção adotamos quatro recortes para analisarmos neste texto devido a
sua brevidade.

O primeiro vídeo analisado intitula-se \enquote{Minha estante nova: montagem,
projeto, como ficou (antes da organização)}, produzido pela
\textit{booktuber} Mell Ferraz cujo canal intitula-se
\enquote{Literature-se}\footnote{Disponível em:
  \url{https://www.youtube.com/watch?v=NdVi4FZ05sc}}. Nele, Mell Ferraz
apresenta sua nova estante de madeira, ampla, bonita e imponente, sem os
livros. Na apresentação do vídeo a \textit{booktuber} diz:

\begin{quote}
\textbf{Recorte 1:}
Olá pessoal! Eu sou a Mell Ferraz e, sim, você está no canal
Literature-se, apesar deste cenário \textbf{super diferente}. Eu
estou na minha casa e esta é a \textbf{minha estante}, que estou
há muito tempo falando para vocês que um dia ela chegaria, apesar de
todos os \textbf{percalços}, que eu vou contar neste vídeo para
vocês e explicar melhor de onde surgiu \textbf{a ideia}, como
foi \textbf{o desenvolvimento} dessa ideia e também como colocar
ela (sic) em prática e chegarmos até hoje que é a finalização da
montagem desta \textbf{minha estante nova} aqui em casa.
\end{quote}

%De acordo com o "Butterick’s Practical Typography", manual de tipografia, jamais se deve utilizar duas maneiras diferentes de destacamento em um mesmo texto, quiçá ao mesmo tempo. Além disso, segundo o mesmo, as linhas inferiores, underlines, sublinhamentos, são uma maneira de destacar um texto durante o uso de máquinas tipográficas, ou seja, em uso digital é antiquado e pouco-recomendado, em face do itálico e do negrito.

Logo após esta apresentação é veiculada a vinheta de abertura do canal e
em seguida um clipe de 4 minutos mostrando a montagem da estante pelos
marceneiros, com furadeiras e todo material de marcenaria. Nomear a
\textbf{minha estante} é algo comum nos diversos vídeos
encontrados, não só dos recortes aqui apresentados, mas de todo o
\textit{corpus}. O lugar da estante parece simbolizar um sentido não para
todo e qualquer espectador do vídeo, mas para aqueles que também são
adeptos desta prática, outros \textit{booktubers}. As regularidades
enunciativas mostram que os dizeres são direcionados a um público
específico formado pelo que Foucault denominou em \textit{A ordem do
discurso} de sociedades do discurso \enquote{cuja função é conservar ou
produzir discursos, mas para fazê-los circular em um espaço fechado,
distribuí-los somente segundo regras estritas, sem que seus detentores
sejam despossuídos por essa distribuição} \cite[p.~39]{foucault2010ordem}.
Apesar de os vídeos serem abertos para o grande público, percebe-se a
formação de grupos ou nichos que se organizam a partir de estruturas
prévias, visto o número enorme de vídeos de estantes que mostram uma
forma de ostentar para o seu semelhante e compartilharem entre si suas
\enquote{leituras de si}.

O grau da novidade - \textbf{super diferente} - é um modo
próprio de dizer, pois, grosso modo, não há nada de diferente numa
estante, ainda mais vazia, mas o enunciado faz o seu percurso de
sentido, uma vez que não cumpre uma comunicação para qualquer um, mas
para aqueles que esperam e aguardam ansiosamente por vídeos como este.

Ainda nesse recorte, podemos verificar que a sequência -
\textbf{percalços, a ideia, o desenvolvimento} - remetem a um
sentido de projeto. A estante de Mell Ferraz foi projetada e ao marcar a
sua leitura de si, a alma que entrega a si por meios dos livros precisa
definir que passou por todos os passos de um projeto e que a leitura que
desenvolve em seu canal é uma leitura projetada, não é qualquer leitura.
Assim, planejar a estante e mostrar o resultado final deste planejamento
gera uma ligação fortuita com os espectadores do vídeo que acompanham
não só os vídeos mas todo o desenrolar, como se fosse uma websérie, dos
projetos da \textit{booktuber}.

No segundo recorte, assim como a nomeação \enquote{minha estante} do vídeo
anterior, a apresentação do \textit{booktuber} Christian Assunção no vídeo
\enquote{Nova estante: como fazer}\footnote{Disponível em:
  \url{https://www.youtube.com/watch?v=0O1tq\_CCr1g\&list=WL\&index=177}} traz
uma nomeação interessante como a \enquote{A saga da estante}. Vejamos

\begin{quote}
\textbf{Recorte 2:}
Oi gente! A saga da estante tá acabando, né? Primeiro, eu vou contar o
que aconteceu, de ontem pra hoje... Quem me segue lá no Instagram
já sabe o que tá acontecendo, viu? Eu tava procurando uma estante e aí
as estantes tudo cara, muito cara e as que eu tava encontrando eram
frágeis pra livros...
\end{quote}

Em todos os vídeos sobre estantes, o que mais importa não é falar de
leituras, mas falar do material, da compra, da forma espacial como se
vai organizar os livros na estante fazendo uma junção de discursos que
vêm de outros campos como a economia, o comércio, a arquitetura. Neste
recorte, ao intitular como \enquote{a saga da estante} também vemos a mesma
regularidade do sentido de projeto do recorte anterior. Há aqui ainda o
trocadilho com o mundo dos livros, uma vez que sagas são narrativas e
desloca-se o sentido da narrativa dos livros com trama e enredo, para
uma narrativa própria que são os projetos de estante. O autor se marca e
se subjetiva, singularizando dizeres que estão nas regularidades dos
discursos que todos os vídeos de \textit{bookshelf tour}, mais precisamente
de montagem de estantes guardam.

No decorrer do vídeo, Christian Assunção detalha como foi produzindo sua
estante de forma artesanal, como foi à loja da Leroy Merlin e comprou
tábuas avulsas, depois envernizou, afixou as mãos francesas na parede,
todo um percurso que não condiz com um percurso de leitura, mas que gera
interesse, uma vez que o vídeo tem quase 12 mil visualizações. Depois de
apresentar a montagem da estante, o material feito, as medidas de cada
repartição, Christian Assunção enuncia que vai organizar os livros.

\begin{quote}
\textbf{Recorte 3:}
Vou organizar depois \textbf{mostro} pra vocês, aqui mesmo nesse
Vlog...  
\end{quote}


É preciso destacar que a nova estante de Christian é colocada ao lado de
uma outra estante abarrotada de livros e que não é qualquer estante
(assim como todas que são mostradas nos vídeos deste tipo). É uma
estante bonita, na cor branca, de material de boa qualidade. Ninguém
mostra uma estante de ferro, mal organizada, cheia de poeira ou estantes
simples. O verbo mostrar foi destacado na enunciação da organização da
estante com os livros para inferenciar que os sentidos caminham para a
exibição, a ostentação, a necessidade da estante estar organizada para
que possa ser mostrada. O sujeito mostra-se a si, mas não se mostra de
qualquer jeito. Há toda uma encenação, uma cenografia, que precisa ser
performizada para alcançar a circulação do discurso que se quer fazer
veicular e dos sentidos produzidos na referida ação. Logo após, aparecer
com a estante organizada, já com alguns livros, Christian faz outra
enunciação.

\begin{quote}
\textbf{Recorte 4:}
Eu acho que é só isso que eu tenho de \textbf{Cosac...}
Enfim, esses são os Cosac, eu deixei todos aqui nesse lugar.
\textbf{Que lindo, gente! Era o meu desejo} deixá-los sempre
assim \textbf{expostos}.
\end{quote}

Temos neste enunciado uma construção metonímica, ao se tomar a editora
como um todo no lugar dos seus livros publicados. Destacar a marca que
faz referência à editora Cosac Naify de um modo íntimo chamando apenas
pela primeira parte do nome, demonstra intimidade e ao mesmo tempo uma
ordem consumista que coloca o capital sobreposto ao interesse leitor. O
deslumbramento do sujeito - \textbf{\enquote{Que lindo, gente!}} - que revela
os seus desejos, as suas aspirações - \textbf{\enquote{Era meu desejo\ldots}}
coloca um sentido de sacralização da estante, tomada como algo com um
sentimento superior a apenas um suporte de livro. A estante é algo para
ser cultuado, admirado, reverenciado, desde que tudo seja \enquote{exposto}.

Interessante pensarmos nessas novas, não tão novas assim, formas de se
mostrar que se é um leitor entrecruzadas com aquilo que \textcite{galinari2005autorialidade}
chama de \enquote{filosofia espetacular da autorialidade}. O autor parte das
concepções de \textcite{debord_sociedade_1997,subirats_cultura_1989} acerca da \enquote{sociedade do
espetáculo}. Segundo esta filosofia, proposta por \textcite{galinari2005autorialidade}:

\begin{quote}
[\ldots] o livro tende a ser apenas o suporte legitimante de um
autor que se quer arte, que deseja trocar de lugar com a obra para,
finalmente, ser lido e assimilado. Talvez seja justamente aí que se
encontra pressuposto um novo arquétipo de leitor-modelo -- o
leitor-espectador -- , o qual acaba fazendo do livro um duplo fetiche:
(i) de decoração, ou seja, como um enfeite doméstico da sala de estar e
(ii) de autopromoção, isto é, como uma forma cômoda de comunicar
\textit{inteligência}, mediante a exibição de uma \textit{vitrine}
\cite[p. 52–53]{galinari2005autorialidade}.
\end{quote}

O autor está tratando da questão da autoria, mas aqui pode ser
transposta para a questão da leitura e da formação do leitor na
contemporaneidade, uma vez que os destaques em itálico na citação
mostram essa prática que aqui apresentamos nos dizeres/enunciados
materializados linguisticamente em vídeos que se discursivizam. \textcite{galinari2005autorialidade} continua:

\begin{quote}
A produção de toda essa \enquote{cultura}, provavelmente, efetivaria o autor
como uma boa estratégia de \textit{marketing}. Seria, por exemplo, o caso
de autores-celebridades como Jô Soares ou Chico Buarque: o \textit{Xangô
de Baker Street} e o \textit{Budapeste} não venderiam antes pela
paternidade do que por qualquer outra coisa? A autoria não seria o
principal fator responsável pelo sucesso das vendas, ao invés dos
sentidos imanentes à obra? \cite[p. 53]{galinari2005autorialidade}
\end{quote}

Mais uma vez o excerto de \textcite{galinari2005autorialidade} contempla aqui a estratégia de  marketing da qual lançam mão os \textit{booktubers}, pois muitas vezes, ao citar Cosac Naify, por exemplo, o \textit{booktuber} está ganhando um
patrocínio da editora, nem que seja um livro novo na sua caixa de
correio. Em um outro vídeo que compõe o \textit{corpus} da pesquisa, não
aqui analisado em detalhes, a \textit{bookotuber} Isabella Lubrano inicia
seu \textit{bookshelf tour}\footnote{Disponível em:
  \url{https://www.youtube.com/watch?v=abKuqpybE\_0\&t=517s}} com o pedido
para as pessoas curtirem determinada página na internet ou rede social
para que ela possa ser contemplada com uma viagem para a Alemanha,
promoção de um instituto citado pela \textit{influencer} literária.

Neste sentido, respaldados pelas análises já feitas por outros
estudiosos, nossas análises discursivas permitem-nos chegar ao movimento
da \enquote{intimidade} para a \enquote{ostentação}. Uma prática simples e que se marca com um gesto de intimidade, de proximidade com o público leitor,
se desmascara em uma prática de ostentação, de mostrar quem pode mais,
quem tem mais livros, qual estante é mais bonita e mais equipada, numa
competição indireta entre os \textit{booktubers}. Esses apontamentos
conclusivos são possíveis de serem elaborados, uma vez que a gigantesca
quantidade de vídeos desse formato - mostrando a montagem de uma
estante, compra de materiais, desenvolvimento, organização - é muito
frequente no YouTube. Esses vídeos não estão ali por um simples acaso,
ou por uma prática despretensiosa por parte dos seus executores, mas
representam uma prática mercadológica que gera não só capital social
para os seus autores, mas capital material mesmo, movimentam e fazem
aquecer o mercado literário das editoras e autores.


\section{Tentando fechar ideias: uma conclusão inacabada}\label{sec-tentandofecharideias}

Exibir as estantes é uma forma de marcar-se como leitor e que, para o
espectador que assiste a estes vídeos, o sentido é escasso, foge,
escapa, e o que ele muitas vezes busca são indicações de leitura. Alguns
vídeos atentam para isso: destacar livros, comentá-los que estão na
estante e que podem ser uma leitura interessante para aquele que
assiste. Outros vídeos, porém, mostram o material que é feito a estante,
a montagem, práticas que não interferem de forma alguma na aprendizagem
da leitura e no desenvolvimento das habilidades leitoras. Se antes os
professores de ensino de leitura e escrita sofriam quando o aluno optava
por ler o resumo do livro em vez da obra em si, hoje muitos jovens
preferem \enquote{assistir} aos (comentários dos) livros a lê-los.

Os discursos analisados mostram a regularidade de uma prática que
intercala enunciados interdiscursivos, ou seja, enunciados que se
relacionam entre si discursivamente, retomando discursos que se
entrelaçam na formação da intimidade e da ostentação como subjetividades
enunciativas das leituras do eu. Dessa forma, apresenta-se um sujeito
marcando-se a si no discurso, pela circulação e veiculação, numa prática
do mostrar-se que é um leitor, muitos mais do que se é. O fato de ser
realmente ou não um leitor se perde no espaço e no tempo. Não há como
comprovar. O que se indicia, todavia, é que o mostrar antecede o ler e
que os formatos audiovisuais, outras formatações de obras em vídeos, em
cores, imagens, sons repercutem como sentidos estabilizados, congelados
de que o leitor só assim o será se mostrar a si como leitor na grande
rede.

A leitura está em todos os lugares, mesmo naqueles lugares colocados à
margem. Alguns lugares ora tidos como leituras à margem como é o caso
dos canais literários no YouTube, outras vezes se organizam entre si e
formam sociedades fechadas de leitura que não promovem, nem estimulam a
leitura, apenas mantêm a ordem e os sentidos desta prática. Toda e
qualquer leitura, contudo, produz sentido e deve ser analisada dentro de
uma história discursiva da leitura, porque a mídia e a grande massa
muitas vezes censuram e negam essas leituras. Verificar não se isso é
bom ou ruim, mas traduzir interpretações em condições próprias de
enunciados sujeitos imbricados na cultura de si, nos faz aproximar dos
mecanismos do saber, do poder e das tecnologias de si pensando numa
sociedade da resistência, como enfatizava Foucault.


\printbibliography\label{sec-bib}


\end{document}
