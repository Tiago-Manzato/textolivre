\section{Leitura como prática de si: apontamentos foucaultianos}\label{sec-leituracomopraticadesiapontamentosfoucaultianos}

O interesse pelas práticas de si surgiu em \textcite{foucault2014historia} no terceiro volume
da \textit{História da Sexualidade 3: o cuidado de si.} Todo o volume é
dedicado a sondar as artes de si mesmo. \textcite{miranda2006licao}, no
prefácio ou introdução intitulado \enquote{A lição de Foucault} do livro
\textit{O que é um autor?}, coletânea de textos na qual está inserido o
texto foucaultiano \enquote{A escrita de si}, publicado em 1983, pontuam que:
\enquote{As teses de Foucault sobre a `estética da existência' entroncarão
neste novo espaço, que é um espaço onde a própria crítica do que somos e
do que fizeram de nós emerge como um problema histórico} \cite[p. 11]{miranda2006licao}. No capítulo segundo do terceiro volume da
\textit{História da Sexualidade,} intitulado \enquote{A cultura de si}, \textcite{foucault2014historia}
descreve detalhadamente do que se trata a \enquote{arte da existência} e o que
seria esta \enquote{cultura de si}.

\begin{quote}
Ora, é esse tema do cuidado de si, consagrado por Sócrates, que a
filosofia ulterior retomou, e que ela acabou situando no cerne dessa
\enquote{arte da existência} que ela pretende ser. É esse tema que,
extravasando de seu quadro de origem e se desligando de suas
significações filosóficas primeiras, adquiriu progressivamente as
dimensões e as formas de uma verdadeira \enquote{cultura de si}. Por essa
expressão é preciso entender que o princípio do cuidado de si adquiriu
um alcance bastante geral: o preceito segundo o qual convém ocupar-se
consigo mesmo é, em todo caso, um imperativo que circula entre numerosas
doutrinas diferentes; ele também tomou a forma de uma atitude, de uma
maneira de se comportar, impregnou formas de viver; desenvolveu-se em
procedimentos, em práticas e em receitas que eram refletidas,
desenvolvidas, aperfeiçoadas e ensinadas; ele constituiu, assim, uma
prática social, dando lugar a relações interindividuais, a trocas e
comunicações e até mesmo a instituições; ele proporcionou. enfim, um
certo modo de conhecimento e a elaboração de um saber \cite[p. 58]{foucault2014historia}.
\end{quote}

A citação de Foucault corrobora a percepção de \textcite{miranda2006licao} uma vez que a cultura de si se dá no desenvolvimento da história e por meio de
práticas diversas. Como o cuidado de si tem um alcance geral, podemos
perceber nas práticas de mostrar as estantes todo um modo próprio de
mostrar a si que arrola um processo de subjetivação, ou seja, de
constituição dos sujeitos imbricados em diferentes práticas discursivas.

Em toda sua obra, Foucault procura debater sobre os diferentes modos de
constituição do sujeito (seja quanto às formas de sujeição, seja quanto
às aberturas e às possibilidades de recusa e de resistência, seja ainda
quanto à constituição ética de si). É esta última - a constituição ética
de si - que o filósofo escreve no capítulo \enquote{A cultura de si}, desde os
gregos para depois chegar aos primeiros séculos da era cristã no
restante do livro. Também não há como isolar, na sua concepção de
discurso, aquilo que ele pensa sobre sujeito e o que afirma sobre
relações de poder.

O sujeito na visão foucaultiana é tomado como posição. As posições
ocupadas pelo sujeito advém da inscrição em determinadas formações
discursivas. Portanto, a título de compreendermos o sujeito leitor na
contemporaneidade e entender em quais formações discursivas seus
discursos, ou seus dizeres, se inscrevem atentamos para a conceituação
que \textcite{foucault2009arqueologia} já traz desde \textit{A arqueologia do saber}. Este conceito está ligado com a função enunciativa, e para Foucault o enunciado não é a frase ou a proposição. Ele assim o define: \enquote{Descrever uma formulação enquanto enunciado não consiste em analisar as relações entre o autor e o que ele disse (ou quis dizer, ou disse sem querer), mas em determinar
qual é a posição que pode e deve ocupar todo indivíduo para ser seu
sujeito} \cite[p. 108]{foucault2009arqueologia}.

Qual posição o(s) sujeito(s) ocupa(m) diante das práticas de leitura?
De qual sujeito leitor moderno podemos falar nos entremeios destes
canais de \textit{booktubers} no YouTube que analisamos? Neste texto, a
maioria dos \textit{booktubers} analisados são famosos e com grande
visualização dentro desta famosa rede social audiovisual. Há também
produtores de vídeos deste tipo do universo literário que não são
famosos, mas também marcam as suas leituras de si na grande rede e com
isso pode-se ter um movimento mais democrático da crítica literária e do
incentivo à leitura. Ao mesmo tempo em que há esse movimento
democrático, podemos perceber uma ordem da repetição e se há tantos
canais no YouTube deste mesmo nicho e com as mesmas práticas que são as
\textit{bookshelf tour} tal acontecimento vão ao encontro daquilo que
\textcite{chartier2019lersem} pontua como \enquote{A ilusão biográfica}, que \enquote{conduz o indivíduo a pensar-se como irredutivelmente único e singular quando na verdade seu discurso ou sua memória se pautam em modelos amplamente compartilhados} \cite[p. 8]{chartier2019lersem}. Essa ilusão é muito característica nas práticas que vamos analisar logo mais.

Enfatizando a revolução do texto eletrônico, \citeauthor{chartier2019lersem} no mesmo estudo chama a atenção para aspectos que vão estar presentes nesta construção de si que traz elementos não do universo dos livros, mas da afetividade que como estamos designando carrega sentidos de intimidade. O
historiador destaca:

\begin{quote}
[...] ela [a revolução do texto eletrônico] redefine a
materialidade das obras porque desata o laço visível que une o texto e o
objeto que o materializa e a partir do qual chega ao leitor, não se
limitando mais ao autor ou ao editor o domínio sobre a forma e o formato
das unidades textuais que se quer ler. Assim, é todo o sistema de
percepção e de uso dos textos que se encontra transformado \cite[p. 13–14]{chartier2019lersem}.
\end{quote}

É nesta transformação do sistema de percepção dos livros e da leitura
que os \textit{booktubers} se inscrevem. Não se pode pensar a percepção
destes espaços apenas com um critério avaliativo se isso é bom ou ruim,
mas é preciso observar a sua existência, as condições de possibilidade
(cf. \textcite{foucault2009arqueologia}) que pactuam para que a cultura de si aí se exerça como posicionamento de um sujeito que lê a si e ao mundo. Chartier
resgata ainda \textcite{lajolo_literatura_2017} oferecendo-nos suporte de
análise desta transformação que são os \textit{booktubers} que não mais
mostram apenas livros ou resenhas clássicas, mas até a montagem de suas
estantes ou de sua biblioteca. Vejamos:

\begin{quote}
Por outro lado, a inventividade dos criadores (particularmente no campo
da literatura infantil e juvenil) aproveita as possibilidades digitais
para propor gêneros, objetos, criações irredutíveis à forma impressa. As
inovações não se limitam a introduzir no \enquote{livro} os gêneros da rede
(\textit{e-mails}, \textit{blogs}, \textit{links}). Elas também produzem criações, segundo as
expressões de \textcite{lajolo_literatura_2017}, explorando o
\enquote{hibridismo de linguagens} ou \enquote{amálgamas de linguagens}. O \enquote{site} substitui o livro, a liberdade do leitor, que pode escolher entre opções narrativas, substitui o absolutismo do texto, e, muitas vezes, a
gratuidade do acesso substitui o comércio editorial. Essa aposta não é
sem importância, pois pode nos levar tanto à introdução, na textualidade
eletrônica, de alguns dispositivos capazes de perpetuar os critérios
clássicos de identificação de obras tal como é no impresso, relativos a
sua identidade e propriedade, quanto também ao abandono dessas
categorias para inventar uma nova maneira de compor novas produções
estéticas que explorem uma \enquote{plurimidialidade}, mais rica que a simples
relação entre texto e imagens e que localizem o leitor numa posição que
lhe permita fazer escolhas ou participar do processo criativo. Na
primeira hipótese, o desaparecimento do livro como objeto material não
significaria seu desaparecimento como modalidade de discurso que supõe a
percepção e compreensão da coerência e totalidade da obra. Na segunda
hipótese, ao contrário, se desenha uma nova ordem do discurso na qual o
leitor produz, corta, desloca, associa ou reconstrói fragmentos móveis e
maleáveis \cite[p. 14–15]{chartier2019lersem}.
\end{quote}

Para finalizar essa conceituação da leitura como prática de si numa nova
espacialidade que é o YouTube encarnada em gestos e práticas que são
talvez os mesmos de outros tempos, mas deslocados e ressignificados
pelos mecanismos digitais, como Chartier ilustrou e demonstrou com
\textcite{lajolo_literatura_2017}, os \textit{booktubers} também fazem parte dessa perpetuação do discurso dos livros e da leitura e ajudam a reconstruir
significados da leitura enquanto se marcam como sujeitos leitores e
muitos deles já como sujeitos autores, escrevendo e publicando suas
próprias narrativas, ou seja, livros. Em seu clássico \textit{A ordem dos
livros} \textcite{chartier1999ordem} escreve que: \enquote{A mais interessante pergunta formulada pela história da leitura hoje é, sem dúvida, aquela que diz respeito às relações entre esses três conjuntos de mutações: as tecnológicas, as
formais e as culturais}. \cite[p. 24]{chartier1999ordem}. Percebemos que todos
esses conjuntos das mutações estão também na perspectiva da leitura como
prática desenvolvida pelos comentadores e resenhadores de obras
literárias no YouTube e que num movimento ora intimista e ora de
ostentação mostram suas estantes desde a sua montagem e que, de agora em
diante, passaremos a analisar nas práticas discursivas destes sujeitos
como se dá as emergências destes acontecimentos.

