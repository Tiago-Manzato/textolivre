\section{O Interacionismo sociodiscursivo e alguns conceitos teóricos}\label{sec-ointeracionismo}

Nesta seção, apresentaremos alguns conceitos oriundos do quadro
teórico-metodológico proposto pelo Interacionismo Sociodiscursivo que
auxiliaram o professor-formador e os estagiários na elaboração do curso.
Abordaremos, apenas os conceitos de texto, gênero textual, modelo de
análise textual, modelo didático de gênero e capacidades de linguagem,
que foram os mais produtivos para nosso trabalho.

O ISD é um quadro teórico proposto na Unidade de Didática das Línguas da
Faculdade de Psicologia e Ciências da Educação da Universidade de
Genebra, Suíça, e se solidificou na obra de \textcite{bronckart_atividade_1999}. Os
estudos propostos nesta corrente baseiam-se na perspectiva sobre o
desenvolvimento de pessoas trazida por \textcite{vygotski_pensee_1997} e também nas
contribuições de \textcite{bakhtin_estetica_2006}, no que diz respeito aos gêneros
discursivos e sua forma de observá-los e analisá-los. O ISD acredita que
as condutas humanas são resultado de processos sócio-históricos
marcados, principalmente, pelo uso da linguagem. Nesta corrente teórica,
defende-se que os seres humanos agem em sociedade por meio da linguagem.

Para \textcite{bronckart_atividade_1999}, o agir social linguageiro se dá por meio de
textos. O texto seria, portanto, uma unidade de comunicação ou de
interação geral, ou seja, uma unidade do agir que apresenta uma mensagem
organizada que visa a produzir um efeito sobre o destinatário, num
determinado espaço e num determinado tempo. Segundo \textcite[p. 78]{bronckart_atividade_1999}, visto que o conceito de texto designa \enquote{toda unidade de produção
verbal situada, acabada e autossuficiente (do ponto de vista acional ou
comunicacional)} e na medida em que \enquote{[\ldots] todo texto se inscreve
necessariamente num conjunto ou num gênero}, o melhor seria adotar a
expressão gênero de texto no lugar de gênero do discurso\footnote{Embora
	conheçamos as implicações de ordem teórica que as duas terminologias
	encerram, optamos, assim como \textcite{bronckart_atividade_1999}, pela utilização da
	expressão \enquote{gêneros de texto/textuais.}}, sem prejuízo de sentido.
Assim, podemos afirmar que a noção de gênero de texto para o autor é
equivalente à noção de gênero de discurso proposta por \textcite[p. 262, grifos do autor.]{bakhtin_estetica_2006}, segundo a qual \enquote{[\ldots] cada campo de
utilização da língua elabora seus \emph{tipos relativamente estáveis} de
enunciados, os quais denominamos \emph{gêneros do discurso}}, que
apresentam, por sua vez, três características fundamentais: o tema, a
estrutura composicional e o estilo.

Com o intuito de analisar textos pertencentes a diferentes gêneros,
\textcite{bronckart_atividade_1999,bronckart_atividade_2006} propõe um modelo de análise da arquitetura
interna dos textos, composto de três níveis estruturais: a
infraestrutura geral do texto, que pode ser dividida em plano global do
texto, tipos de discurso e de sequências; os mecanismos de
textualização, em que se observam a conexão, a coesão nominal e a coesão
verbal; os mecanismos enunciativos, constituídos das modalizações e das
vozes presentes nos textos. Além dos parâmetros descritos, \textcite{bronckart_atividade_1999,bronckart_atividade_2006} postula, antes da análise textual propriamente dita, a
necessidade de um estudo sobre o contexto de produção que deu origem ao
texto, em nível mais geral e em nível da ação de linguagem que embasa o
texto.

Antes da produção de qualquer gênero, o produtor do texto deve mobilizar
alguns elementos que farão parte da estrutura interna do texto em
questão, isto é, compreender de maneira adequada a ação de linguagem que
deu origem ao texto e que é suscetível de influenciar os três níveis.
Assim, no nível do contexto de produção, analisa-se, primeiramente, o
contexto imediato que deu origem ao texto. Esse contexto pode ser
interpretado de duas formas: a partir de seus aspectos concretos e
objetivos relativos ao mundo físico da produção do texto (o emissor; o
receptor; o local físico; o momento de produção); a partir dos aspectos
sociossubjetivos, que compreendem os estatutos dos aspectos anteriores
(o enunciador; o destinatário; o local social; o objetivo da interação).
É importante lembrar que se deve observar não só o contexto mais
imediato que provocou a ação de linguagem, mas também o contexto de
produção mais amplo, dito sócio-histórico.

Levantadas algumas hipóteses sobre o contexto de produção, deve-se
observar o primeiro nível de análise, ou seja, a infraestrutura geral do
texto, composto pelo plano global dos conteúdos temáticos, pelos tipos
de discurso e pelas sequências textuais. O plano global deve ser
entendido como se fosse um resumo do texto, uma lista de conteúdos. Os
tipos de discurso são: discurso interativo, discurso teórico, relato
interativo e narração. Ainda neste primeiro nível de análise, \textcite{bronckart_atividade_1999} propõe a análise das sequências presentes no texto, baseando-se
nos trabalhos de \cite{adam_1992} e classificando-as da seguinte forma:
descritivas, narrativas, argumentativas, explicativas, dialogais e
descritivas de ações.

O segundo nível de análise apresenta a análise da coesão verbal, nominal
e os mecanismos de conexão do texto. Em estudos mais recentes, \textcite{bronckart_teorias_2021} considera que a coesão verbal é parte integrante da análise dos
tipos de discurso, já que sua compreensão é fundamental para fazer a
distinção entre o mundo conjunto e disjunto. Já a coesão nominal
compreende os processos de anáfora e catáfora realizados por meio de
retomadas nominais e pronominais. Os mecanismos de conexão são aqueles
que articulam as macroideias do texto, marcados por meio de conectivos,
conjunções, advérbios, grupos preposicionais ou nominais.

O terceiro e último nível de análise refere-se aos mecanismos
enunciativos, por meio da análise das vozes, implícitas ou explícitas,
veiculadas no texto e que podem ser do(s) autor(es) empírico(s), de
personagens ou sociais. Articuladas às vozes observam-se as
modalizações, que podem ser de natureza lógica deôntica, apreciativa e
pragmática.

Adotando a proposta teórica formulada por \textcite{bronckart_notion_1999}, \textcite{dolz_pour_1998} desenvolveram pesquisas que visam a estudar o processo
de transposição didática do conceito de gêneros para o contexto
educacional. Consideravelmente difundidos no campo da Didática das
Línguas, esses estudos consideram os gêneros textuais como mediadores
essenciais das atividades humanas e, principalmente, (mega)instrumentos
psicológicos, em perspectiva vygotskiana, possibilitando a comunicação
em múltiplas práticas sociais. Para \textcite{schneuwly_generos_2004}, os gêneros
podem ser definidos pelos conteúdos que são ou que se tornam dizíveis
por meio deles, por uma estrutura comunicativa que lhes é peculiar, por
um conjunto de unidades de ordem linguística que revelam traços da
posição enunciativa do enunciador, pelo conjunto de sequências textuais
e pelos tipos discursivos que compõem a estrutura textual. Ainda que
possuam uma natureza maleável e bastante diversificada, os gêneros
textuais podem ser instrumentos bastante proveitosos ao
ensino-aprendizagem de línguas, em virtude seu caráter cultural e
didático, por meio do qual as práticas de linguagem podem ser
incorporadas nas atividades pessoais, profissionais, familiares,
religiosas, etc. dos estudantes.

Com vistas à transposição didática do conceito de gêneros textuais para
o ensino de línguas, \textcite{schneuwly_generos_2004} apresentam dois conceitos
fundamentais para o desenvolvimento das capacidades de linguagem dos
alunos: o modelo didático de gênero (MDG) e a sequência didática (SD).
No entanto, o material elaborado para o curso que descrevemos não foi
estruturado sob a forma de SD por não estar voltado apenas às atividades
de produção (oral e escrita), mas também às atividades de compreensão
(oral e escrita). Isso não evitou, contudo, que as atividades destinadas
à produção fossem trabalhadas a partir das características do gênero
textual em questão, mapeadas em um MDG.

O MDG é um instrumento descritivo que deve ser elaborado pelo docente
para nortear as práticas de produção oral e/ou escrita dos alunos em
determinado gênero textual. Seu principal objetivo é elencar as
principais características que constituem um gênero textual e
selecionar, em seguida, aquelas que podem e devem ser ensinadas. De
acordo com \textcite{de_pietro_modedidactique_2003}, o MDG se apresenta por meio da intersecção de três dimensões distintas: uma estrutura (a definição
geral do gênero; os parâmetros do contexto comunicativo; os conteúdos
específicos; a estrutura textual global; as operações linguageiras e
suas marcas linguísticas); uma construção (as práticas sociais de
referência; a literatura a respeito dos gêneros; as práticas de
linguagem dos alunos; as práticas escolares). Pelos interesses deste
texto, não apresentaremos os MDG elaborados pelos estagiários para os
gêneros trabalhados no material.

Além de organizar as atividades de ensino de um gênero textual, o MDG
serve também para orientar o desenvolvimento de capacidades de linguagem
dos alunos, ou seja, as capacidades de agir em diferentes práticas
sociais por intermédio da linguagem, nosso principal interesse com a
elaboração do material. As capacidades de linguagem \cite{dolz-mestre_acquisition_1993} evocam as aptidões requeridas do aprendiz para a
produção de um gênero numa situação de interação determinada e podem,
também, ser transpostas para a produção de outros gêneros. As
capacidades de linguagem estão em constante interação durante o processo
de produção de um gênero e podem ser divididas, de maneira didática, em
três categorias: capacidades de ação, capacidades discursivas e
capacidades linguístico-discursivas.

As capacidades de ação tratam das adaptações necessárias às
características da situação de produção do texto e a mobilização dos
conteúdos no momento da produção. Tanto o leitor quanto o produtor do
texto devem adequar sua produção de linguagem ao contexto de produção,
levando em conta dois parâmetros: um físico e concreto (emissor;
receptor; lugar físico da produção; momento de produção); um
sociossubjetivo (enunciador; destinatário; lugar social em que o texto é
produzido; objetivos da interação), fatores que influenciam diretamente
a organização do texto pertencente a um gênero em questão. As
capacidades discursivas envolvem modelos discursivos sobre a
infraestrutura global do texto, ou seja, a organização geral do texto
que se dá por meio de quatro tipos de discurso e das seis sequências
predominantes em um determinado gênero, apresentados anteriormente.
Enfim, as capacidades linguístico-discursivas desenvolvem-se pelo
domínio, da parte do produtor do texto ou do leitor, de operações
psicolinguísticas e de unidades linguísticas não só mediante categorias
sintáticas, fonéticas ou morfológicas, mas também por meio de
posicionamentos enunciativos do autor do texto, gerenciamento de vozes,
modalizações, operações de textualização, construção de enunciados e
escolha de itens lexicais.

Isso posto, um último ponto deve ainda ser discutido a respeito das
capacidades de linguagem. Diferentemente da maioria dos materiais de
ensino publicados atualmente e destinados ao aprendizado de idiomas, a
proposta de trabalho deste curso é inovadora no sentido de não buscar
desenvolver nos alunos a noção de competência, mas desenvolver as
capacidades de linguagem. De acordo com \textcite{bronckart_notion_1999}, a noção de competência linguística foi elaborada por Chomsky, na década de 1950,
em um de seus textos fundadores daquilo que se denominou nas ciências
humanas de \enquote{revolução cognitiva}. Grosso modo, a noção de competência
tende a se concentrar em dispositivos de linguagem inatos e universais,
de natureza biológica e cognitiva, inscritos no potencial genético do
indivíduo, e expressos por uma gramática interna que, ao se eximir de
todo determinismo sócio-histórico, define-se absoluta e independente de
qualquer contexto concreto.

Em perspectiva oposta à descrita, o ISD ao se inscrever em uma
perspectiva vygotskiana, que ressalta o papel social na aprendizagem,
acredita que qualquer ação de linguagem é situada histórica e
socialmente, como assevera \textcite[p. 99]{bronckart_atividade_1999}: \enquote{[\ldots] a noção de ação de linguagem reúne e integra os parâmetros do contexto de produção e do conteúdo temático, tais como um determinado agente os mobiliza, quando empreende uma intervenção verbal.}. Logo, na
perspectiva interacionista sociodiscursiva, ao produzir textos
pertencentes a um gênero, o produtor mobiliza capacidades de linguagem
(de ação, discursivas, linguístico-discursivas) que não inatas e
biológicas, como se mostra a noção de competência em Chomsky, mas podem
ser aprendidas e desenvolvidas pelos alunos durante a aprendizagem de
determinado gênero. É nesse sentido que os gêneros textuais podem ser
vistos como instrumentos, em sentido vygotskiano, para o desenvolvimento
dessas capacidades.