\documentclass[english]{textolivre}

% metadata
\journalname{Texto Livre}
\thevolume{17}
%\thenumber{1} % old template
\theyear{2024}
\receiveddate{\DTMdisplaydate{2023}{8}{31}{-1}}
\accepteddate{\DTMdisplaydate{2023}{11}{15}{-1}}
\publisheddate{\DTMdisplaydate{2024}{4}{17}{-1}}
\corrauthor{Ronaldo Corrêa Gomes Junior}
\articledoi{10.1590/1983-3652.2024.47924}
%\articleid{NNNN} % if the article ID is not the last 5 numbers of its DOI, provide it using \articleid{} commmand 
% list of available sesscions in the journal: articles, dossier, reports, essays, reviews, interviews, editorial
\articlesessionname{dossier}
\runningauthor{Braga and Gomes Junior}
%\editorname{Leonardo Araújo} % old template
\sectioneditorname{Daniervelin Pereira}
\layouteditorname{João Mesquita}

\title{"In the palm of your hand": the ecological and complex nature of teacher agency}
\othertitle{"Na palma da mão": a natureza ecológica e complexa da agência do professor}

\usepackage[inline]{enumitem}
%\usepackage{easyReview}

\author[1]{Junia de Carvalho Fidelis Braga ~\orcid{0000-0002-8450-2061}\thanks{Email: \href{mailto:juniadecarvalhobraga@gmail.com}{juniadecarvalhobraga@gmail.com}}}
\author[1]{Ronaldo Corrêa Gomes Junior~\orcid{0000-0003-4165-8629}\thanks{Email: \href{mailto:ronaldocgomes@gmail.com}{ronaldocgomes@gmail.com}}}
\affil[1]{Universidade Federal de Minas Gerais, Faculdade de Letras, Belo Horizonte, MG, Brasil.}

\addbibresource{article.bib}

\begin{document}
\maketitle
\begin{polyabstract}
\begin{abstract}
The potential of mobile phones as a resource to mediate teaching and learning opportunities has been the subject of research in the educational context. However, the role of these devices in teacher education, especially regarding the exercise of teacher agency, has been insufficiently investigated. The concept of agency has been discussed from different perspectives, and many studies in this area share the idea that agency is one of the characteristics of human behavior. In this paper\footnote{The authors acknowledge the support received from FAPEMIG (Minas Gerais State Agency for Research and Development) in the form of scholarships and financial support.}, we attempt to discuss the concept of agency through the lens of the ecological approach and complexity, perspectives that shed light to the way human beings’ exercise of agency emerges as agents who can influence and be influenced by their context. The aims of this study were to identify: i) instances of agency in the actions of pre-service teachers; ii) actions mediated by the use of mobile phones by these teachers; and iii) intrapersonal issues that can influence the agency of these teachers. Data was collected through a semi-structured questionnaire in undergraduate classes on the use of digital technologies in language teaching and learning. The data was then processed and coded using Atlas-TI software, and the analysis and discussion of the data was based on the characteristics of agency as a complex system. With this study, we hope to demonstrate the multifaceted nature of the concept of agency as these mobile technologies are integrated into teaching practices and to underscore its importance for language-teacher pre-service initial education.



\keywords{Teacher education \sep Mobile technologies \sep Agency \sep Complexity \sep Ecological perspective}
\end{abstract}

\begin{abstract}
O potencial dos celulares como recurso para mediar oportunidades de ensino e aprendizagem tem sido tema de pesquisas no contexto educacional. No entanto, o papel desses dispositivos no contexto de formação, especialmente com relação ao exercício de agência do professor, é tema pouco explorado. O conceito de agência tem sido discutido sob diferentes perspectivas e muitos estudos nessa área compartilham o pensamento de que agência é uma das características do comportamento humano. Neste trabalho, buscamos discutir o conceito de agência pelas lentes da perspectiva ecológica e da complexidade, por se tratar de perspectivas que esclarecem a forma como o exercício da agência pelos seres humanos emerge como agentes que podem influenciar e serem influenciados pelo seu contexto. O objetivo deste estudo foi o de identificar: i) instâncias de agência na atuação de professores em formação inicial; ii) ações mediadas pelo uso de celulares por esses professores; e iii) questões intrapessoais que podem influenciar a agência desses professores. Os dados foram gerados por meio de um questionário semi-estruturado em turmas de graduação e especialização sobre o uso de tecnologias digitais no ensino e aprendizagem de línguas. Esses dados foram tratados e tabulados com o apoio do software Atlas-TI e a análise e discussão dos dados foram pautadas em características da agência como sistema complexo. Com este estudo, esperamos evidenciar a natureza multifacetada do conceito de agência em se tratando da integração de tecnologias móveis na prática docente, destacando a sua importância para a formação inicial de professores de línguas.

\keywords{Formação de professores \sep Tecnologias móveis \sep Agência \sep Complexidade \sep Perspectiva Ecológica}
\end{abstract}
\end{polyabstract}

\section{Introdução}\label{sec-introdução}

Neste artigo, buscamos apresentar um relato de experiência sobre o
processo de elaboração de atividades didáticas para um curso
\emph{online} de natureza assíncrona, oferecido no ambiente
virtual Moodle, destinado ao ensino de francês língua estrangeira (FLE)
para alunos de um centro de extensão de uma universidade federal
brasileira.

O curso foi desenvolvido no quadro de um programa de formação de
professores em línguas clássicas e modernas que busca congregar três
vertentes da vida acadêmica: o ensino, por meio de práticas
didático-pedagógicas desenvolvidas pelos alunos-estagiários (doravante
estagiários); a pesquisa, com a possibilidade de os estagiários se
dedicarem ao estudo aprofundado de temas relacionados ao ensino e a
aprendizagem de línguas estrangeiras; a extensão, já que os cursos
ministrados são abertos à comunidade externa e interna da universidade
em questão. Neste programa de formação, há um projeto destinado ao
ensino do francês, em vários níveis, que propõe uma formação continuada
didático-pedagógica e linguística aos estagiários. Essa formação foi
ministrada por um professor efetivo da habilitação em francês que, no
centro de extensão, atuava tanto como supervisor do idioma francês
quanto como professor-formador. Para o curso que descreveremos, dois
estagiários mostraram-se interessados em elaborar o material e
receberam, durante o período de trabalho, uma bolsa acadêmica de
extensão.

Durante a formação, coube ao professor-formador atuar junto aos
estagiários e concentrar-se na elaboração do material didático para o
curso, em suas mais diferentes etapas, pelo prisma teórico-metodológico
proposto pelo quadro do Interacionismo Sociodiscursivo \cite{bronckart_atividade_1999,bronckart_teorias_2021,schneuwly_generos_2004,graca_da_2023,tocaia_leitura_2019,tocaia_letramento_2022}. Durante a formação, também foram realizados encontros semanais de orientação, momento em que foram
lidos e discutidos conceitos teórico-pedagógicos relativos à teoria
sociointeracionista. O objetivo central desses encontros era construir,
junto aos estagiários, um conjunto de conhecimentos comuns,
materializados não só pela discussão de propostas e práticas
pedagógicas, mas também pela discussão de conceitos da corrente teórica
em questão. Ao final desta etapa de estudo, um projeto de ação, que
compreendia a elaboração do material para o primeiro módulo do curso,
intitulado \emph{Français Online 1}, foi executado.

Neste projeto de ação, quatro grandes pressupostos nortearam a
elaboração do material para o curso. O primeiro relacionou-se ao fato de
que aprender uma língua, seja ela a própria língua materna (LM) ou uma
língua estrangeira (LE), é aprender a se comunicar, isto é, agir por
meio da linguagem, amparado pela produção e compreensão de gêneros
textuais diversos, nas múltiplas situações sociais. Neste primeiro caso,
a preocupação foi preparar os alunos que desejavam aprender o idioma
francês a dominar operações linguageiras que subsidiassem a compreensão
e a produção de gêneros textuais, uma vez expostos a uma variedade de
discursos orais e escritos em LE. Ao prepararmos o aluno para o domínio
da LE em diferentes situações sociais, buscávamos auxiliá-lo a construir
representações cada vez mais complexas da língua-alvo, encorajando-o a
transpor seus limites no aprendizado e a agir socialmente em LE. O
segundo pressuposto dizia respeito ao desenvolvimento de capacidades de
linguagem \cite{dolz-mestre_acquisition_1993,schneuwly_generos_2004}, que
poderiam ser mobilizadas não apenas para a produção dos gêneros textuais
propostos no material do curso, mas que possivelmente seriam transpostas
a outros gêneros textuais diferentes daqueles estudados, desde que
apresentassem aspectos contextuais, discursivos e
linguístico-discursivos similares. O terceiro era o desejo, enquanto
professor-formador, de que os próprios estagiários, por meio de uma
experiência de transposição didática \cite{chevallard_transposition_1981}, entendida, grosso modo, como um conjunto de transformações que um dado conhecimento de cunho científico (saber científico) sofre ao ser transposto para o ensino escolar (saber escolar), tivessem participação efetiva na
construção do material, o que lhes conferiria o papel de igualmente
responsáveis pelo processo de elaboração e condução do material e das
práticas didáticas a ele relacionadas. O quarto e último referia-se ao
desafio de se construir um curso virtual assíncrono para o ensino de
francês à distância que buscasse garantir a interatividade no processo
de ensino e aprendizagem do idioma, e que mesmo mediante a ausência
física do professor, propusesse um material didático organizado a partir
de textos autênticos na LE, de método indutivo, progressivo no
desenvolvimento dos conhecimentos e das capacidades de linguagem e,
principalmente, em permanente diálogo com os alunos participantes por
meio da plataforma Moodle.

Dessa maneira, para relatar nossa experiência, este texto organiza-se em
cinco partes, além desta introdução: primeiramente, faremos uma
apresentação da plataforma Moodle e de suas possíveis funcionalidades
destinadas ao ensino de línguas; em seguida, descreveremos os
pressupostos teórico-metodológicos na perspectiva sociointeracionista
que orientaram a elaboração do curso em questão; posteriormente,
elaboraremos algumas considerações sobre o contexto de produção do curso
para, então, descrevermos seu processo de elaboração, suas
características essenciais, sua organização estrutural, sua metodologia
de ensino e sua relação com o desenvolvimento das capacidades de
linguagem; por fim, seguem as considerações finais.
\section{Ecological approach and complexity theory}\label{sec-Ecological approach and complexity theory}

According to \textcite[p. 11]{vanlier2004}, the ecological approach “looks at the entire situation and asks: What is it in this environment that makes things happen the way they do?” For the author, this approach encompasses the study of the context and all the things going on around it, at school, in the classroom, at the desk, and so on. In addition, the ecological approach is not limited to spaces; it encompasses movement, process, and action. As \textcite{vanlier2010a} explains, ecology studies the relationships and interactions between elements in the environment or ecosystem. For both the ecological perspective and complexity theory, the context and the system are inseparable. The environment influences — and is influenced by — other systems in which it is nested. For the purposes of this investigation, we consider it essential to look at these relationships and their influence on the actions that unfold in a situated context, particularly those regarding agency.


\textcite[p. 43]{mercer2012} reminds us that contexts are not static or monolithic. Rather, “they need to be understood as representing dynamic systems composed of a multitude of components which can combine and interact in complex, unique ways.” Drawing on \textcite{vanlier2004,mercer2012} emphasizes that the concept of affordance is fundamental for understanding agency since affordances “represent the interaction between contextual factors (micro and macro-level structures, artifacts) and learners' perceptions of them and the potential for learning inherent in this interaction.” As \textcite[p. 127]{gibson1986} points out, affordances are: "what it [the environment] 'offers' to the animal, what it 'provides' or 'furnishes', either for good or ill”. For \textcite[p. 43]{mercer2012} , “[a]gency thus emerges from the interaction between resources and contexts and the learners’perceptions and use of them.”


The adoption of the ecological approach to understanding human systems is not novel. It has gained momentum, however, with discussions on the principles of complexity theory, which perceives the world as integrated, based on established relationships. Like the ecological perspective, complexity theory involves agents in an interaction process in which the dynamics of systems, agents, and context are inextricably intertwined.

Complexity theory has come to consolidate its importance in different areas, including Education. There is consensus that although there are different approaches to looking into a phenomenon, all the perspectives seem to share the idea that “all complexity perspectives embrace organic, holistic models composed of complex dynamic systems as opposed to more traditional linear models” \cite[p. 43]{mercer2012}. In essence, a complex system usually has elements or agents that interact with each other, is open, and adapts based on contextual demands, whether internal or external. From these interactions, there is constant action and reaction between agents. The system evolves dynamically, that is, nothing in it is fixed. According to \textcite[as quoted in Mercer, 2012, p. 43]{cilliers2010}
“one of the most defining features of complex systems is that they have emergent properties, for instance, properties which cannot simply be reduced to properties of characteristics of components in the system.” Interactions and adaptations enable the agents of a system to self-organize, leading to the emergence of new patterns and behaviors. A new order spontaneously arises, and it does not seem to be governed by known physical laws.

We consider that discussions that recognize the relationships established in their contexts and surroundings are relevant to this study because they investigate systems that learn, that is, systems that organize and reorganize themselves based on their contextual demands, their network relationships, and dynamics. We also believe that such discussion can shed light on social practices mediated by mobile devices in the context of language teacher education.
\section{Mobile learning}\label{sec-mobilelearning}

Current language learning methodologies are strongly influenced by pedagogies that emphasize digital literacies \cite{dudeney2016} and multiliteracies \cite{cazden1996}, integrating a multitude of technologies and societal language practices that coexist within the cosmopolitan milieu. The advent of mobile technology, along with its linguistic facets, aligns seamlessly with this trend.

Understanding the advantages of mobile learning is pivotal for elucidating its conceptual framework. As outlined by \textcite[p. 1]{kukulska2010}, the main advantages of this mode of learning include:

\begin{itemize}
	\item immediate access to information, social networks, and situation-relevant help;
	\item flexible use of time and space for learning;
	\item continuity of learning between different settings;
	\item good alignment with personal needs and preferences;
	\item easy creation and sharing of simple content like photos, videos and audio recordings; e
	\item greater opportunity for sustained language practice while carrying out activities such as walking, waiting, or commuting.
\end{itemize}

Learning with mobility can afford individuals opportunities to use the language as they create, use, and share digital content within situated and contextualized practices.

Regarding language learning, \textcite{braga2017a} emphasizes that the ubiquitous presence of personal mobile devices facilitates the alignment of these objects with the perception of affordances, giving rise to spontaneous and local activities made up of resources available on the Internet. Furthermore, \textcite[p. 49–50]{braga2017b} consider mobile learning as

\begin{quote}
	a movement that goes beyond a mere intervention 'in' pedagogical practice. More than that, this mode of teaching and learning points to a change 'of' the practice itself, in such a way that the multiple contexts in which students and teachers circulate become sources of resources explored via mobile technologies for the teaching and learning of languages.
\end{quote}


Hence, it is important to emphasize that the mere integration of technologies does not ensure the productivity and authenticity of mobile language learning practices. A paradigm shift in teaching and learning is necessary, particularly one that acknowledges the pervasive influence of mobility on our cognitive processes, behaviors, and language use.

Despite possible negative ethical ramifications in mobile learning, such as cyberbullying, disinformation, and hate speech, the affordances for interactivity and multimedia communication facilitated by mobile technology hold significant promise for triggering agency. As emphasized by \textcite[p. 299]{andrews2011}, “mobile learning has the potential to subvert teacher dominance in the classroom and the didactic approaches to education which arose during the industrial era, such as the transmission model.” Instead, it has the potential to introduce a learner-centered paradigm, allowing students to exert more control over their individual learning processes.
\section{Agency}\label{sec-agency}

The concept of agency has been gaining prominence in Education, especially in Applied Linguistics, as it deals with issues aimed at possible actions by the learner, the teacher, the educator, or other participants in the educational community. It is present in discussions related to autonomy, identity, motivation among other constructs in this area. In his lecture at the 19th World AILA Congress in 2021, Benson, in a brief retrospective of discussions on agency, conflates autonomy and agency when considering “[a]utonomy: a more comprehensive capacity to purposefully control or direct agency.” We endorse the understanding that autonomy refers to the capacity for self-governance and self-direction and it may enable and trigger agency: the capacity of individuals to act intentionally to achieve specific goals.


Agency has a variety of definitions. One of the most-widely circulated definitions in Applied Linguistics is that of \textcite[p. 112]{ahearn2001} for whom agency is "the socioculturally mediated capacity to act".  \posscite{ahearn2001} discussions were considered a milestone in underscoring the sociocultural nature of the agency. Aligned with \posscite{ahearn2001} position, \textcite{lantolf2006} consider that agency is socioculturally mediated and dialectically exercised. This implies that within a specific timeframe and spatial context, certain limitations and affordances make some actions probable, possible, and even impossible. \textcite{stetsenko2020} advances by proposing the exercise of agency through a process in which agents are co-authors of the world and contribute to social transformation in a movement that involves individual and collective agency. \textcite{picard2010} emphasizes the need for new definitions of agency since modern definitions of this concept tend not to contemplate the interconnectivity between humans, non-humans, and environments.


According to \textcite[p. 1]{sang2020}, the notion of teacher agency is related to the ability to make active choices in educational practices. In his words:

\begin{quote}
    The notion of teacher agency is being used to describe the agentic capacity to make active choices in educational practices. Teacher agency refers to a teacher's competence to plan and enact educational change, direct, and regulate their actions in educational contexts. Becoming a crucial variable in studying teaching behavior and teacher education, researchers have investigated agency of student teachers, novice teachers, and experienced teachers to examine how agency affects learning to teach and acting to reform. In educational contexts, policy implementation and reform innovation rely on individual and/or collective agency of teachers.
\end{quote}

\textcite{sang2020} emphasizes that studies on teacher agency need to take account of the social structure, considering that teachers' actions are influenced by this structure, and may even be institutionally restrained. We agree with \textcite[p. 2]{sang2020} that “[i]n the ecological view, agency positions it within the contingencies of contexts in which agents act upon their beliefs, values, and attributes they mobilize in relation to a particular situation”.


Agency has also been discussed under the lens of complexity theory. For \textcite{mercer2012}, due to the complex nature of agency, conclusive and widely accepted definitions are difficult to find. \textcite{mercer2012} emphasizes the need to recognize and define agency from the data generated in an investigation. In one noteworthy case, \textcite[p. 42]{mercer2011} observed agency from two overlapping dimensions: the first points to the sense of the learner's agency, that is, “how agentic an individual feels both generally and in respect to particular contexts”, and the second concerns “learner's agentic behavior in which an individual chooses to exercise their agency through participation and action, or indeed through deliberate nonparticipation or non-action”. In the words of \textcite[p. 43]{mercer2012}, “[a]gency is therefore not only concerned with what is observable but it also involves non-visible behaviors, beliefs, thoughts and feelings; all of which must be understood in relation to the various contexts and affordances from which they cannot be abstracted”.


\textcite{larsen2019} also recognizes agency as a complex system and, supported by \textcite{mercer2011, mercer2012} and other researchers, asserts that agency is relational, emergent, spatially and temporally situated. It can be achieved, and it changes through iteration and co-adaptation; furthermore, it is multidimensional and heterarchical. For the purpose of this study, we explore possible dimensions of teacher agency considering the following characteristics:


\begin{itemize}
	\item Agency is relational
\end{itemize}


Agency is not inherent to the individual, that is, there is not something internally innate responsible for the observed behavior. Rather, “agency is interpellated from the self-organized dynamic interplay of factors internal and external to the system, persisting only through their constant interaction with each other” \cite[p. 65]{larsen2019}. In this sense, agency is relational and “is always related to possibilities in context and therefore inseparable from them, and possibilities, in turn, are ecological and not merely physical characteristics of the world, defined in terms of the relationship of 'systems' between the organism and its environment”.

\begin{itemize}
	\item Agency is emergent
\end{itemize}


\textcite{larsen2019} describes agency as a spontaneous activity connected to the world, forming coordinative structures that are units of selection in evolution and intentional change. Complex systems organize themselves harmoniously in functional synergies, sensitively adapting to the context and providing selective advantages. Agency emerges early in human life when children realize they can bring about change and understand their ability to transform the world. Drawing on \cite{kelso_self-organizing_2016}, \textcite[p. 65]{larsen2019} tells of a child who, at an early age, “becomes aware that it can make things happen; for example, by kicking its legs, it can make a mobile move or shaking its fist, a rattle sound”.

\begin{itemize}
	\item Agency is situated in space and time
\end{itemize}

Agency is influenced by the past, present, and future, responding to changing historical situations. Language learner agency is holistic, spanning a learner’s life story, past experiences, present and future goals. Its effectiveness occurs in the present, as observed by \textcite{biesta2007,mercer2012}.

\begin{itemize}
	\item Agency is Multidimensional
\end{itemize}

Agency goes beyond simple actions. As discussed by \textcite{lantolf2006,deters2015}, it involves the ability to assign relevance and meaning to various elements. Joana, a research participant in \posscite{mercer2012} study, exemplifies the multifaceted nature of agency, which is intertwined with intrapersonal aspects such as emotions, beliefs about language acquisition, self-image, personality, and motivation. Undoubtedly, these factors can collectively influence a person's agency.


In addition to these features,\textcite[p. 8]{mairitsch2023} also point out that agency can be both socially constructed and socially distributed. The social construct feature is related to “the dynamic interplay between a learner, their psychologies (in particular, how learners learn and interact with their environments), and their social contexts.” As for the socially distributed feature, \textcite[p. 5]{mairitsch2023} emphasize that learners’agency is not only socially determined, but also socially distributed. The authors \cite[p. 7]{mairitsch2023} claim that the distributed dimension of agency “is not just a characteristic of an individual. Rather, it can be viewed as a collective attribute promoted when groups work well together and feel empowered in a supportive and psychological safe climate”.


Although the features of agency in \textcite{mercer2012}, \textcite{larsen2019,mairitsch2023} certainly contribute to the discussions in this study, we share \posscite{mercer2012} thoughts that it is important to identify features of agency when examining data. With that in mind, we hope to contribute to the discussions on agency as an ecological and complex system by identifying some of the features discussed by both researchers, as well as other features yet to be identified when it comes time to analyze the data.

\section{The study}\label{sec-thestudy}

This is a qualitative study of an interpretive nature. According to \textcite[p. 16]{flick2009}, qualitative research "starts from the notion of the social construction of the realities under study and is interested in the perspectives of the participants, their everyday practices and their everyday knowledge related to the issue under study". As \textcite{bortoni2008} explains, the qualitative-interpretive approach is interested in detailing a specific situation and not in creating universal laws; it is interested in discovering "how patterns of social and cultural organization, local and non-local, relate to the activities of particular people as they choose how to carry out their social action" \cite[p. 41]{bortoni2008}.


Twenty pre-service teachers enrolled in a course dedicated to the integration of technological resources in language teaching at the Federal University of Minas Gerais (UFMG), participated in this study. Of all the participants, 18 are already working as language teachers. Of those who teach only one language, 11 teach only Portuguese and 3 teach English only. Of those who teach more than one language, 2 teach Portuguese and English and 2 teach Portuguese and Spanish.


The data for this study was generated through an e-questionnaire with prompts in Portuguese to encourage the writing of narratives about the use of mobile technologies in formal and informal contexts. We sought to understand how such technologies are used to create learning opportunities in these contexts. As this study focuses on the use of mobile technologies for learning and their relationship to agency, we selected the narratives generated by the following prompts:

\begin{enumerate}[label=\alph*)]
	\item Using cell phones to learn at university is... - Express your ideas and feelings. List the apps you use, explain why and share your experiences.
	\item Using cell phones to learn outside the classroom is... - Express your ideas and feelings. List the apps you use, explain why and share your experiences.
\end{enumerate}

Data analysis followed the parameters and procedures of qualitative research \cite{dornyei_research_2007}. Firstly, using the software for qualitative analysis Atlas TI, we searched for significant units \cite{holliday_doing_2016}, i.e., statements of the participants' exercise of agency. After repeated readings, we coded these statements according to their recurrences. We then created categories to group together instances of agency related to the affordances perceived in mobile devices.
\section{Data analysis}\label{sec-dataanalysis}

In this section we present the results of our data analysis and categorization process. We have therefore chosen to present the emerging patterns identified in the social spaces in which the participants circulate, underscoring patterns that seem to influence their agency in relation to mobile technologies. First, we present instances of agency in the academic lives of pre-service teachers based on the affordances perceived in mobile technology for the participants' routines as university students. Next, we discuss excerpts that show agency in their professional lives, in their daily lives as practicing teachers. Finally, we present some instances of agency that reinforce the ubiquity of mobile technology and show how it influences the participants' actions in social life.

\subsection{Academic Life}\label{subsec-academiclife}

As \textcite{mercer2012} points out, learners perceive affordances and act in meaningful ways as a result of their interactions with the environment. In the participants' narratives, the exercise of agency appears to be related to the affordances of mobile technology to access information, as the following excerpts illustrate.

It is necessary to use cell phones at the university because they allow students to access materials that are fundamental to their education. (P9)

Access to PDF texts also makes reading in class much easier, as we avoid excessive use of paper. I use adobe acrobat, adobe scan, siga… (P19)

Because of the lack of face-to-face and virtual teachers, I've tried to delve deeper into the content through video lessons and websites that provide supplemental materials. With the Internet, I was able to study and learn different interesting content, as well as search for different materials on the same content, but by different authors. (P13)

The use of cell phones to study outside of school is essential these days because it allows you to access a variety of websites and have information in the palm of your hand at all times. (P15)

[$\ldots$] Everyone has the autonomy to research what they want. We also have YouTube with lots of content, classes, lectures, etc. If you know what to research and use the right apps, you can learn a lot. (P10)

But it can also be harmful, because it's so easy to find information. I see the need to use more conventional methods as well. (P14)

Access to information, texts, content, and supplementary materials is an action that is frequently mentioned in the participants' narratives. In the data analyzed, the access to "materials that are fundamental to their [students'] education" (p.9) and "PDF texts" in the university context seem to be strongly associated with the affordances of mobile technology. In addition, the use of printed technology in place of digital technology also appears to facilitate reading in the classroom by mitigating the "excessive use of paper" (P19).

The use of mobile devices to find information from various sources also seems to be a common practice among the participants. They acknowledge the use of their mobile phones as a technology tool to "search for different materials on the same content" (P13), thereby enhancing their research consultations and enabling them to check facts or even validate different pieces of information on a given topic. This multiple and powerful search seems to empower learners in that it makes them perceive information as something that can be touched and/or controlled, something that is "always in the palm of [one’s] hand" (P15).

Affordances are “either for good or ill” \cite[p. 127]{gibson1986}. The ability to search through infinite sources is perceived both positively and negatively by different participants. While one person believes that it is possible to "learn a lot" from this type of autonomous research if well-informed and guided (P10), another believes that easy access to diverse information "can also be harmful" and that it would be necessary to "use more conventional methods as well" (P14). These excerpts not only reinforce the complementary relationship between agent and environment \cite{gibson1986}, but also emphasize that agency is intertwined with contextual possibilities, emerging through the self-organized interplay of factors both internal and external to the system \cite{larsen2019}. Given the sociocultural nature of agency, affordances or constraints can render an action (un)likely and (im)possible for a given agent depending on temporal and/or spatial contexts \cite{lantolf2006}.

Another instance of agency identified in the data from this study that seems to emerge in/from the academic context relates to the perceived affordance of mobile devices to study.

I usually use my tablet to study because I absorb knowledge through active reading and I find the tablet to be an excellent tool for this: I can highlight and take notes on theoretical texts. (P12)

I have a more traditional way of studying. I like paper, doodling, making diagrams and summaries. As a last resort, I can open a document in Word or Adobe, but I never use them to study. Or I do a quick Google search on my phone to look up the meaning of a word, check a simple piece of information, or look for a synonym when I write a text. (P6)

In delineating the affordance of studying, we highlight the conceptualization of mobile devices as mediating tools for engaging in study activities, such as facilitating "active reading," enabling the learner-user to "highlight and take notes on theoretical texts" (P12), for instance. It is important to underscore how, once again, mobile technology seems to empower the participant since, from his perspective, it would allow him not only to read, but also to "absorb knowledge" by interacting with texts. Highlighting and annotating are not actions strictly limited to digital environments and devices, as another participant who claims to have "a more traditional way of studying" and likes "paper, doodling" (P6) shows. From a complex perspective, agency is also related to intrapersonal aspects \cite{mercer2012}; in this case, to the learner's belief that one or the other technology (the mobile device or paper) would be more effective for studying.

For \textcite{larsen2019}, the learner's agency spans the learner's life experiences, their present and their future, and its effectiveness is perceived in the present. Interacting with mobile devices in the academic context seems to have an impact on participants' conceptions of time, making them realize that their phones enable them to speed time up.

A complement or extension and enhancement of activities and learning. It's through the cell phone that we do assignments, send files, print, read texts... The cell phone has become a quick tool that is always available to the student. (P3)

So I started using digital media more to take notes and do assignments. I believe that cell phones at university make it easier to search for information and speed up learning, if they are used prudently and correctly of course. (P16)

That mobile phones enable multiple actions seems to give new meaning to learners' notions of time when they implicitly establish a counterpoint between their previous and current academic experiences. The fact that one can "do assignments, send files, print, read texts" (P3) using a single device "that is always available to the student" means that the mobile phone is seen as an instrument that can "speed up" (P16) everyday actions in the life of a university student. In so far as actions are carried out more quickly, this is also perceived as improving and optimizing learning, further supporting the belief that faster equals more effective.

In addition to speed, the just-in-time aspect of mobile technology \cite{pegrum_mobile_2014,godwin-jones_smartphones_2017}, along with its ubiquity and multifunctionality, appears to offer the opportunity to maximize time.

Today, after going through this 100\% online experience, I see that cell phones and technology make it much easier to study, as they avoid spending time commuting to university, offer more comfort to study at home and make the process much easier. (P10)

I also organize my pending university activities on my tablet and cell phone using an agenda app so that I can make better use of time. (P12)

After going through a formal academic experience entirely online and contrasting their experiences, the participants show that they have realized that mobile devices allow them to better organize their routines, to the extent that they can, for example, "avoid spending time commuting to the university" (P10), allowing them to "make better use of time" (P12). In this sense, mobile technology has the potential to give a new meaning to time, allowing the participants, pre-service teachers to organize themselves and exercise their agency in order to study and fulfill their academic obligations satisfactorily.

Another instance of agency mediated by mobile phones and tablets in academic life seems to emerge from the perceived potential of these devices to provide opportunities for distraction.

In the classroom, cell phones are a potential source of distraction. That's the simplest analysis, but over the course of the pandemic I've come to change my mind. I have ADHD, but I no longer see the cell phone as a source of distraction, far from it, it is a possibility for autonomy and independence of research within the classroom. The teacher no longer needs to be the sole, authoritative source of information. I'm not even talking about checking whether what the teacher says is true or not, it's more about being able to expand tangents of knowledge in the middle of the lesson when opportunities arise. I really value students' autonomy in learning. (P5)

I mostly use my cell phone at university to get away from it all. I go on Instagram, Pinterest or OLX in search of visual stimulation. I find it hard to concentrate on one thing for long, so I end up using apps in class out of anxiety. I need these breaks. (P6)

Reflecting on the presence of cell phones in the classroom, one participant recognizes that a simplistic analysis could consider that "cell phones are a potential source of distraction" (P5). However, despite dealing with ADHD, the participant points out that the cell phone is a "possibility for autonomy and independence", allowing him to seize opportunities for research and expand his search for knowledge during class. In this sense, by carrying a device during class, the learner recognizes its capacity to foster engagement and connection with the learning process in an autonomous way, expanding his learning opportunities in the classroom and beyond the school walls.


Distraction is also perceived from another angle, as in the narrative of a participant who states that the use of cell phones at university is "mostly to get away from it" (P6). Again, the compression of time-space is perceived, as the participant emphasizes the potential of the cell phone to be in other places beyond the spatial context in which one is physically present. Because she feels anxious and has difficulty concentrating, the participant admits that she needs visual stimuli to stay in the classroom. In this way, the pre-service teacher seems to imply that using the cell phone to distract herself is important for her academic life.

The narratives of P5 and P6 seem to corroborate that the exercise of agency is influenced by intrapersonal factors, such as ADHD and anxiety, and is subject to adaptations and interactions with other systems. In the case of the two would-be teachers, there is interaction with the teacher, the classroom, and online environments. These issues bring us back to the idea that the exercise of agency is dynamic, open, and susceptible to change. The relational and emergent dimensions of agency are also present in the discourses of the participants in this study, since the exercise of agency emerges from interactions with factors, elements, and systems that make up the situated context of each of these participants. For example, we can mention interpersonal and intrapersonal factors, the relationships established with cultural artifacts (videos, PDF documents, mobile devices etc.) as well as systems such as the university, the classroom, online environments, and social media, to name but a few.

\section{Conclusion}\label{sec-conclusion}

The findings of this study suggest that the exercise of agency in initial education contexts is multifaceted. Like the findings of \posscite{mercer2011,mercer2012} empirical work focusing on language learners, the analysis of the narratives in this study indicates that agency is influenced by the intricate interconnectedness of various factors and elements coexisting in the participants' systems. It was possible to identify interpersonal factors, such as the perceived opportunities to interact with agents (human and non-human) in formal and informal online environments, as well as intrapersonal factors, such as the impact on emotions, attitudes, and beliefs about the best ways to learn. The data also show that the exercise of agency is dynamic, subject to change, and open, as it can be influenced by other systems.

Throughout the analysis, the relational nature of agency \cite{larsen2019} proved to be quite salient, as the data unveiled a reciprocal interaction between internal and external factors and the actions emerging from these relationships within contexts and their possibilities.

The results point to the potential of mobile devices in facilitating the exercise of agency among the participating pre-service teachers. These devices allow teachers to access information, speed up time, study, and even provide opportunities for distraction. When it comes to their praxis, these teachers also recognize the possibilities of mobile technology to motivate, bring the classroom to the 21st century, and engage learners. The potential for mobile technologies to impact social life was also evident from the data, as participants at various points in their stories emphasized how pervasive and important technology is to everyday life and citizenship.

In terms of the possible implications of this study for teacher education, one of the possible insights may stem from the recognition that in order to understand the possibilities of teacher agency – pre-service and in-service – it is important to consider the environments in which these agents circulate, the technologies and other agents with which they interact, the nested systems that make up their ecosystems, and the ways in which these dynamics affect and feed back into the interaction between intra- and interpersonal aspects.

We recognize the complexity of agency in the context studied, and although some dynamics and the complex fabric of teachers' agency have been highlighted in the data analyzed, further research can highlight other units of analysis in relation to inter- and intrapersonal aspects, elements, and systems that can further the understanding of the role of these "agents", thus contributing to research on teacher agency.

\printbibliography\label{sec-bib}

\begin{contributors}[sec-contributors]
\authorcontribution{Junia de Carvalho Fidelis Braga}[conceptualization,datacuration,formalanalysis,investigation,methodology,projadm,software,validation,visualization,writing,review]
\authorcontribution{Ronaldo Correa Gomes Junior}[conceptualization,datacuration,formalanalysis,methodology,validation,visualization,writing,review]
\end{contributors}

\end{document}



