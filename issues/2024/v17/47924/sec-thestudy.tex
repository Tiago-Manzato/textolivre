\section{The study}\label{sec-thestudy}

This is a qualitative study of an interpretive nature. According to \textcite[p. 16]{flick2009}, qualitative research "starts from the notion of the social construction of the realities under study and is interested in the perspectives of the participants, their everyday practices and their everyday knowledge related to the issue under study". As \textcite{bortoni2008} explains, the qualitative-interpretive approach is interested in detailing a specific situation and not in creating universal laws; it is interested in discovering "how patterns of social and cultural organization, local and non-local, relate to the activities of particular people as they choose how to carry out their social action" \cite[p. 41]{bortoni2008}.


Twenty pre-service teachers enrolled in a course dedicated to the integration of technological resources in language teaching at the Federal University of Minas Gerais (UFMG), participated in this study. Of all the participants, 18 are already working as language teachers. Of those who teach only one language, 11 teach only Portuguese and 3 teach English only. Of those who teach more than one language, 2 teach Portuguese and English and 2 teach Portuguese and Spanish.


The data for this study was generated through an e-questionnaire with prompts in Portuguese to encourage the writing of narratives about the use of mobile technologies in formal and informal contexts. We sought to understand how such technologies are used to create learning opportunities in these contexts. As this study focuses on the use of mobile technologies for learning and their relationship to agency, we selected the narratives generated by the following prompts:

\begin{enumerate}[label=\alph*)]
	\item Using cell phones to learn at university is... - Express your ideas and feelings. List the apps you use, explain why and share your experiences.
	\item Using cell phones to learn outside the classroom is... - Express your ideas and feelings. List the apps you use, explain why and share your experiences.
\end{enumerate}

Data analysis followed the parameters and procedures of qualitative research \cite{dornyei_research_2007}. Firstly, using the software for qualitative analysis Atlas TI, we searched for significant units \cite{holliday_doing_2016}, i.e., statements of the participants' exercise of agency. After repeated readings, we coded these statements according to their recurrences. We then created categories to group together instances of agency related to the affordances perceived in mobile devices.